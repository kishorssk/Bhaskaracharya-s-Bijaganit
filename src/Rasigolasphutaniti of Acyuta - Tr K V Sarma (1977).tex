राशिगोलस्फुटानीतिः 
[ग्रहणबीजम्]
\footnote{}इन्दुमार्गतिरश्चीनं द्विस्पृक्सूत्रं भवेद् यदा ।
तदैव परमासत्तिर्ग्रहएो सोमसूर्ययोः ॥ १ ॥
मध्यकाले तु न तथा द्विस्पृक्सूत्रस्य संस्थितिः ।
परमग्रासकालोऽतो भिद्यते मध्यकालतः ॥ २ ॥
त्र्प्रर्कमार्गतिरश्चीनां रेखां प्राप्नोति चन्द्रमाः ।
यदा, तदा स्फुटैक्यं स्याद् , इति केचन सूरयः ॥ ३ ॥
पाताद् यावतिथे भागे क्रान्तिवृत्ते रविर्भवेत् ।
विज्ञेपमएडले चन्द्रः पातात् तावतिथेऽन्तरे ॥ ४ ॥
यदा भवेत् , तदैव स्यात् पर्वान्त इति केचन ।
त्र्प्रर्कमार्गतिरश्चीनसूत्रात् पातानुसारतः ॥ ५ ॥
प्राग्वा पश्चाच्च भवति वर्त्मसाम्यं,\footnote{} ततोऽपि च ।
पातासन्नप्रदेशे स्यात् सन्निकर्षः परस्तयोः ॥ ६ ॥
व्यहीन्दावोजपदगे पातः पृष्ठगतस्तयोः ।
पुरोगतो युग्मपदे ; तस्मान्मध्याख्यकालतः ॥ ७ ॥
परमग्रासकालोऽयं प्राक्पश्चाच्च भवेत् क्रमात् ।
त्र्प्रासत्तिकालः पर्वान्ताद् भिन्नः पज्ञद्वयेऽप्यतः ॥ ८ ॥
भेदस्य तारतम्ये हि केवलं कलहो भवेत् ।
विविच्य नोक्तो भेदोऽयं यद्यप्यार्यभटादिभिः ॥ ९ ॥
तथापि युक्तिसिद्धत्वात् स्वीकृतो ह्यच्तुतादिभिः ।
[पर्वान्तनिर्णयः - प्रथमः पक्षः]
मध्यकाले ह्यर्कबिम्बात् स्फुटक्षेपान्तरे विधुः ॥ १० ॥
ततः प्रागथवा पश्चादर्कासत्तिर्भवेदिति ।
ज्ञातुं तत्कालसरएािर्विज्ञेया शीतदीधितेः ॥ ११ ॥
विधोस्तत्कालसरएािः कथं ज्ञेयेति चेच्छृएाु ।
द्युषष्ट्यंशजभूपृष्ठस्फुटगत्यन्तरं भुजम् ॥ १२ ॥
दिनषष्ट्यंशजस्पष्टक्षेपखएडः शराभिधः ।
तयोर्वर्गैक्यमूलं स्यात् तत्कालसरएािर्विधोः  ॥ १३ ॥
त्र्प्रर्कमार्गानुसारी स्याद् बाहुस्तद्व्यस्तदिक् शरः ।
चन्द्रमार्गात्मकः कर्एाः	प्रमाएाक्षेत्रमीदृशम् ॥ १४ ॥
मध्यक्षेपः श्रुतिर्वाहुः परमासतिकालजम् ।
बिम्बान्तरं, चन्द्रमार्गे तयोरग्रान्तरं शरः ॥ १५ ॥
इच्छाक्षेत्रमिदं; तत्र ज्ञातेन श्रवएोन हि ।
त्र्प्रज्ञातौ दोःशरौ नेयावनुपातेन ; तद्यथा ॥ १६ ॥
प्रमाएाक्षेत्रकर्एास्य यद्येतौ बाहुसायकौ ।
तदेच्छाक्षेत्रकर्एास्य कौ स्तो बाहुशराविति ॥ १७ ॥
इच्छाक्षेत्रगतौ स्यातां बाहुबाएाौ; तयोः शरः ।
गत्यन्तरात्मको ज्ञेयो यतोऽसौ चन्द्रमार्गगः ॥ १८ ॥
दिनषष्ट्यंशसम्बन्धिस्फुटगत्यन्तरात्मना । 
प्रमाएाक्षेत्रकर्एोन नाडिकैका भवेद् यदि ।
तदेच्छाक्षेत्रबाएो\footnote{} कः काल इत्यनुपाततः\footnote{} ॥ १९ ॥
परमग्रासकालस्य भेदः स्यान्मध्यकालतः ।
द्विःकृत्वो हरएां कार्यं कर्एोनैवं यतस्ततः ॥ २० ॥
मध्यक्षेपाच्छरहतात् कर्एाकृत्याप्तनाडिकाः ।
स्वर्एां कुर्यान्मध्यकाले तयोर्दिग्भेदसाम्पतः ॥ २१ ॥
परमग्रासकालोऽयं, मध्यक्षेपाद् भुजाहतात् ।
कर्एााप्तो बिम्बभेदः स्यात् परमग्रासकालजः ॥ २२ ॥
तद्वनं बिम्बयोगार्धं भवेद् ग्रासप्रमा परा\footnote{} ।
विधोरर्ककलाप्राप्तिः पर्वान्त इति यन्मतम् ॥ २३ ॥
तन्मते प्रोक्तमखिलं युज्यते नात्र संशयः ।
वर्त्मसाम्यं हि पर्वान्त इति पक्षे विधोर्गतिः ॥ २४ ॥
क्षेपवृत्तानुसारी स्यात् ततोऽस्या बाहुता कथम् ।
तर्हि गत्यन्तरांशस्य बाहुत्वं नेह कल्प्यते ॥ २५ ॥
किन्तु कर्एात्वमिति चेद् , भवत्वेवं शशिग्रहे ।
रविग्रहे तु कर्एात्वमपि तस्या न युज्यते ॥ २६ ॥
यतो नतिवशादिन्दोर्वर्त्म भिन्नं प्रतिक्षएाम् ।
त्र्प्रनुगत्यन्तरांशस्य बाहुत्वं भास्वतो ग्रहे ॥ २७ ॥
बाएात्वं नतिखएडस्य पक्षेऽस्मिन् यदि\footnote{} कल्प्यते ।
तदा\footnote{} तयोर्वर्गयोगमूलस्यात्रेन्दुमार्गता ॥ २८ ॥
इति चेन्न, यतो नात्र तयोर्दोःकोटिरूपता ।
तथा हि नतिखएडः स्याद्व्यस्तदिक्कोऽर्कमार्गतः ॥ २९ ॥
गत्यन्तरांशकश्चास्मिन् पक्षे स्यादिन्दुमार्गतः\footnote{} ।
त्र्प्रतोऽनयोर्मिथो न स्याद्दोःकोटित्वमिति स्यितम् ॥ ३० ॥
किञ्च लम्बनखएडेन घटिकाकालजेन च ।
ऊनितो गतिभेदांशो बाहुर्ज्ञेयः पुरोदितः ॥ ३१ ॥
तत्र लम्बनखएडोऽयं रविमार्गगतो भवेत् । 
गत्यन्तरांशकश्चात्र चन्द्रमार्गंगतो यतः ॥ ३२ ॥
ततोऽनयोर्वियोगोऽपि वस्तुतस्तु न युज्यते ।
एतद्दोषनिरासार्थं यद्यत्र नतिलम्बने ॥ ३३ ॥
तयोः खएडौ च नीयन्ते भास्करोक्तेन मार्गतः\footnote{} ।
तर्हि लम्बनखएडस्य गत्यन्तरलवेषु हि ॥ ३४ ॥
युक्तैव संस्कृतिः, किन्तु विक्षेपनतिखएडयोः ।
मिथो योगो वियोगो वा नोपपन्नो भवेत् तदा ॥ ३५ ॥ 
यतो नतिः क्षेपवृत्ततिरश्चीनैव तन्मते ।
क्रान्तिवृत्ततिरश्चीनो विक्षेपः सर्वसम्मतः\footnote{} ॥ ३६ ॥
एवं ह्युभयतः पाशारज्जुरत्रापतिष्यति ।
किञ्च गत्योर्वियोगोऽपि पक्षेऽस्मिन् नैव युक्तिमान्\footnote{} ॥ ३७ ॥
भिन्नमार्गगतत्वस्य तयोरप्यविशेषतः ।
``लग्नोनान्त्येन्दुदोःकोट्यो"\footnote{} रित्यादाविन्दुलग्नयोः ॥ ३८ ॥
वियोगोऽपि न युक्तोऽत्र यतस्तौ भिन्नवृत्तगौ ।
किञ्चात्र ``कृतनत्येन्दोः क्षिप्त्यानीतं स्थितेर्दलम् ॥ ३९ ॥
कृत्वा तत्काललम्बं च पर्वान्ते निर्एायस्तयोः"\footnote{} ।
इत्युक्तस्थितिदलस्यानीतिरपि न युज्यते ॥ ४० ॥
यतः स्थितिदलक्षेत्रगतिभेदांशरूपयोः ।
इच्छाप्रमाएायोर्नात्र वस्तुतस्तुल्यरूपता ॥ ४१ ॥
तस्माद्विधोरन्त्यभुक्तिरन्त्य\footnote{} स्फुट इति द्वयम् ।
क्रान्तिवृत्तानुसार्यैव स्वीकर्तव्यं रविग्रहे ॥ ४२ ॥
\footnote{}तथा च सूर्यशशिनोर्गत्योश्च स्फुटयोरपि ।
विश्लेषो युज्यते तद्वद् भुजाद्यानीतिरेव च ॥ ४३ ॥
क्रियाक्रमे त्वन्त्यभुक्तिरपि विक्षेपवृत्तगा ।
गह्यतेऽतो भजादीनामानीतिर्नातिसुन्दरा ॥ ४४ ॥
राशिगोलस्फुटैक्यं हि यदा स्यादर्कचन्द्रयोः ।
स एव कालः पर्वान्तो ग्राह्यः सूर्यग्रहे सदा ॥ ४५ ॥
त्र्प्रन्यथा लम्बसंस्कारः पर्वान्ते नैव युज्यते ।
क्रान्तिवृत्तकलारूपाः यतो लम्बनलिप्तिकाः ॥ ४६ ॥
[राशिगोलस्फुटसंस्कारः]
इन्द्वादीनामपि गतिः क्रान्तिवृत्तानुसारिएाी\footnote{} ।
त्र्प्रतो न क्षेपतो भेदः स्फुटस्यापीति यन्मतम्\footnote{} ॥ ४७ ॥
समलिप्ताकाल एव पर्वान्तस्तन्मते भवेत् ।
चन्द्रादयः क्षेपवृत्ते भ्रमन्ति सततं यतः ॥ ४८ ॥
ततः स्फुटोऽपि तेषां स्यात् स्वतोऽपि क्षेपवृत्तगः ।
इति पक्षे हि पर्वान्तः साम्यकालो न केवलम् ॥ ४९ ॥
त्र्प्रस्मिन् पक्षे हि चन्द्रस्य राशिगोलस्फुटाप्तये ।
स्फुटीकरएातः पश्चात् कार्यं यत्नान्तरं यतः ॥ ५० ॥
तत्प्रकारश्चाच्युतेन कीर्तितः स्फुटनिर्णये\footnote{} ।
``पातोनस्य विधोस्तु कोटिभुजयोर्जीवे मिथस्ताडये -
दन्त्यक्षेपशराहतं वधममुं विक्षेपकोट्या हरेत् ।
\footnote{}लब्धं व्यासदलोद्धृतं हिमकरे स्वर्एां, विपाते विधौ 
युग्मायुग्प्तपदोपगे ; विधुरयं स्पष्टो भगोले भवेत्" ॥ ५१ ॥
\footnote{} इत्यत्र वासना ज्ञेया कलास्वन्तरवद् बुधैः ॥ ५२ ॥
त्र्प्रयमेव हि संस्कारो लघूकृत्य क्रियाक्रमे ।
तेनैव ``क्षेपवीरांशं कुर्याद्" इत्यादिनेरितः\footnote{} ॥ ५३ ॥
राशिगोलस्फुटानीतिरच्युतेनैवमीरिता\footnote{} ।
क्रान्तिवृत्तगतिश्चैवमानेया गोलवित्तमैः ॥ ५४ ॥
[इत्यच्युतमतानुसारिणी राशिगोलस्फुटानीतिः ॥]

राशिगोलस्फुटसंस्कारविवेचनम् 
`पातोनस्य विधोस्तु' इत्यादिश्लोकस्य वासना 
पातोनस्य विधोस्तु कोटिभुजयोर्जीवे मिथस्ताडयेद् 
त्र्प्रन्त्यक्षेपशराहतं वधममुं विक्षेपकोट्या हरेत् । 
लव्धं व्यासदलोद्धृतं हिमकरे स्वर्एां, विपाते विधौ 
युग्मायुग्मपदोपगे; विधुरयं स्पष्टो भगोलेे भवेत् ॥
\footnote{}त्र्प्रत्र वासना प्राणकलान्तरवत् ज्ञेया । कथं पुनः राशिगोलस्फुटगत्यानयनमिति चेत् उच्यते । पातोनेन्दुं द्विगुणीकृत्य ततः कोटिज्यामानीय तया स्फुटगतिं हत्वा क्षेपकोट्या हरेत् । तत्राप्तं त्र्प्रन्त्यक्षेपशरेण हत्वा त्रिज्ययाप्तं फलं द्विघनस्य पातोनेन्दोः कर्किनक्रादिवशाद् गातौ धनमृणं वा कुर्यात् । तदा इन्दोः क्रान्तिवृत्तस्फुटगतिर्भवति । एवं सामान्येन इन्दोः राशिगोलगतस्फुटगत्योरानयनमुक्तम् ।
उपरागे तु गुणकारभूताया व्यहीन्दुकोट्या हारभूतायाः क्षेपकोट्याश्च प्रायेण विज्यातुल्यत्वनियमात् ताभ्यां न किञ्चित् कृत्यमस्ति । त्र्प्रतस्तदानीं त्र्प्रन्त्यक्षेपशरो गुणकारः । स च अच्युतमते प्रायेण द्वादशसंख्यः । तत्र व्यासार्धे द्वादशभिर्हृते सति हारकः स्यात् । स चाक्षरसंख्यया `सुन्दर' (२८७) इति भवति । तदेवं व्यहीन्दुदोर्ज्या `सुन्दरां' शस्फुटान्तरं गतिः, `सुन्दरां' शो गत्यन्तरमिति च स्थितम् ।
तत्र यदि विक्षेपात् स्फुटान्तरमानीयते तर्हि विक्षेपस्य व्यहीन्दुदोर्ज्या`रम्यां'शत्वात् `सुन्दररम्यां' शो हारको ग्राह्याः । स च चतुर्विंशतिसंख्यः । एतत्सर्वमभिप्रेत्य हि [अच्युकृते उपराग-] क्रियाक्रमे `पर्ववत् क्षेपवीरांशं' (I.42), `युगयुक्पदजात् क्षेपाद् वारांशाप्तं स्वमृणं क्रमात्' (II.6) इत्याद्युच्यत इति ध्येयम् ।
एतत् स्फुटान्तरसम्बन्धिकालानयने पुनरिदं त्रैराशिकम् - यदि गत्यन्तरांशतुल्याभिः षष्टिर्विनाड्यो लभ्यन्ते तदा `क्षेपवीरांश' तुल्याभिः कति विनाड्य इति । त्र्प्रत्र विक्षेपस्य षष्टिर्गुणकारः, चर्तुर्विशतिर्हारकः । एतौ द्वादशभिरपवर्तितौ क्रमात् पञ्चसंख्यो द्विसंख्यश्च भवतः । तदुक्तं -
युगयुक्पदजात् क्षेपात् पञ्चघ्नाद् द्विघ्नगतिभिदांशाप्ता ।
विघटीः स्वमृणं कुर्यात् पूर्णान्ते व्यत्ययेन दर्शान्ते ॥ 
(उपरागक्रियाक्रमः, I.41)
इति । त्र्प्रत्र हि शशिग्रहे नतिलम्बनयोरभावात् विक्षेपवृत्तरूपैवेन्दोस्तत्कालसरणिः । ततश्च त्र्प्रर्कबिम्बाक्रान्तक्षेपपार्श्वाख्यवृत्तं यदा चन्द्रः प्राप्नोनि तदैव परमग्रासः । स च व्यहीन्दौ युक्पदगे साम्यकालात् प्रोक्तविनाडिकाभिः पश्चात् स्यात् । त्र्प्रयुक्पदगे तु प्राक् ।
सूर्यग्रहणे तु पूर्वोक्तयुक्तिभिः इन्दोरर्ककलाप्राप्तिकाल एव पर्वान्तत्वेन ग्राह्याः । स च व्यहीन्दौ युकपदगे साम्यकालात् प्राक्, त्र्प्रयुकपदगे पश्चात् । त्र्प्रत उक्तं - `व्यत्ययेन दर्शान्ते' इति । तस्मात् वर्त्मसाम्यं हि पर्वान्त इति पक्षे (cf. राशिगोल., 4-6) विधोर्गतिः क्षेपवृत्तानुसारी स्यात् । `ततोऽस्या बाहुता कथम्' (राशिगोल., 25a) इत्यारभ्य क्रियाक्रममते सूर्यग्रहणे यानि दूषणान्युक्तानि तत्सर्वं निराकरणीयमिति यद्याग्रहः तर्हि प्रथमाध्यायान्ते -
सोऽयं संस्कृतशीतांशुराद्योऽन्त्यः समलिप्तिकः ।
तद्गतिर्निजसिन्दूरभागोना च रविग्रहे ॥
इति वक्तव्यं स्यात् । त्र्प्रस्मिंश्च पक्षे विक्षेपानयने -
व्यहीन्दुदोर्ज्या रभ्याप्ता लम्बहारकताडिता ।
युद्धार्थाप्ता युता क्षेपो लिप्तात्र्यंशविलिप्तया ॥
(cf., उपरागक्रियाक्रमः, I. 39)
इति पाठः कार्यः । लम्बनविनाडीसंस्कारे च -
इन्दोर्मध्यगतिघ्न्यः प्राक् लम्बहारविभाजिता ।
(cf., उपरागक्रियाक्रमः III 13)
इति पठनीयम् ।
किञ्च एवं इष्टग्रासानयने `केवलक्षेपवीरांश' संस्कारो न कार्यः । ततश्च तत्कालदृग्गतेर्भेदहाराप्ता लम्बनलिप्तिकाः कृत्वा इन्दौ तेन तत्तत्कालभानोः कार्यं स्फुटान्तरम् इत्येव इष्टग्रासविधौ वक्तव्यं स्यात् । चन्द्रग्रहणे तु विक्षेपवृत्तगतगत्योरन्तरमेव ग्राह्यम् । तच्च त्र्प्रर्कगति `सुन्दरां ' शयुतकेवलगत्यन्तरमेव । न चैवं स्थित्यर्धानयनानुपपत्तिः शङ्क्या, स्थित्यर्धकोट्या त्र्प्रपि क्षेपवृत्तगतया एव स्वीकारात् ।
ननु इन्दोः स्फुटं द्वयमपि स्वतो विक्षेपवृत्तगतमेव । ततश्चाद्येन्दोरानीतः क्षेपः क्रान्तिवृत्ततिरश्चीन एव स्यात् ।
इष्टाग्रगा त्विह वृत्तिद्वययोगमूला 
दोर्ज्येष्टमण्डलगता खलु कर्णरूपा ।
दोर्ज्याग्रकेऽन्यवृतिपार्श्वकयोस्तथैव 
स्पृष्टे तु मण्डल इहाभिमतान्तरालम् ॥
इति न्यायात् ।
एवं च इन्दुमार्गविक्षेपयोर्मिथो दोःकोटित्वाभावात् स्थित्यर्धकोट्या इन्दुमार्गगतत्वं न संभवत्येवेति चेन्मैवम् । न ह्यस्मिन् पक्षे मध्यकालजेन्दुस्फुटादानीतविक्षेपेण स्थित्यर्धकोटिरानीयते, येन उक्तदोषः स्यात् । किन्तु मध्यकालीनबिम्वान्तरेणैव । तच्च इन्दुमार्गविपरीतदिक्कमेवेति न किञ्चिदसमञ्जसम् ।
यद्यस्मिन् पक्षेऽपि सम्पर्कार्धमण्डलस्यार्कगत्या चलनात् तत्कालार्कात् पातं विशोध्यानीतक्षेपेणैव स्थित्यर्धमानेयमिति युक्तितः स्थित्यर्धस्याविशेषणं कार्यमेव । तथापि इह मध्यकालक्षेपात् स्पर्शमोक्षजार्कादानीतक्षेपयोर्भेदस्याल्पत्वात् ततोऽल्पतर एव स्थित्यर्धभेद इतीह त्र्प्रविशेषाकरणेऽपि न दोशः । एतदेव ह्यभिप्रेत्य क्रियाक्रमे चन्द्रग्रहणे स्थित्यर्धस्याविशेषणं नोक्तमिति ध्येयम् ।
एवमिष्टग्रासानयनेऽपि यदि तत्कालार्काद् विक्षेप त्र्प्रानीयते तर्हि स्फुटान्तरानयनार्थं चन्द्रे `क्षेपवीरांश'संस्कारो न कार्यः । ततश्च -
मध्येष्टकालविश्लेषनाडीर्गप्यन्तरांशकैः ।
हत्वा तद्वर्गयुक्तेष्टार्कोत्था इष्टक्षेपकृतेः पदम् ॥
बिम्बान्तरं तदूनं स्यान्मानैक्यार्धं तदा ग्रहः ॥
इत्येवमिष्टग्रासानयनकर्म स्यात् , त्र्प्रथवा `इष्टघ्नेनेष्टकालजम्' इत्यत्र `इष्टार्कात्त्विष्टजे ग्रहे' इति वा पाठः कार्यः । सोमयाज्युक्तपद्येष्वपि त्र्प्राद्यश्लोके `सगतीनानयेत् सायनांशान्' इत्यतः पश्चात् -
किन्त्वर्कस्योपरागे विधुचरमगति स्वात् सुदारांशहीना, 
प्राह्या चन्द्रोपरागे रविगतिसदरांशान्वितो भुक्तिभेदः ।
इत्युत्तरार्धं पाठ्यम् ।
पुनः `नाभिघ्ना'दित्यादि द्वितीयश्लोकः । तत्रापि -
चाल्यः सूर्योपरागे चरमहिमकरः सूक्ष्मपर्वान्तकाला-
दाद्येन्दुः साम्यकालादुभयमपि पुनः ग्लौग्रहे साम्यकालात् ।
इत्युत्तरं बोध्यम् ।
ननु ग्रहणगणने `पर्वान्तः', `समलिप्ता' चेति द्वौ पदार्थावेव लेखनीयत्वेन प्रसिद्धौ । इह तु `साम्यकाल'स्यापि लेखनीयत्वात् गौरवमिति चेत् न । यथास्थितपाठेऽपि तस्य लेख्यत्वात् । क्रियाक्रमे तु चन्द्रोऽपि स्फुटपर्वान्ते नीत इति यद्यपि साम्यकालेन न प्रयोजनं, तथापि तत्कालार्कस्यापि ज्ञेयत्वेन लेखनेन लाघवातिशयः । प्रत्युत `क्षेपवीरांश'संस्काररूपं क्रियागौरवमधिकमिति दिक् ।
नन्वेवमपि गतिसंस्कारोऽत्नाधिक इति चेत् न । इष्टग्रासे व्यहीन्दुदोः `सुन्दरांश'संस्कारस्येह त्र्प्रकरणोयत्वेन लाघवस्यापि सत्त्वात् । तस्मान्न किञ्चिद् गौरवम् । त्र्प्रत एव हि लघुभास्करीयव्याख्याने पारमेश्वरे इन्दुमार्गगतस्थित्यर्धक्षेत्रसिद्ध्यर्थं स्थित्यर्धानयने मध्येष्टविक्षेपयोरेकदिग्गतत्वे त्र्प्रन्तरवर्गं, भिन्नदिक्त्वे तयोर्योगवर्गं च विक्षेपवर्गहीनसम्पर्कार्धकृतौ प्रक्षिप्य मूलीकरणं कार्यमित्युक्तम् । (Cf. लघुभा. व्याख्या. IV.12, ed., Anandasrama Skt. Ser., No 128, Poona, 1946, p. 51).

ऽ\footnote{}
