\documentclass[twoside,12pt,openany]{book}
श्रीनारायणपण्डितविरचितः 
बीजगणितावतंसः 
एकमनेकस्योक्त (नित्यं) व्यक्तस्य गुणवतो जगतः ।
गणनाविधेश्च बीजं ब्रह्म\footnote{ब्रह्मे} च गणितं च तद् वन्दे ॥ १ ॥
अजगोलोऽयमियानिति करकलितामलकसन्निभो येन ।
व्यक्तिचक्रे ह्यगणितगणितेन (च) तत् किमस्त्यन्यत् ॥ २ ॥
गणितमिति नाम लोके ख्यातमभूदगणितस्य शास्त्रस्य । 
अगणितविक्रमविष्णोस्त्रिविक्रमश्चेति नामेव ॥ ३ ॥
सद्गुरुफृपयाऽनुभवैरभ्यासैः परमतत्त्वमिव योगी ।
यो वेत्ति कर्म साङ्ख्यं स भवति सङ्ख्यावतां धुर्यः ॥ ४ ॥
यो यो यं यं प्रश्नं\footnote{यं यः प्रश्नः ।} पृच्छति सम्यक्करणं न तस्यास्ति ।
व्यक्तेऽथाव्यक्ते तु प्रायस्तत्करणमस्त्येव\footnote{मस्सेव ।} ॥ ५ ॥
व्यक्तक्रियया ज्ञातु प्रश्ना न खिलीभिवन्ति\footnote{ज्ञानुं प्रश्नान्यखी भ ।} नाल्पधियः ।
बीजक्रियां च तस्माद् वच्मि व्यक्तां सुबोधां च ॥ ६ ॥

[बिजोपयोगि-गणितम् ]
(१) षट्त्रिंशत् परिकर्माणि 
(i) धनर्णषड्विधम 
धतर्ण\footnote{धनार्ण ।}सङ्कलिते करणसूत्रमार्याद्वयम् -
रूपाणामव्यक्तानां नामाद्यक्षराणि लेख्यानि ।
उपलक्षणाय तेषामृणगानामूर्ध्वबिन्दूनि ॥ ७ ॥
योगे धनयोः क्षययोर्योगः स्यात् स्वर्णयोर्भवेद् विवरम् \footnote{वर्णयो ।}।
अधिकादूनमपास्य च शेषं तद्भावमुपयाति ॥ ८ ॥
उदाहरणम् -
रूपत्रयञ्च रूपकपञ्चकमस्वं धनात्मकं वाऽपि ।
वद सहितं झटिति सके स्वर्णमृणं स्वं च यदि वेत्सि\footnote{स्वर्णमृणं च यदि वेसि ।} ॥ १ ॥
न्यासः - रू ३ रू ५ । अत्र धनयोर्योगे योग इति योगे जातं रू ८ ।
न्यासः - रू ३ रू ५ । ऋणयोर्योगे योग इति जातं योगे रू ८ ।
न्यासः - रू ३ रू ५ । स्वर्णयोर्विवरमिति जातमृणभावं शेषं रू २ । अयं योग एव ।
न्यासः - रू ३ रू ५ । स्वर्णयोर्विवरमिति जातं धनभावं शेषं रू २ । अयं योग एव । एवं भिन्नेष्वपि ।
इति धरर्णसङ्कलनम् ।
धनर्णव्यवकलने सूत्रमार्यार्धम् -
स्वमृणत्वमृणं स्वत्वं शोधकराशेः समुक्तवद्योगः ।
उदाहरणम् - रूपाष्टकं रूपकपञ्चकेन\footnote{रूपकं प} 
क्षयं क्षयेनापि धनं धनेन ।
धनं क्षयेण क्षयगं धनेन 
व्यस्तं च संशोध्य वदाशु\footnote{वदाश्रु ।} शेषमु ॥ २ ॥
श्रीनारायणपण्डितविरचितः बीजगणितावतंसः 
न्यासः - रू ८ रू ५ । अत्र शोधके\footnote{अत्र शोध अत्रशोघके ।} ऋणं स्वत्वमिति\footnote{स्वमृणत्वमिति ।} जातं स्वं रू ८ रू ५ । प्राग्वद्योगे जातं रू ३ । एतच्छेषम् ।
न्यासः - रू ८ रू ५ । स्वमृणत्वमिति जातमृणत्वं रू ८ रू ५ । प्राग्वद्योगे जातं रू ३ ।
न्यासः - रू ८ रू ५ । ऋणं स्वत्वमिति जातं स्वं रू ८ रू ५ । प्राग्वद्योगे जातं रू १३ ।
न्यासः - रू ८ रू ५ ।\footnote{रू ८ रू ५ ।} स्वमृणत्वमिति जातमृणत्वं रू ८ रू ५ । प्राग्वद्योगः रू १३ । एतदन्तरम् ।
इति धनर्णव्यवकलना । 
अथ धनर्णगुणने सूत्रमार्यार्धम् -
ऋणयोर्धनयोर्घाते स्वं स्यादृणधनहतावस्वन् ॥ ९ ॥
उदाहरणम् -
रूपद्वयं रूपकपञ्चकेन 
धनं धनेन क्षयगं क्षयेण ।
धनं क्षयेण क्षयगं धनेन 
निघ्नं पृथक् किं गुणने फलं स्यात् ॥ ३ ॥
न्यासः - रू २ रू ५ गुण्यगुणकौ\footnote{गुणागुणकौ ।} । धनयोर्घाते स्वं स्यादिति जातं रू १० ।
न्यासः - रू २ रू ५ । ऋणयोर्घाते\footnote{योद्याते ।} स्वं स्यादिति जातं धनं रू १० ।
न्यासः - रू २ रू ५ । ऋणधनहतावस्वमिति जातं रू १० ।
न्यासः - रू २ रू ५ । प्राग्वज्जातमृणं रू १० । एवं भिन्नेष्वपि ।
इति धनर्णगुणना ।
धनर्णभागहारे सूत्रम् -
ऋणधनगुणने यच्चोपलक्षणं तच्च भागहरणेऽपि ।
उदाहरणम् -
द्विनिघ्नरूपत्रितयं द्विकेन 
धनं धनेनर्णमृणेन भक्तम् \footnote{धनं धनेनर्णम् ।} । 
ऋणं धनेन स्वमृणेन वापि 
सखे वदाश्वत्र\footnote{दाश्च ।} हृतौ फलं मे ॥ ४ ॥
न्यासः - रू ६ रू २ । अत्र गुणेन\footnote{नेन ।} यच्चोपलक्षणमिति यथा धनयोर्घाते धनं तथा धनयोर्भजने धनमिति भागे हृते जातं रू ३ ।
न्यासः - रू ६ रू २ । भागे हृते जातं रू ३ ।
न्यासः - रू ६ रू २ । गुणनवद् भागे हृते जातं रू ३ ।
न्यासः - रू ६ रू २ । भागे हृते जातं रू ३ ।
इति धनर्णभागहारः ।
धरर्णवर्गवर्गमूलयोः करणसूत्रम् -
ऋणधनयोश्च कृतिः स्वं धनमूलं धनमृणं भवेद्वापि ।
अकृतित्वादृणराशेर्मूलं नास्त्येव सिद्धमिति ॥ १० ॥
उदाहरणम् -
सखे चतुर्णामधनात्मकानां\footnote{मधुना ।}
धनात्मकानाञ्च\footnote{अनात्मकाना धनात्मका ।} कृतिं वदाशु \footnote{दाश्रु ।} ।
धनस्य रूपद्विगुणाष्टकस्य 
क्षयस्य वा मित्र पृथक्पदं किम् ॥ ५ ॥
न्यासः - रू ४, रू ४ । जातौ वर्गौ १६, १६ ।
न्यासः - रू १६ । जातं मूलं रू ४, अथवा मूलं रू ४ ।
न्यासः - रू १६ । अस्य क्षयगतस्य राशेरकृतित्वान्मूलं नास्तीति सिद्धम् ।
इति धनर्णवर्गमूले \footnote{मुले ।} ।
इति धनर्णषङ्विधम् ।

श्रीनारायणपण्डितविरचितः बीजगणितावतंसः 
(ii) शून्यषड्विधम् 
शून्यसङ्कलितव्यवकलितयोः करणसूत्रम् -
स्वर्णं\footnote{स्वर्ण ।} शून्येन युतं विवर्जितं वा तथैव तद् भवति ।
शून्यादपनीतं तत् स्वर्णं व्यत्यासमुपयाति \footnote{समुत्प ।} ॥ ११ ॥
उदाहरणम् -
रूपपञ्चकमृणं धनं सखे 
खेन युक्तमथवा विवर्जितम् ।
शून्यतः पृथगपास्य तानि वा 
किं भवेद् गणक मे पृथग्वद ॥ ६ ॥
न्यासः\footnote{सेः न्यो ।} - रू ५, रू ५ । एतानि खेन युतान्यूनितान्यविकृतानीव ।
न्यासः - रू ५, रू ५ । एतानि शून्यतशच्युतानि जातानि व्यवस्तानि\footnote{ब्यक्तानि ।} रू ५ रू ५, ।
इति शून्यसंकलनव्यवकलने ।
शून्यगुणने सूत्रमार्यर्धम् \footnote{The Portion enclosed within [] does not form part of the text. It is added to complete the text.} -
खं राशिना विगुणितं खं स्याद्राशिः खगुणश्च खं भवति ।
उदाहरणम् 
धनर्णभूतैस्त्रिभिरेव सङ्गुणं
खं किं फलं स्यात्कथयाशु तन्मे ।
धनात्मकाश्चाप्यधनात्मकास्त्रयः 
खसंगुणाश्चापिफलं प्रचक्ष्व ॥
न्यासः - गुण्यः रू ०, गुणकः रू ३, रू ३ । गुणने जातमुभयोः फलम् रू ० ।
न्यासः - गुण्यः रू ३, रू ३ । गुणकः रू ० । गुणने जातमुभयोः फलम् रू ० ।
इति शून्यगुणनाविधिः ।
कल्पिताः । यावत्तावत् -कालक-नीलक-पीतक-लोहितक-हरितक-चित्रक-क्रपिलक-पाटलक-पाण्डुरक-धूम्रक-शबलक-श्यामलक-मेचक-धवलक-पिशङ्गक-सारङ्गक-वभ्रुक-गौरक इत्याद्या वर्णाः, अथवा वर्णाः कादयः, अथवा मधुरादयो\footnote{मूधु ।} रसपर्यायाः, अथवा असदृशप्रथमाक्षरनामपदार्थाः कल्प्यन्ते । एषु समजात्योर्वहूनां वा योगवियोगौ कार्यौ । अददृशजात्योर्बहूनां वा वर्णानां पृथक् स्थितिः स्यात् । तेषां पर्यायानामप्युक्तानामृणधन - योगाद्युपलक्षणं\footnote{ते पर्यायषामद्युक्तानां ऋणधनयोगाद्रूपल ।} रूपवद् भवतीति ।
उदाहरणम् -
अव्यक्तषट्कं च धनं सरूप -
मव्यक्तयुग्मञ्च विपञ्चरूपम् ।
किमेतयोरैक्यमृणं\footnote{कि मेनयो ।} धनञ्च 
तद्व्यस्तयोः सङ्कलन वदाशु\footnote{दाश्रु ।} ॥ ९ ॥ 
न्यासः - या ६ रू १ 
			या २ रू ५ 
समजात्योः स्वस्थाने योग इति जातं तथा न्यस्ते या ६ या २, रू १ रू ५ । ऋणधनयोः रूपवदुपलक्षणमिति\footnote{दुपलक्षणलक्षण ।} योगे जातं या ८ रू ४ ।
आद्यपक्षस्य ऋणत्वं प्रकल्प्य न्यासः -
या ६ रू १ 
या २ रू ५ 
योगे जातं या ४ रू ६ ।
द्वितीयपक्षस्य वैपरीत्यं कृत्वा न्यासः -
या ६ रू १ 
या २ रू ५ 
(योगे जातं) या ४ रू ६ 
उभयोर्व्यत्यासे न्यासः - या ६ रू १ 
							 या २ रू ५ 
योगे जातं या ८ रू ४ ।
श्री नारायणपण्डितविरचितः बीजगणि वितंसः 
उदाहरणम् -
अव्यक्तवर्गद्वितयं सरूप-
मव्यक्तयुग्मेन युतं च किं स्यात् ।
अव्यक्तषट्कं क्षयगं सरूपं 
शोध्यं तु षडूरूपसुसंयुतेभ्यः\footnote{पसयु ।} ॥
अव्यक्तकेभ्यो गणक ! प्रचक्ष्वा -
ष्टाभ्योऽवशेषं यदि वेत्सि\footnote{बेसि} बीजम् ।१० ।
प्रथमोदाहरणे न्यासः -
याव २ या ० रू १
याव ० या २ रू ०
योगे जातं याव २ या २ रू १ ।
द्वितीयोदाहरणे न्यासः -
या ८ रू ६ 
या ६ रू १ 
(वि) योगे जातं या १४ रू ५ ।
इत्यव्यक्तसंकलनव्यवकलने ।
अथाव्यक्तगुणने करणसूत्रम् -
स्याद्रूपवर्णघाते\footnote{स्याडूप ।} वर्णो, द्वित्र्यादितुल्यजातिबधे ।
तत्कृतिघनादयः स्युः\footnote{स्यु ।}, तद्भावितमसमजातिवधे ॥ २१ ॥
गुणकारसमुत्थानि स्वजातिखण्डानि योजयेदेव\footnote{देबे ।} ।
अव्यक्तवर्गकरणीगुणनासु व्यक्तवज्ज्ञेयम्\footnote{वद्ज्ञे ।} ॥ २२ ॥
उदाहरणम् -
यावत्तावद्द्वितयसहितं रूपषट्कं विनिघ्नं 
यावत्तावत्त्रितयरहितै\footnote{द्द्वित ।} रूपकैः पञ्चभिश्च\footnote{श्वंच ।} ।
गौणं किं स्याद् वद मम फलं हे सखे ! कल्पयित्वा 
व्यक्ते वर्णे पटुरसि यदि त्वं गुणाकारमार्गे ॥ ११ ॥
न्यासः -
गुण्यः या २ रू ६ 
गुणकः या ३ रू ५ 
व्यक्तवद्गुणिते जातं याव ६ या ८ रू ३० ।\footnote{व्यक्तं वर्ण । }
गुणकस्य धनर्णयोर्व्यत्यासे न्यासः -
या २ रू ६ 
या ३ रू ५ 
गुणिते जातं याव ६ या ८ रू ३० ।
गुण्यस्य व्यत्यासे न्यास -
या २ रू ६
या ३ रू ५ 
गुणिते जातं याव ६ या ८ रू ३० ।\footnote{याव ६ या ८ रू ३० ।}
गुण्यगुणकयोर्व्यत्यासे न्यासः - 
या २ रू ६ 
या ३ रू ५ 
गुणिते जातं याव ६ या ८ रू ३० ।
इत्यव्यक्तगुणना ।
श्रीनारायणपण्डितविरचितः बीजगणितावतंसः 
अव्यक्तभागहारे सूत्रम् -
शुद्ध्यति यर्यैर्वंर्णै रूपैर्भाजको\footnote{र्यै र्यैवर्णैरूपै भा ।} हतो भाज्यात् ।
क्रमशः स्वेषु स्वेषु स्थानेषु फलानि तानि स्युः ॥ २३ ॥
गुणनफलस्य (= भाज्यस्य) गुण्यभागहारस्य भार्गाथ\footnote{गार्थ ।} न्यासः -
(i) याव ६ या ८ रू ३० 
या २ रू ६
(ii) याव ६ या ८ रू ३०
या २ रू ६
(iii) याव ६ या ८ रू ३० 
या २ रू ६ 
(iv) याव ६ या ८ रू ३० 
या २ रू ६ 
भागे हृते जातो गुणः या ३ रू ५, या ३ रू ५, या ३ रू ५,, या ३ रू ५ ।
इत्यव्यक्तभागहारः ।\footnote{गहरः ।}
(अव्यक्तवर्गे उदाहरणम् -)
अव्यक्तानां रूपपञ्चोनितानां 
षण्णां वर्गं वा युतानां प्रचक्ष्व ।
चेद् बीजज्ञोऽसि त्वमस्याः कृतेश्च 
मूलं विद्वन् ! ब्रूहि तन्मे पृथक् किम् \footnote{पृथः क्किं} ॥१२ ॥
न्यासः -
या ३ रू ५
या ३ रू ५
स्थाप्योन्त्यकृति\footnote{स्थाथात्य ।} र्द्विसङ्गुणाऽन्त्यगुणेत्यादिना जातौ वर्गी -
याव ३६ या ६० रू २५
याव ३६ या ६० रू २५ 
(अव्यक्त वर्गमूले करण) सूत्रम् -
मुलान्यादायादौ वर्गेब्यस्तद्द्वयोर्द्वयोर्घातम्\footnote{द्वयोद्बयोधातं} ।
द्विगुणं\footnote{द्विगुण ।} जह्याच्छेषान् मूलमितीदं वदन्तीह ॥ २४ ॥
पूर्ववर्गमूलार्थ\footnote{पूर्ववर्गभूलार्यं ।} न्यासः -
याव ३६ या ६० रू २५ 
याव ३६ या ६० रू २५
एतयोर्मूले या ६ रू ५ । या ६ रू ५ ।
इत्यव्यक्तवर्गमूले ।
इत्यव्यक्तषड्विधम् ।
(iv) वर्णषड्विधम् 
उदाहरणम् -
यावत्तावत्त्रितयमधनं कालकाः षट् क्षयं भोः 
विद्वन् ! नीलाप्टकमपि धनं पीतकाः स्वं च पञ्च ।
रूपाढ्यैस्तैर्द्विगुणितमितैस्तेऽपि युक्ता वियुक्ताः 
जानासि त्वं यदि झटिति मे ब्रूहि वर्णाः कति स्युः ॥ १३ ॥
तैस्तैर्हता\footnote{तैस्तै हृता ।} कथय किं गुणने फलं स्याद् 
भक्त\footnote{भुक्त ।} च तद् गणकवर्य\footnote{गुणकवर्ग ।} गुणेन तेन ।
गुण्यस्य मे कथय वर्गमतश्च मूलं 
चेद् वर्णषड्विधविधाववधिं गतोऽसि ॥ १४ ॥
न्यासः -
या ३ का ६ नी ८ पी ५ रू १ 
या ६ का १२ नी १६ पी १० रू २ 
योगे जातं - या ६ का १८ नी २४ पी १५ रू ३ । वियोगे जातं - या ३ का ६ नी ८ पी ५ रू १ ।
इति वर्णसङ्कलनव्यवकलने ।
(गुणने) न्यासः -
गुण्यः या ३ का ६ नी ८ पी ५ रू १ 
गुणकः या ६ का १२ नी १६ पी १० रू २
श्री नारायणपण्डितविरचितः बीजगणितावतंसः 
गुणिते जातं - याव १८ याका ७२ यानी ६६ याणी ६० या १२ काव ७२ कानी १६२ कापी १२० का २४ नीव १२८ नीपी १६० नी ३२ पीव ५० पी २० रू २ ।\footnote{याव १८ याकाभा १२ यानी ६६ पापी ६० या १२ काव ७२ कानी १६२  कापी १२० का २६ नीव १२८ नीपी १६० नी ३२ पीव ५० या २० रू २ । } एत एव यथाक्रमेण न्यस्ता जाताः याव १८ काव ७२ नीव १२८ पीव ५० याका ७२ यानी ६६ यापी ६० कानी १६२ कापी १२० नीपी १६० या १२ का २४ नी ३२ पी २० रू २ ।
(गुण्येन) भक्ते जातो गुणकः या ६ का १२ नी १६ पी १० रू २ ।
इति वर्णगुणनभजने ।
वर्गार्थं न्यासः - या ३ का ६ नी ८ पी ५ रू १ ।
उक्तवज्जातो वर्गः - याव ६ याका ३६ यानी ४८  यापी ३० या ६ काव ३६ कानी ६६ कापी ६० का १२ नीव ६४ नीपी ८० नी १६ पीव २५ पी १० रू १ । यथाक्रमं न्यासः - याव ६ काव ३६ नीव ६४ नीपी ८० नी १६ पीव २५ पी १० रू १ । यथाक्रमं न्यासः - याव ६ काव ३६ नीव ६४ पीव २५ याका ३६ यानी ४८ यापी ३० कानी ६६ कापी ६० नीपी ८० या ६ का १२ नी १६ पी १० रू १ ।
अस्माद् वर्गाज्जातं\footnote{वर्गज्जातं ।} मूलं - या ३ का ६ नी ८ पी ५ रू १ ।
इति वर्णवर्ग-वर्गमूले ।
इति वर्णषड्विधम् ।
(v) करणीषड्विधम् ।
अथ करणीषड्विधम् -
मूलं ग्राह्यं राशेर्यस्य तु\footnote{मूले ग्राह्यो राशेर्यस्य नु ।} करणीति नाम तस्य स्यात् ।
सङ्गुणनं भजनं वा कुर्याद् वर्गस्य वर्गेण ॥ २५ ॥
लध्व्या वापि महत्या पृथक् करण्यो हृते च तत्पदयोः\footnote{चेत्पदयोः ।} ।
युतिवियुति\footnote{युतिविद्युती ।} कृती च तया गुणिते योगान्तरे भवतः ॥ २६ ॥
गुणिते वापि करण्यावनल्पया वाऽल्पया च तत्पदयोः ।
युतिवियुतिकृती भक्ते\footnote{भक्तै ।} ह्यभीष्टया योगविवरे स्तः\footnote{योगविवरस्त ।} ॥ २७ ॥
अथवा लध्व्या महतीं भक्त्वैतन्मूलमेकयुक्तोनम्\footnote{भक्तेकं तन्मूलयुक्तो ।} ।
स्वध्नं लध्व्या गुणितं युतिवियुती स्तो महत्यैवम् \footnote{महत्येव ।} ॥ २८ ॥
ख्पवदपि च करण्योर्घातपदेन द्विसङ्गुणेन\footnote{न हिस ।} युतिः । 
युक्तोना\footnote{युक्तो वा ।} युतिवियुती पृथक्स्थििति\footnote{पृथवित्स्थितिः ।} स्यान्न घातपदम् \footnote{घात्प ।} ॥ २९ ॥
करणीनां\footnote{णानां } तु बहूनां योेगे केनापि राशिना छित्वा ।
तन्मूलयुतिः स्वघ्ना छेदगुणा\footnote{च्छेद} स्याद्युतिस्तासाम् ॥ ३० ॥
उदाहरणम् -
षटसिद्धसङ्ख्ययोर्योगविशेषौ\footnote{संख्योर्यो ।} वद मे द्रुतम् ।
करएयोर्द्वित्रिमित्योश्च योगशेषे तयोर्वद ॥ १५ ॥
न्यासः - क ६ क २४ । अत्र लघ्व्या ६ भक्ते १ । ४ अथवा महत्या भक्ते  १४ । १ । मूले १ । २ वा १२ । १ । युतिवियुती ३ । १ वा ३२ । १२ । कृती ९ । १ वा ९४ । १४ । लघ्व्या गुणिते ५४ । ६ महत्या वा ५४ । ६ । जाते योगान्तरे\footnote{तेर्यो ।} क ५४ क ६ ।
द्वितीयप्रकारेण -
क ६ क २४ । एते\footnote{क ६ क २४९ ते ।} महत्या २४ गुणिते १४४ । ५७६ लघ्व्या वा ३६ । १४४ अनयोर्मूले\footnote{मुले ।} १२ । २४ वा ६ । १२ । युतिवियुती ३६ । १२ वा १८ । ६ । कृती १२९६ । १४४ वा ३२४ । ३६ । महत्या भक्ते लघ्व्या\footnote{लघ्या ।} वा भक्ते जाते योगान्तरे क ५४ क ६ ।
अथवा तृतीयप्रकारः -
क ६ क २४ । लघ्व्या ६ महती महत्या वा लघ्वी भक्ता ४ वा १४ । । मूलं २ वा १२ एकयुतमूनं च ३ । १ वा । ३२ । १२ । स्वघ्नं ९ । १ वा ९४ । १४ । लघ्व्या ६ महत्या २४ वा गुणिते\footnote{भक्ते ।} जाते योगान्तरे क ५४ क ६ ।
अथ चतुर्थप्रकारः-
क ६ क २४ । एते रूपाणि प्रकल्प्य रू ६ रू २४ । घातपदेन १२ द्विगुणेन\footnote{घातपदे १२ तद्वि ।} २४ करण्योर्युतिः ३० युतोना जाते योगान्तरे करण्यौ क ५४ क ६ ।
द्वितीयोदाहरणे करण्यौ क २ क ३ । अनयोर्घाते मूलाभावात् पृथक् स्थितिरिति योगे जातं क २ क ३ । अन्तरे च क २ क ३ ।
इति करणीसङ्कलनव्यवकलने ।
करणीगुणनादौ सूत्रम् - 
करणीनां (च) बहूनां यासां संयोगसम्भवोऽप्यस्ति ।
तासां योगं कृत्वा कार्य गुणनादि वा कर्म ॥ ३१ ॥
क्षयरूपकृतिः क्षयगा भवेद्यदा सा\footnote{भवेघतस्क ।} प्रयाति करणीत्वम् ।
क्षयगतकरणीमूलं रूपत्वं क्षयगतं भवति ॥ ३२ ॥
उदाहरणम् -
षड्रूपाढ्या गएाक । करग्रीपञ्चसङ्ख्या च गुएयो
द्वेऽष्टौ पञ्च-प्रमितकरणीखएडसङ्ख्या गुणाश्च ।
षड्रूपोने शरनखमिते वा गुएो किं फलं स्यात् 
तद्गुएयाप्तां वद गुएामितिं प्रौढता चेत्तवास्ति ॥ १६ ॥
प्रथमन्यासः\footnote{प्रथमं न्यासः ।}-
गुण्यः क ३६ क ५ ।
गुमकः क ८ क ५ क २ ।\footnote{गुणकः क ८ क ५ क ५ ।}
गुणिते जातं क २८८ क १८० क ७२ क ४० क २५ क १० । एतास्वनयोः क २८८ क ७२ योगे जातं क ६४८ । पुनरनयोः क ४० क १० योगे जातं क ६० । पुनरस्याः क २५ मूलं रू ५ । यथाक्रमं न्यासः - रू ५ क ६४८ क १८० क ६० ॥
अथवा लघुकर्मणा - गुण्यः क ३६ क ५ ।
गुणकः क ८ क ५ क २ ।
द्विकाष्टमित्योर्योगे कृते जातो गुणकः क १८ क ५ ।
गुणिते जातं तदेव रू ५ क ६४८ क १८० क ६० ।
द्वितीयोदाहरणे -
गुण्यः क ३६ क ५ ।
गुणकः क ३६ क २० क ५ ।
गुणिते जातं क १२६६\footnote{क भ ६६ ।} क ७२० क १८० क १८० क १०० क २५ । एतास्वासां\footnote{क २५६ ता ।} क १२६६ क १०० क २५ मूलानि रू ३६ रू १० रू\footnote{क ।} ५ । योगः रू २१ । पुनरनयोः क १८० क १८० योगः ० । इमामस्यां क ७२० संयोज्य जातं क ७८० । यथाक्रमं न्यासः - रू २१ क ७२० । वा लघुकर्मणि वा 
गुण्यः क ३६ क ५ ।
गुणकः क ३६ क २० क ५ ।
नखशरितयोर्योगे कृते जातो गुणकः क ४५ । क ३६ ।
गुणिते जातं तदेव रू २१ क ७२० ।
अपि च -
रूपद्वयाढ्यकरएीद्वितयेन निघ्ना\footnote{विघ्ना ।}
दन्तस्मृतीभयुगलप्रमिताः\footnote{दंतस्मृतीनयुगुलं ।} करएयः ।
किं स्यात्कलं कथय तत् त्वरितं\footnote{यत्त्वरितं ।} करएया\footnote{करिष्या ।} 
निघ्ना भुजङ्गयमलोन्मितयाऽथवा ताः ॥ १७ ॥
प्रथम\footnote{प्रथमं ।}न्यासः -
गुण्यः क ३२ क १८ क ८ क २ ।
गुणकः क ४ क २ ।
श्री नारायणपण्डितविरचितः बीजगणितावतंसः
गुणिते जातं क १२८ क ७२ क ६४ क ३६ क ३२ क १६ क ८ क ४ । एताष्वासां क ६४ क ३६ क १६ क ४ मूलनाि रू ८ रू ६ रू २ योगे रू २० । पुनरेताः (क १२८ क ७२ क ३२ क ८ ) द्विकेन छिन्नाः क ६४ क ३६ क १६ क ४, आसां मूलयुतिः २० स्वघ्ना ४०० पूर्वच्छेदेन २ गुणिता जातः\footnote{जाते ।} अवर्गकरणीनां योगः क ८०० । यथाक्रमं न्यासः रू २० क ८०० ।
अथवा गुण्यकरणीवां योगः क २०० । गुणकः क ४ क २ । गुणितं जातं तदेवं रू २० क ८०० ।
द्वितीयोदाहरणे 
गुण्यः क ३२ क १८ क ८ क २ ।
(गुणकः क८ क२ ।)
गुणिते योगे च कृते जातं रू ६० ।
इति करणीगुणनम् ।
करणीभागहारे सूत्रम् -
छेदकरण्यो निखिलाः कत्यपि वा वर्गंसिद्धये हन्यात् ।
तासां मूलसमासो रूपसमानो यथा भवति ॥ ३३ ॥
लब्धकरण्यो गुणकस्तद्गुणहारं त्यजेद् भाज्यात् ।
शुद्धिर्न भवेद्यदि वा तद्गुणकच्छेदकरणीनाम् ॥ ३४ ॥
योगमपि भाज्यराशेर्जह्यात्सदृशखण्डे च ।
रूपाभावे हारो येन निघ्नः\footnote{येनघ्नः ।} शुध्यति तत्फलं स्यात् ॥ ३५ ॥
पूर्वगुणफलस्य स्वगुणच्छेदस्य\footnote{णछेद ।} भागार्थ न्यासः -
(i) रू ५. क ६४८ क १८० क ६० । छेदः क ३६ क ५ ।
(ii) रू २१ क ७२० । छेदः क ३६ क ५ ।
(iii) रू २० क ८०० । छेदः क ४ क २ ।
(iv) (रू ६० । छेदः क १८)
प्रथमोदाहरणे ``छेदकरण्यो निखिलाः कत्यपि वा वर्गसिद्धये हन्यादि"ति छेदकरणी क ३६ क ५, एतयोरियं (क) ५ वर्गसिद्धये करणीपञ्चगुणिता (क) २५, अस्या मूलं ५ रूपसमं, रू ५ एतत्समं करण्योर्गुणः । क ५ एतद्गुणं हरं रू ५ क १८० भाज्याद् विशोध्य शेषं क ६४८ क ६० । ``रूपाभावेे हारो येनघ्नः शुध्यति तत्फलं स्यादि"ति अष्टादश करण्या गुणं हरं भाज्याद्विशोध्य (निःशेषं जातं) लब्धं क १८ इथि लब्धो गुणः क ५ क १८ ।
द्वितीयोदाहरणे\footnote{क २५ क १८ द्वितोयो ।} षठ्त्रिशत्करणीमृणषट्त्रिंशत्करण्या क ३६ पञ्चकरणीं पञ्चचत्वारिंशत्करण्या (क) ४५ सङ्गुण्य जातं क १२६६ क २२५ मूलैक्यं रूपसमं, करण्यौ क ३६ क ४५ एतद्गुणं हरं भाज्याद् विशोघ्य लब्धं करणी क ३६ क ४५ ।
तृतीयोदाहरणे रू २० क ८०० छेदः क ४ क २ । अत्र द्बिकरणीं द्विशतकरण्या क २०० सङ्गुण्य जातं क ४०० मूलं रू २० । द्विशत्या करण्या गुणं हरं भाज्याच्छोध्य लब्धं करणी क २०० ।
चतुर्थोदाहरणे रू ६० छेदः क १८ । अत्र द्विशतकरण्या\footnote{द्विशति ।} गुणितं हरं भाज्याद्विशोध्य लब्धं करणी\footnote{करणी करणी } क २०० ।
यथाक्रमं लब्धकरण्यः क १८ क ५ । क ३६ क ४५ । क २०० । क २०० ।
करणीविश्लिषणे सूत्रम् -
योगजकरणीं कृत्या कयाऽपि शुद्धर्यथा भवेद्\footnote{भजेत् ।} विभजेत् ।
तत्खण्डानि स्वगुणानि\footnote{स्वगुणितानि ।} लब्ध्या हतानि च करण्यः ॥ ३६ ॥
पूर्व खण्डत्रयमासीदिति प्रथमलब्धकरणी क १८ क ५ । अत्र योगजकरणी १८ त्रिवर्गेण ६ हृता लब्धं क २ । त्रयाणां (खण्डे) रू २ रू १ स्वघ्ने क ४ क १ लब्ध क २ गुणिते जाते करणीखण्डे क ८ क २ । एवं लब्धं क ८ क ५ क २ ।
द्वितीयोदाहरणे लब्धं क ३६ क २० क ५ ।
तृतीयस्ये खण्डचतुष्टयमासीदिति इयं क २०० । दशवर्गाच्छताल्लब्धं क २ । दशानां खण्डानि रू ४ रू ३ रू २ रू १ एतानि स्वगुणानि क १६ क ६ क ४ क १ लब्ध क २ हतानि क ३२ क १८ क ८ क २ जाता लब्धकरण्यः ।
चतुर्थोदाहरणेप्येता एव क ३२ क १८ क ८ क २ ।
अपि च -
बाणाग्नयः खदहनाः शशिलोचनानि\footnote{य खदहना राशि ।}
वस्विन्दवोऽब्धिशशिनो यमलैन्दवश्च ।
खण्डानि तानि करणीशरताडितानि 
द्वित्रीषुखण्डविहृतानि सखे फलं किम् ॥ १८ ॥
न्यासः -
भाज्यः क १७५ र १५० क १०५ क ६० क ७० क ६० ।
भाजकः क ५ क ३ क २ ।
``रूपाभावे हारो येन निघ्नः (शुध्यति तत् ) फलं स्यात्" इति प्राग्वद् भागे\footnote{भावे ।} हृते जातं क ३५ क ३० । 
अथवाऽन्यथोच्यते -
छेदेऽभीष्टकरण्या \footnote{क्षयधनता ।}ऋणधनताव्यत्ययोऽसकृत्कार्यः ।
भाज्यहरौ\footnote{भाज्यहारौ ।} सङ्गुणयेद्यावच्छेदे करण्यैका ॥ ३७ ॥
विभजेत्तया करण्या भाज्योद्भूताः करण्यश्च ।
लब्धा योगजकरणी चेत् स्याद्विश्लेषणं प्राग्वत् ॥ ३८ ॥
प्रथमोदाहरणे भागार्थ न्यासः -
(भाज्यः) क २५ क ६४८ क १८० क ६० ।
छेदः क ३६ क ५ ।
``छेद\footnote{छिदे ।}ऽभीष्टकरण्या ऋणधनताव्यत्ययोऽसत्कृकार्य" इति छेदकरण्योः पञ्चमिताया ऋणत्वं प्रकल्प्य क ३६ क ५ अनया हते भाज्ये योगे च कृते जातं (क) १७२६८ क ४८०५ छेदके च क ६६१ । अनया हृते भाज्ये लब्धो गुणकः क १८ क ५ । प्राग्वद् विश्लेषणे जातं क ८ क ५ क २ ।
द्वितीयोदाहरणे न्यासः -
रू २१ क ७२० । छेदः क ३६ क ५ ।
अत्र छेदे पञ्चमितकरण्या ऋणत्वं प्रकल्प्य भाज्ये गुणिते योगे च कृते जातं रू १८६ क ४३२४५ छेदे च  क ६६१ । अनेन\footnote{अन्येन ।} भाज्ये हृते विश्लेषसूत्रेण पृथक्कृते च जातं क ३६ क २० क ५ ।
तृतीयोदाहरणे न्यासः -
क ४०० क ८०० । छेदः क ४ क २ ।
अत्र चतुःकरण्या ऋणत्वं प्रकल्प्य (भाज्ये गुणिते योगे च कृते जातं क ८०० । छेदे च क ४ । अनेन भाज्ये हृते ) विश्लेषणे (च) जातं क ३२ क १८ क ट क २ ।
चतुर्थे न्यासः -
क ३६०० । छेदः\footnote{छे ।} क १८ ।
``यावच्छेदे करण्यैके" ति स्वयमेवैका\footnote{वैक । } तया\footnote{तया हृते ।} भाज्ये हृते लब्धं क २०० विश्लेषिते जातं क ३२ क १८ क ८ क २ ।
अनन्तरोक्तोदाहरणस्य न्यासः -
क १७५ क १५० क १०५ क ६० क ७० क ६० ।
छेदकः क ५ क ३ क २ ।
अत्र छेदकरणीषु द्विकरण्या ऋणत्वं प्रकल्प्य जातं क ५ क ३ क २ । अनया भाज्ये गुणिते योगे च कृते जातं क २१०० क १८०० क १२६० क १०८० । छेदके च क ३६ क ६० । अत्र षट्त्रिंशत्करण्या ऋणत्वं प्रकल्प्य जातं क ३६ क ६० । अनया भाज्ये (क २१०० क १८०० क १२३० क १०८० ) गुणिते योगे च कृते जातं क २०१६० क १७२८० छेदे च जातं क ५७६ । अनया भाज्येऽस्मिन् क २०१६० क १७२८० हृते लब्धं क ३५ क ३० । करणीखण्डेषु ययोर्ययोर्योगसम्भवोऽप्यस्ति तयो (र्तयो) र्योगं कृत्वा गुणन-भजन-वर्ग-वर्गमूलानि कार्याणि ।
इति करणीभागहारः ।
अथ करणीवर्गे सूत्रम् -
धनगतकरणीनां वा क्षयगानां तत्समानरूपाणि ।
स्युर्धनगतानि वर्गे शेषाः स्वमृणोपलक्ष्यास्ताः ॥ ३९ ॥
अन्तरकरणीवर्गे सम्प्रप्तेऽपि च तयोर्धनक्षययोः ।
व्यस्तं स्वर्ण सुधिया कार्य वर्गस्तयोस्तुल्यः ॥ ४० ॥
उदाहरणम् -
वर्गं करण्योर्द्विकराममित्यो-
स्त्रिषङ्द्विकानां\footnote{षट्विका ।} च पृथग्वदाशु\footnote{वदाश्रु ।} ॥
सप्तद्विपञ्चत्रिकसम्मिताना-
मङ्गेषुरामाक्षिमहीमितानाम् ।
दन्तस्मृतीभद्विकसम्मितानां\footnote{दत ।}
चेद् वेत्सि विद्वन् करणीविधानम् ॥ १९ ॥
न्यासः - अलघुपूर्वकं न्यस्ताः क ३ क २ । क ३ क ३ क २। क ७ क ५ क ३ क २। क ६ क ५ क ३ क २ क १ । क ३२ क १८ क ८ क २ । जाता यथाक्रमं वर्गाः रू ५ क २४ । रू ११ क ७२ क ४८ क २४ । रू १७ क १४० क ८४ क ५६ क ६० क ४० क २४ । रू १७ क १२० क ७२ क ४८ क २४ क ६० क ४० क २० क २४ क १२ क ८ । पञ्चमोदाहरणे क ३२ क १८ क ८ क २ योगे जातं क २०० अस्य वर्गः रू २०० ।
उदाहरणम् -
पञ्चत्रिमितकरण्योरन्तरवर्गं वदाशु\footnote{वर्ग्गं वगाश्रु ।} मे विद्वन् । 
पड्चत्रिद्विमितानांं स्वस्वर्णानामृणर्णधनगानाम्\footnote{मृणार्ण ।} ॥ २० ॥
न्यासः - क ५ क ३ । एतयोरन्तरं क ५ क ३ वा क ५ क ३ । अनयोर्वर्गस्तुल्य एव रू ८ क ६०\footnote{रू ८ रू ६० ।} ।
द्वितीयस्य न्यासः - क ५ क ३ क २ वा क ५ क ३ क २ । अनयोर्वर्गस्तुल्य एव रू १० क ६० क ४० क २४ ।
इति करणीवर्गः\footnote{बर्ग्गः ।} ।
करणीवर्गमूले\footnote{वर्ग्गा ।} सूत्रम् -
करणीवर्गे नियमः सङ्कलितमितानि (खण्डकानि ) स्युः ।
एककरण्या वर्गे रूपाण्येव द्वयोः सरूपा च ॥ ४१ ॥
तिसृणां तिस्त्रश्च तथा षडपि चतुर्णां दशापि\footnote{षडपि च चतुर्णां दशपि च ।} पञ्चानाम् ।
षण्णामपि पञ्चदशेत्येवं ज्ञेयानि खण्डानि ॥ ४२ ॥
सङ्कलितात्मकमूले\footnote{मूल ।} तन्मितखण्डैक्यतुल्यरूपाणि ।
रूपकृतेः प्रोज्भय\footnote{प्राेह्य ।} पदं तेनोनयुतानि रूपाणि ॥ ४३ ॥
दलिते करणीखण्डे सन्ति करण्यः कृतौ\footnote{कृताः ।} शेषाः ।
महती रूपाणि तयोः प्राग्वत्साध्येऽपरे खण्डे ॥ ४४ ॥
सङ्कलितात्मकमूलाभावे खण्डेषु तेषु खण्डानि ।
विश्लेष्य यथा मूलप्राप्तिः स्यादन्यथैवासत् ॥ ४५ ॥
अथ प्रकारान्तरमाह -
अथवा सर्वकरण्यश्चतुर्विभत्ता न्यसेदनल्पायाः ।
आद्यासन्नकरण्योर्हतिराद्याप्ता च तत्पदं करणी ॥ ४६ ॥
ते एव तया भक्ते करणीखण्डे परे भवतः ।
क्रमशस्त्तैरपि खण्डैः शेषाः भक्ताः परा करणी ॥ ४७ ॥
तैरपि मुहुः सलब्धैः शेषाः करणीर्भजेत् प्राग्वत् ।
पदकरणीवर्गयुतिं विशोधयेद् रूपवर्गेभ्यः\footnote{भाज्यरूपेभ्यः ।} ॥ ४८ ॥
एवं कृते तु न यदा तदा भवेद्योगजा करणी ।
विश्लेषसमुत्पाद्याः करणीवर्गे करण्योऽन्याः\footnote{करण्योन्या । } ॥ ४९ ॥
पूर्वसिद्धवर्गाणां\footnote{पूर्बसिद्धिव ।} प्रथमो वर्गः रू ५ क २४ । अत्रैकमेव करणीखण्डम् । अत्र परिभाषितम् -
करणीखण्डमितिर्या\footnote{मितिर्वा ।} द्विगुणा रूपांघ्रियुक्तया\footnote{क्तता ।} मूलम् ।
\footnote{रूपक ।}रूपदलेन१२ विहीनं सङ्कलितपदं भवत्येव\footnote{वत्यैव ।} ॥ ५० ॥
करणीखण्डं (१) द्विगुणं २ रूपाड्घ्रि १४ युक्तं ९४ मूलं ३२ रूुपार्धेन १२ विहीनं १ एतदेव सङ्कलितपदम् । एतन्मिता करणी क २४ रूपकृतेः\footnote{कृते ।} २५ अपास्य शेषमूलं १ अनेनोनयुतानि रूपाणि ४ । ६ दलिते करणीखण्डे\footnote{खण्डे ४२ क ३ ।} क २ क ३ । एत एव मूलकरण्यौ क ३ क २ । 
द्वितीयवर्गस्य (न्यासः) - रू ११ क ७२ क ४८ क २४ । अत्र करणीखण्डमितिः ३ द्विगुणा\footnote{द्विगुणः ।} ६ रूपाङ्घ्रि १४ युता २५-४ मूलं ५२ रूपदलेन १२ विहीनं सङ्कलितपदम् २ । एतन्मिते द्वे करणीखण्डे अऽटचत्वारिंशत्\footnote{शति ।} चतुर्विशतिकरणी ४८ । २४ तुल्यानि रूपाणि ७२ रूपकृतेः १२१ अपास्य शेषं ४९ मूलं ७ अनेन रूपाणि ११ ऊनयुतानि ४ । १८ दलिते जाते करणीखण्डे क २ क ९ । अत्र महती रूपाणि प्रकल्प्य न्यासः - रू ९ क ७२ । द्विसप्ततितुल्यानि रूपाणि रूपकृतेः ८१ प्रोज्झ्य पदं ३ ऊनयुतरूपाणाम् ६ । १२ अर्धे\footnote{अद्धे ।} ३ । ६ यथाक्रमं न्यासः - क ६ क ३ क २ ।
अथवा - द्वसप्तति-चतुर्विशतिमितकरणीतुल्य क ७२ क २४ रूपाणि ९६ रूपकृतेः १२१ प्रोज्झ्य २५ पदं ५ ऊनयुतरूपाणां ६ । १६ अर्द्ध\footnote{अद्धे ।} जाते मूलकरण्यौ क ३ क ८ । अत्र महती रूपाणि प्रकल्प्य रू ८ क ४८ रूपकृतेः ६४ अष्टचत्वारिंशत् करण्या\footnote{अष्टाविंशतिक ।}स्तुल्य रूपाण्यपास्य १६ मूलं ४ ऊनयुत ४ । १२ रूगणामर्धे जाते मूलकरण्यौ क २ क ६ । यथाक्रमं न्यासः - क ६ क ३ क २ ।
अथवा - द्विसप्तत्यष्टचत्वारिशत्करण्यास्तुल्यानि रूपाणि १२० (रूपकृतेः १२१ ) अपास्य १ मूलं १ ऊनयुतरूपाणा १० । १२ मर्धे जाते मूलकरण्यौ क ५ क ६ । अत्र महतीत्युपलक्षणं\footnote{क्षणः ।} क्वचिदल्पाऽपि रूपाणि प्रकल्प्य न्यासः - रू ५ क २४ । प्राग्वज्जाते मूलकरण्यौ क २ क ३ । यथाक्रमं न्यासः - क ६ क ३ क २ ।
तृतीयवर्गस्य न्यासः - रू १७ क १४० क ८४ क ६० क ५६ क ४० क २४ । अत्र वर्गे षट्पञ्चाशत्-चत्वारिंशत्-चतुर्विशतिकरणीतुल्यानि क ५६ क ४० क २४ रूपाणि रूपकृते २८९ रपास्य\footnote{ते १८९ र ।} शेषं १६९ मूलं १३ ऊनाधिकरूपाणा ४ । ३० मर्धे २ । १५ जाते मूलकरण्यौ क २ क १५ । अत्र महती रूपाणि (प्रकल्प्य) रू १५ क १४० क ८४ क ६० । अत्र रूपकृतेः २२५ चतुरशौतिषट्ति\footnote{षष्ठि ।} करण्यास्तुल्यानि रूपाण्यपास्य शेषमूलं ९, ऊनयुतरूपाणामर्धे जाते मूलकरण्यौ क ३ क १२ । महती रूपाणि (प्रकल्प्य) रू १२ क १४० । शेषकरण्यास्तुल्यानि (रूपाणि ) रूपकृतेरपास्य शेषं ४ मूलं २ ऊनयुतरूपाणामर्धे जाते मूलकरण्यो क ५ क ७ । मूलकरणीनां यथाक्रमं न्यासः - क ७ क ५ क ३ क २ ।
चतुर्थवर्गस्य न्यासः - रू १७ क १२८ क १२० क १०८ क ९६ क ६० क ४० क २० । अत्र करणीखण्डानि सप्त वर्तन्ते ७ द्विगुणानि १४ रूपचतुर्थाशयुतं\footnote{र्थांस ।} ५७-४ (अवर्गत्वात् ) अस्य सङ्कलितपदाभावः । अतः स्वबुद्ध्या इयं क १२८ चतुर्वर्गेण\footnote{चतुर्थ ।} भक्का लब्धं ८ चतुर्णां समे खण्डे ३ । १ स्वघ्ने ९ । १ पूर्वलब्ध्या ८ गुणिते करण्यौ क ७२ क ८ । खण्डाष्टके सङ्कलितपदाभावः । अतः करणीयं क ९६ द्विवर्गेण ४ भक्ता लब्धं २४ द्विकस्य समे खण्डे १ । १ पूर्वलब्ध्या २४ गुणिते करणीखण्डे क २४ क २४ । खण्डनवकेऽपि सङ्कलितपदाभावः । अतः करणीयं क १०८ षड्वर्गेण भक्ता लब्धं ३ षण्णां समे खण्डे ४ । २ अनयोर्वर्गौ १६ । ४ पूर्वलब्ध्या गुणितौ जाते करणीखण्डे क ४८ क १२ । यथाक्रमं न्यस्ते जातं रू १७ क १२० क ७२ क ६० क ४८ क ४० क २४ क २४ क २० क १२ क ८ । अत्र करणीखण्डदशकं वर्तते । दशानां संकलितपदम् ४ । इदं खण्डसङ्ख्या । खण्डचतुप्टयस्यास्य क २४ क २० क १२ क ८ करण्यास्तुल्यानि रूपाणि रूपकृते\footnote{कृते २८९ ।} २६९  अपास्य मूलं १५ ऊनयुतरूपाणामर्धे करण्यौ क १ क १६ । महती रूपाणि रू १६ । क १२० क ७२ क ६० क ४८ क ४० क २४ षण्णां करणीनां सङ्कलितपद ३ मितखण्डत्रयस्यास्य क ४८ क ४० क २४ तुल्यानि रूपाणि रूपकृतेरपास्य शेष १४४ मूलं १२ ऊनयुतरूपार्धे\footnote{युतंरूपार्द्धे ।} करणीखण्डे क २ क १४ । पूर्ववन्महती रूपाणि प्रकल्प्य रू १४ क १२० क ७२ क ६० करणीत्रये करणीद्वयस्यास्य क ७२ क ६० करण्यास्तुल्यानि रूपाणि रूपकृतेरपास्य मूलं ८, ऊनयुतरूपाणामर्धे करणीखण्डे क ३ क ११ । पुनर्महती रूपाणि रू ११ क १२० करणीतुल्यानि रूपाणि रूपकृतेरपास्य मूलं १ ऊनयुतरूपाणामर्धे जाते करणीखण्डे क ५ क ६ । मूलकरणीनां यथाक्रमं न्यासः - क ६ क ५ क ३ क २ क १ । 
पञ्चमवर्गस्य न्यासः -रू ६० क २३०४ क० १०२४ क ५७६ क २५६ क १४४ क ६४ । अत्र करणीषट्के करणीत्रयस्यास्य क २५६ क १४४ क ६४ करण्यास्तुल्यानि रूपाणि रूपकृते ३६०० रपास्य शेषमूलं ५६ ऊनयुतरूपाणारमर्धे करण्यौ क २ क ५८ । महती रूपाणि प्रकल्प्य रू ५८ क २३०४ क १०२४ क ५७६ करणीत्रये करणीद्वयस्यास्य क १०२४ क ५७६ करण्यास्तुल्यानि रूपाणि रूपकृते ३३६४ रपास्य मूलं ४२ ऊनयुतरूपाणामर्द्धे मूलकरण्यौ क १८ क ३२ । करणीनां यथाक्रमं न्यासः क ३२ क १८ क ८ क २ ।
अथवा - वर्गराशे रूपं हित्वा सर्वकरणीभ्यो\footnote{सर्वकरणीभ्यां ।} मूलानि रू ६० रू ४८ रू ३२ रू २४ रू १६ रू १२ रू ८ एषामैक्यं वर्गराशिः रू २०० । ``मूलं ग्राह्यं राशेर्यस्य तु करणीति नाम तस्य स्यादि" ति करणीत्वात् (मूलं क २०० विश्लेषणे क ३२ क १८ क ८ क २ ।
अथ द्वितीयप्रकारेण मूलानयनम् । प्रथमवर्गस्य न्यासः - रू ५ क २४ । अत्रैककरणीखण्डत्वाच्चतुर्भक्ता रू ५ क ६ आद्यासन्नद्वयहतिरित्याद्यासन्नकरणीखण्डे न स्तः, करण्योर्घाते षट् ६ रूपाणि प्रकल्प्य रू ६, योगे पञ्च\footnote{पुंच}, ``योगकृतेश्चतुराहतघातोनायाः पदं विवरमि" त्यन्तरं रू १, ``योगे द्विष्ठेऽन्तरयुतहीने\footnote{द्विष्ठे १ तरयुतहीनस् ।} तावर्धितौ च राशी स्त" इति सङ्क्रमणेन जाते खण्डे ३ । २ । एत एव मूलकरण्यौ क ३ क २ इति सङ्गक्रमणेन सिध्यति ।
द्वितीयवर्गस्य न्यासः -रू ११ क ७२ क ४८ क २४ । चतुर्भक्ताः करण्यः\footnote{ण्य ४ ।} रू ११ क १८ क १२ क ६ । अत्राद्यासन्ने क १२ क ६, हतिः ७२, आद्यया भक्के ४, मूलं करणी क\footnote{भक्ते क ४ मू २ ।} २ । अनया त एव करण्यौ क १२ क ६ भक्ते जाते त एव मूलकरण्यौ क ६ क ३ । मूलकरणीनां यथाक्रमं न्यासः - क ६ क ३ क २ । (आसां वर्गाः क ३६ क ९ क ४ सर्वासां योगः क १२१ रूपकृतौ १२१ विशोधयेत् ।)
तृतीयवर्गस्य न्यासः - रू १७ क १४० क ८४ क ६० क ५६ क ४० क २४ । चतुभेक्ताः रू १७ क ३५ क २१ क १५ क १४ क १० क ६ । आद्यासन्ने क २१ क १५, हतिः ३१५, आद्यया भक्ते ९ मूलं\footnote{मूूवं} ३ इयं करणी क ३ । अनया त एव करण्यौ क २१ क १५ भक्ते जाते करण्यौ क ७ क ५ । पूर्वया सह करणीत्रयं क ७ क ५ क ३ । एभिः खण्डैः (शेष) करण्यो भाज्या\footnote{करण्योऽनल्पादिभिर्भाज्या ।} इति (शेषकरणीनामधो ) न्यस्ते जातं क १४ क १० क ६
क ७ क ५ क ३ 
(भक्ते लब्धं करणी क २ ) । ययाक्रमं मूलकरण्यः\footnote{मूव ।} क ७ क ५ क ३ क २ । आसां वर्गाः क ४९ क २५ क ९ क ४ । सर्वासां\footnote{सर्बासां ।} योगः क २८९ रूपकृतौ २८९ विशोधयेत् ।
चतुर्थवर्गस्य पूर्ववद् विश्लेष्य चतुर्भक्तकरणीनां न्यासः रू १७ क ३० क १८ क १५ क १२ क १० क ६ क ६ क ५ क ३ क २ । अत्राद्यासन्ने क १८ क १५, हतिः २७०, आद्यया क ३० (भक्ता) ९, मूलं करणी क ३ । अनया त एव भक्ते जाते क ६ क ५, पूर्वया सह जाताः क ६ क ५ क ३ । शेषकरणीनामधो\footnote{करणीनां मध्ये ।} विन्यस्य जातं क १२ क १० क ६ 
क ६ क ५ क ३ 
भक्ते जाता करणी क २ । पूर्वया सह करणीचतुष्टयं क ६ क ५ क ३ क २ । पुनः शेषकरणीनामधो\footnote{करणीनां मध्ये ।} विन्यस्य जातं क ६ क ५ क ३ क २ । भक्ते लब्धं करणी क १ ।
क ६ क ५ क ३ क २ 
मूलकरणीनां यथाक्रमं न्यासः क ६ क ५ क ३ क २ क १ । एतद्‌वर्गकरणीयोगं क २८९ रूपकृतेः २८९ विशोधयेत् ।
पञ्चमवर्गस्य चतुर्भक्तस्य न्यासः - रू ६० रक ५७६ क २५६ क १४४ क ६४ क ३६ क १६ । आद्यासन्नद्वयहतिरित्यादिना प्राग्वन्मूलकरण्यः क ३२ क १८ क ८ क २ । एतद्वर्गकरणीयोगं क ३६०० रूपकृतेर्विशोधयेत् । 
अपि च -
प्रकृतिपुरन्दरनगरसगुणभुजतुल्याश्चतुर्गुणा विद्वन् ।
वर्गे यत्र करण्यः सविश्वरूपाः पदं ब्रूहि ॥ २१ ॥
न्यासः - रू १३ क ८४ क ५६ क २८ क २४ क १२ क ८ । चतुर्भक्ता रू १३ क २१ क १४ क ७ क ६ क ३ क २ । अत्राद्यासन्नहतिरादिहृता मूलं नास्त्यत एकविंशति-षण्मितकरण्यो क २१ क ६ र्वधे क १२६ चतुर्दशकरण्या क १४ भक्ते जातं ९, अस्य मूलं ३ । इयं करणी क ३ । अनया एकविंशति-षट्करण्यौ भक्ते जातं क ७ क २ । यथाक्रमं न्यासः क ७ क ३ क २ । शेषकरणीनामधो\footnote{णीनां मध्ये ।} विन्यस्य जातं क ७ क ३ क २ । भक्ते जाता करणी क १ । क ७ क ३ क २ 
मूलकरणीनां यथाक्रमं (न्यासः) क ७ क ३ क २ क १ ।
अथवा एकविंशति-त्रिकरण्योर्धाते क ६३ सप्तकरण्या क ७ भक्ते ९ मूलं करणी क ३, अनया भक्ते करण्यौ क ७ क १ । यथाक्रमं (न्यासः) क ७ क ३ क १ । शेषकरणीनामधो\footnote{णीनां मध्ये ।} विन्यस्य जातं क १४ क ६ क २ । भक्ते जाता करणी क २ । मूलकरणीनां 
क ७ क ३ क १ 
यथाक्रमं (न्यासः) क ७ क ३ क २ क १ ।
(अथवा) चतुर्दश-द्विकरण्योर्घाते (क) २८, सप्तकरणीभक्ते मूलं करणी क २, त एव तया भक्ते करण्यौ क ७ क १ । यथाक्रमं (न्यासः) क ७ क २ क १ । शेषकरणीनामधो\footnote{णीनां मध्ये ।} विन्यस्य जातं क २१ क ६ क ३ । भक्ते जाता करणी क ३ । यथाक्रमं 
क ७ क २ क १ ।
(न्यासः) क ७ क ३ क २ क १ । इति ।
आद्यैनीयं मूलक्रमो विस्तारितः, इदानीमस्माभिर्बालावबोधार्थं सुगमः कृतः । यदि सङ्कलितपदमितखण्डेभ्यो न्यूनाधिकानि (करणीखण्डानि) भवन्ति तदा विश्लेष्य संयोज्य मूलं ग्राह्यम् । 
सूत्रम् -
सङ्कलितपदोत्थितयाऽल्पयाऽब्धिहतयाऽपवर्तनो यासाम् ।\footnote{वर्त्तोर्यासां ।}
रूपकृतेस्ता शोध्या\footnote{विशोध्या ।} अपवर्तफलाः करण्यः\footnote{ण्याः ।} स्युः ॥ ५१ ॥
उदाहरणम् -
वसुरसनेत्रप्रमिता यत्र करण्यश्चतुर्गणा वर्गे\footnote{ण्यचतुर्गुणाः वर्गः ।} ।
युक्ता रूपैर्दशभिस्तत्र पदं ब्रूहि मे गणक ॥ २२ ॥
न्यासः - रू १० क ३२ क २४ क ८ ।
अत्र सङ्कलितपदस्य करणीखण्डद्वयस्त तुल्यानि रूपाण्यपास्य मूलं ग्राह्यं पुनरेकस्य एवं क्रियमाणे मूलं नास्त्यतोऽस्यकरणीगतमूलाभावः ।
अथानियमेन सर्वकरणीतुल्यानि रूपाण्यपास्य मूलं ६ ऊनयुतरूपाणामर्द्धे मूलकरण्यौ क २ क ८ अस्य वर्गोऽयम् रू १८ । अथवा द्वात्रिशदष्टमितकरण्योर्योगं कृत्वा न्यासः - रू १० क ७२ क २४ । करण्योस्तुल्यानि रूपाणि \footnote{रूपकृतेरानीय जाते रू २ क ६ ।}रूपकृतेरपनीय (मूलं २ ऊनयुतरूपाणामर्द्धे कृते) जाते क ४ क ६ अस्य वर्गः रू १० क ९६ अतोऽसत् ।
उदाहरणम् -
\footnote{षष्ठिर्द्विप ।}षष्टिर्द्वापञ्चाशद् द्वादश करणीत्रयं कृतौ यत्र । 
दशभी रूपैर्युक्तं\footnote{रूपैयुक्तं ।} तत्र सखे किं पदं ब्रूहि ॥ २३ ॥
(न्यासः) रू १० क ६० क ५२ क १२ । 
अत्र सङ्कलितपदमितस्य करणीद्वयस्यास्य क ५२ क १२ करण्यास्तुल्यानि रूपाण्यपास्य मूलं ६, अतो जाते करण्यौ क २ क ८, अल्पयाऽनया क २ चतुर्गुणया क ८ द्विपञ्चा (शद् )- द्वादशमितयोपरवर्तनं न स्यादतस्ते न शोधितव्ये ।
यत् उक्तं ``सङ्कलितपदोत्थितयाऽल्पया चतुर्गुणये" ति अत्राल्पयेत्युपलक्षणं तेन क्वचिन्महत्याऽपि ।
उदाहरणम् -
तिथिमनुनयनकरण्यश्चतुर्गुणा\footnote{चतुगु ।} रूपदशकसंयुक्ताः ।
किं मूलं ब्रूहि सखे करणीगणिते श्रमोऽस्ति यदि ॥ २४ ॥
न्यासः - रू १० क ६० क ५६ क ८ । 
अत्र करणीखण्डद्वयादस्मात् क ५६ क ८ उत्पन्ने करण्यौ\footnote{करण्येते ।} क २ क ८ । अल्पया चतुर्गुणया क ८ अनयाऽपवर्ते कृते लब्धकरण्यौ क ७ क १ । यथाक्रमं क ७ क २ क १ । शेषविधिना उत्पन्ने करणीखण्डे क ३ क ५ । यथाक्रमं क ५ क ३ क २ । एताः पूर्वकरणीखण्डैरेभिः क ७ क २ क १ तुल्या न स्युरित्यसत् ।
उदाहरणम् -
तिथिमनुरविविश्वककुभ्नगसङ्ख्याः कृतहताः करण्यश्चेत् ।
षोडशरूपसमेता यत्र कृतौ तत्र किं पदं कथय ॥ २५ ॥
न्यासः - रू १६ क ६० क ५६ क ४८ क ५२ क ४० क २८ ।
अत्र सङ्कलित (पदमितकरणी) खण्डत्रयस्यास्य क ५६ क ५२ क ४८ पश्चात् खण्डद्वयस्यास्य क ६० क २८ पश्चादस्यैकस्य क ४० तुल्यानि रूपाणि अपास्येत्यादिना मूलकारण्यः क १० क ३ क २ क १ । अथवा आदौ खण्डत्रयं पश्चात् खण्डद्वयं पश्चादेकमिति नियमः कः, आदावेवैकं\footnote{आदाययेकं ।} क ६० पश्चाद् द्वे एते क ५६ क ४८ पश्चात् त्रीणि एतानि क ५२ क ४० क २८ तुल्यानि\footnote{क ५२ क ४८ तुल्यानि ।} रूपाणि (अपास्य) इत्यादिना खण्डान्युत्पन्नानि क १० क ३ क २ क १ एवमतो मूलमेतदिति\footnote{मैत ।} । एवं करण्य एता एताश्चेत्\footnote{एवं करण्यते चेत् ।} अस्य वर्गः रू १६ क १२० क ८० क ४० क २४ क १२ क ८ अयमुदाहृतो वर्गः स्यात् अतोऽसत् । वर्गराशिमूले गृहीते तन्मूलस्य वर्गः पूर्ववर्गतुल्यो न चेत् तन्मूलमसदिति ।
(अत्र) सर्वत्र करणीनामासन्नमूलकरणेन मूलान्यानीय रूपेषु प्रक्षिप्य मूलं वाच्यम् ।
सूत्रम् -
वर्गे क्षयात्मिका चेत्तामपि करणीं धनात्मिका\footnote{क्षयात्मिका चित्तामपि धनात्मिकी ।} कृत्वा ।
मूलं ग्राह्यं (च) तयोः क्षयात्मिकैका भवत्येव ॥ ५२ ॥
पूर्वोक्तक्षयात्मकवर्गस्य मूलार्थं न्यासः - रू ८ क ६० । अत्रर्णकरणीं धनात्मिकां कृत्वा न्यासः - रू ८ क ६० । प्राग्वन्मूलं क ५ क ३ । एतयोरेका ऋणं क ५ क ३ वा क ५ क ३ ।
द्वितीयस्य न्यासः\footnote{निवासः ।} - रू १० क ६० क २४ । अत्रर्णकरण्यौ धनं प्रकल्प्य प्राग्वन्मूलकरण्यः क ५ क ३ क २ वा क ५ क ३ क २ । अथवा त्रिमितकरण्यां ऋणं कल्पितायां क ५ क ३ क २ अस्य वर्गः रू १० क ६० क ४० क २४ एष वर्ग उद्दिष्टसमो न स्यात् तन्मूलमतोऽसत् । उद्दिष्टवर्गराशेर्मूलराशिवर्गो यथा समः स्यात्तथा धर्नर्ण\footnote{धनर्ण ।} कल्प्यम् ।
इति करणीषड्विधम् ।
तदेवं षट्त्रिशत्परिकर्माणि समाप्तानि ।
(२) कुट्टकः 
अथ कुट्टके सूत्रम् -
भाज्यो हारः क्षेपः\footnote{हारः क्षेपकः ।} केनाप्यपवर्त्य कुट्टकस्यार्थम् ।
येेन विभाज्यच्छेदौ (छिन्नौ) क्षेपो\footnote{विभज्य छेदौ छेपो ।} न तेन खिलम् ॥ ५३ ॥ 
हरभाज्ययोर्विहृतयोरन्योन्यं यो भवेद् ययोः शेषः ।
स तयोरपवर्तनकृत् (तौ) तेनैवापवर्तितौ तु दृढौ ॥ ५४ ॥
दृढभाज्यहरौ विभजेत् परस्परं यावदेकमवशेषम् । 
विन्यस्याधोऽधस्तात् फलानि तदधस्तथा क्षेपम्\footnote{आक्षेपं ।} ॥ ५५ ॥
तदधः खमुपान्त्येनाहते\footnote{पात्येन ।} निजोर्ध्वेऽन्तिमेन\footnote{र्ध्वेतिमे ।} संयुक्ते ।
अन्त्यं जह्यादेवं यावद्राशिद्वयं भवति ॥ ५६ ॥
हरभाज्याभ्यां तष्टावधरोध्वौ ते क्रमेण गुणलब्धी\footnote{वब्धी ।} ।
यदि लब्धयः समाः स्युस्तदा\footnote{लब्धयः समास्युस्तदा ।} गुणाप्ती यथागते भवतः ॥ ५७ ॥
विषमाश्चेत् ते शोध्ये गुणलब्धी स्वस्वतक्षणाच्छेषे\footnote{स्वतक्षणाच्छेषे ।} ।
योगभवे गुणलब्धी निजनक्षणतो विशोधिते क्षयजे\footnote{क्षेपजे ।} ॥ ५८ ॥
इष्टध्नतक्षणयुते बहुधा भवतो गुणाप्ती ते । 
सर्वत्र कुट्टकविधौ कार्य समतक्षणं सुधिया ॥ ५९ ॥
उदाहरणम् -
राशिस्त्रिसप्ततियुतेन शतद्वयेन 
निघ्नो नवोनितशतेन युतश्च कोऽपि ।
भागं प्रयच्छति\footnote{श्यछति ।} विशुद्धमगाब्धिनेत्रै-
र्भक्तः\footnote{भक्तैः} सखे कथय तं गुणकं फलं मे ॥ २६ ॥
न्यासः - भआ २७३ क्षे ९१ ह २४७ । अत्र\footnote{अप्र} ``हरभाज्ययोर्विहृतयोरन्योन्यमि" ति भाज्यो २७३ हारेण २४७ भक्तः शेषं २६, अनेन हारो २४७ भक्तः शेषं १३, (अेनन पूर्वशेषं २६ भक्तं शुध्यति ततोऽपबर्तनराशिः १३ ।) अनेन भाज्यहारक्षेपानपवर्त्य जाता दृढा\footnote{जातौ दृढौ ।} भा २१ क्षे ७ ह १६ । दृढभाज्यभाजकयोः परस्परं भक्तयोः फलान्यधोऽधस्तदधः क्षेपं तदधः खं विन्यस्य जाता वल्ली\footnote{वल्ली ९७० ।} १९७० उपान्तिमेन ७८ स्वोर्ध्वे\footnote{स्वोर्ध्ये ।} ९ (हते ६३) अन्त्येन (०) युते जातं ६१३७ । पुनरूपान्तिमेन ६३ स्वोर्ध्वे १ हते ६३ अन्त्येन ७ युते ७० जातं राशिद्वयम् ७० ६३ । अधरोर्ध्वौ\footnote{अधरोध्वौ ।} तौ ६३ । ७० दृढहारभाज्याभ्यां १९ । २१ तष्टौ जातौ ६ । ७ । समा लब्धय\footnote{लब्ध्य ।} अत एव (एते) एव गुणाप्ती ६ । ७ । ``इष्टघ्नतक्षणयुत"\footnote{द्युत ।} इति एकेन इष्टेन गुणाप्ती २५ । २८ द्विकेन वा\footnote{या ४४ । ४८९ ।} ४४ । ४९ त्रिकेण वा ६३ । ७० एवं बहुधा । 
सूत्रम् -
हारक्षेपकयावा\footnote{हर ।} प्रक्षेपकभाज्ययोस्तदुभयोर्वा ।
अपवर्तितयोर्गुणको लब्धिश्च स्वापवर्तहते\footnote{हता ।} ॥ ६० ॥
उदाहरणम् -
येनाभिहताऽशीतिः समन्विता त्रिंशता च वियुता\footnote{विद्युता ।} वा ।
त्रिगुणत्रयोदशाप्ता शुध्यति तं कथय पृथगाप्तिम् ॥ २७ ॥
न्यासः - भा ८० क्षे ३० ह ३६ । प्राग्वज्जाते गुणाप्ती २४ । ५० । 
अथवा भाज्यक्षेपौ दशभिरपवर्तितौ भा ८ क्षे ३ हा ३६ प्राग्वज्जाता वल्ली ०४१३०। जातं राशिद्वयम् ३१५ । (गुणाप्ती १५ । ३ ) । लब्धयो विषमा अतः स्वतक्षणाभ्यामाभ्यां ३६ । ८ विशोदिते जाते धनक्षेपजे (गुणाप्ती) २४ । ५ । लब्धिः स्वापवर्तनेन दशभिगुणिता जाते त एव गुणााप्ती २४ । ५० ।
अथवा हारक्षेपौ\footnote{भाज्यक्षेपौ ।} त्रिभिरपवर्तितौ भा ८० क्षे १० ह १३ । (प्राग्वत् ) जाते (गुणाप्ती ) ८ । ५० । गुणकस्त्रिभिरपवर्तनेन गुणितो जाते गुणाप्ती त एव २४ । ५० ।
अथवा भाज्यक्षेपौ दशभिरपवर्त्य हारक्षेपौ त्रिभिरपवर्तितौ भा ८ क्षे १ ह १३ । (प्राग्वत् ) जातं राशिद्वयम् ३५ । (गुणाप्ती ५ । ३) लब्धयो विषमा अतः स्वतक्षणाभ्यां १३ । ८ शोधिते जाते ८ । ५ । हरक्षेप-भाज्यक्षेपापवर्तनाभ्यां\footnote{क्षेपावर्तने ।} ३ । १० आभ्यां क्रमेण गुणिते जाते त एव गुणाप्ती २४ । ५० । प्राग्वदेकेन इष्टेन जाते ६३ । १३० द्विकेन वा १०२ । २१० एवमनेकधा ।
द्वितीयोदाहरणे न्यासः - भा ८० क्षे ३० ह ३९ । जाते योगजे गुणाप्ती २४ । ५० । एते स्वतक्षणाभ्यां शुद्धे (जाते) वियोगजे (गुणाप्ती) १५ । ३० । एकेनेष्टेन जाते गुणाप्ती ५४ । ११० द्विकेन ९३ । १९० एवमिष्टवशादनेकधा ।
उदाहरणम् -
को राशिः पञ्चभिः क्षुण्णः सप्तत्रिंशत्समन्वितः\footnote{सप्तत्रिंशतिसंयुतः ।} ।
वर्जितो वा त्रिभिर्भक्तो निरग्रः स्याद् वदाशु\footnote{वदाश्रु ।} तम् ॥ २८ ॥
न्यासः -भा ५ क्षे ३७ ह ३ । जाता वल्ली ११३७० । राशिद्वयं ७४३७ । अधःस्थिते\footnote{अध स्थिते ।} राशौ त्रिभिस्तष्टे द्वादश लभ्यन्ते, ऊर्ध्वस्थिते राशौ पञ्चभिस्तष्टे चतुर्दश\footnote{चतुर्द्दश} लभ्यन्ते । असमा न ग्राह्याः ``कार्यं समतक्षणमि"ति द्वादशसु गृहीतेषु जाते गुणाप्ती १ । १४ चतुर्दशषु\footnote{चर्तुर्द्द ।} गृहीतेषु जाते गुणाप्ती ५ । ४ । ``समतक्षणमि"त्युपचारः यथेष्टघ्नतक्षणयुतेन बहुधा भवतो गुणाप्ती तथेष्टघ्नतक्षणवियुते राशिद्वये बहुधा गुणाप्ती भवतः । ऋणक्षेपे द्वादशफले गृहीते गुणाप्ती २ । ९ चतर्दशफले गृहीते जाते गुणाप्ती ८ । १ । इत्यादि । 
सूत्रम् - 
हरतष्टे धनक्षेरपे लब्धिस्तक्षणफलेन संयुक्ता ।
क्षयगे क्षेपे तक्षणफलोनिते\footnote{निता ।} जायते लब्धिः ॥ ६१ ॥ 
हरतष्टभाज्यराशौ तष्टफलघ्नगुणसंयुता लब्धिः ।
उदाहरणम् -
को राशिः खाभ्रदिङ्निघ्नो दिगश्विनयनैर्युतः\footnote{नैयुतः ।} ।
न्यासः - भा १००० क्षे २२१० ह १४३ । अतः प्राग्वज्जाते गुणाप्ती ६५ । ४७० । (अथवा ) भाज्ये हरेण तष्टे जातं भा १४२ क्षे २२१० ह १४३ । अतः । प्राग्वज्जाते गुणाप्ती ६५ । ८० । भाज्यतक्षणफलं ६ गुणः ६५ अनयोर्घात\footnote{अनयोगतिः ।} ३६० युता जाता लब्धिः\footnote{यतो जाता लब्धि ।} सैव ४७० । अथवा हरतष्टे क्षेपे न्यासः\footnote{यथा ।} भा १००० क्षे ६५ ह १४३ । जाते गुणाप्ती ६५ । ४५५ । क्षेप \footnote{क्षेत्र ।}तक्षणलब्ध्या १५ युता जाता लब्धिः सैव ४७० । अथवा भाज्यक्षेपयोर्हरतष्टयोर्न्यासः\footnote{पयार्हरतष्टयो न्यासः ।} भा १४२ क्षे ६५ ह १४३ । जाते गुणाप्ती ६५ । ६५ । भाज्यतक्षणफलं ६ गुणः ६५ अनयोर्हतिः ३९० क्षेपतक्षणफलं १५ अनयोर्योगः ४०५ अनेन लब्धिः\footnote{लब्धि ।} ६५ युता जाता सैव ४७० ।
द्वितीयस्य न्यासः - भा १००० क्षे २२१० ह १४३ । जाते प्राग्वद् गुणाप्ती ७८ । ५३० । हरतष्टक्षेपे भा १००० क्षे ६५ ह १४३ । जाते गुणाप्ती ७८ । ५४५ । क्षेपतक्षणफलोना जाता लब्धिः सैव ५३० ।
सूत्रम् -
क्षयभाज्ये गुणलब्धी धनवत्साध्ये तु भाज्यतः क्षेपे ।
अल्पे तयो\footnote{तयो ।} क्षयं स्यादेकमनल्पे तु ते सकृद्धनगे ॥ ६२ ॥ 
उदाहरणम् -
क्षयत्रिंशद्गुणो राशिस्त्रिभिर्युक्तोऽथवोनितः ।
सम्भक्तो\footnote{समभ ।} निरग्रः स्यात्त\footnote{स्यात्त ।} गुणं वद वेत्सि चेत् ॥ ३० ॥
न्यासः -भा ३० क्षे ३ ह ७ । भाज्यं धनं प्रकल्प्य धनभाज्ये धनक्षेपे गुणाप्ती २ । ९ । एते\footnote{२ । ९९ ते ।} स्वतक्षणाभ्यामाभ्यां ७ । ३० शोधिते जाते (धनभाज्ये ऋणक्षेपे गुणाप्ती ५ । २१ ) । ऋणभाज्ये धनक्षेपे (तयोरेकमृणमिति ) गुणाप्ती (२ । ९ वा ) ५ । २१ । (एवं ऋणभाज्ये ऋणक्षेपे गुणाप्ती २ । ९ वा ५ । २१ ।)\footnote{एतयोरेकमृणं ५ । २१ वा ५ । २१ ।}
अपि च -
क्षयत्रिंशद्गुणः\footnote{क्षयस्त्रिंशदृणः ।} सप्तनवत्या संयुतोनितः\footnote{संयुतोथवा ।} ।
सप्ताप्तः शुद्धिमायाति तं गुणं वद मे द्रुतम्\footnote{दृतं ।} ॥ ३१ ॥
न्यासः -भा ३० क्षे ९७ ह ७ । ``धनवत्साध्ये " इति\footnote{साध्येति ।} प्राग्वज्जाते गुणाप्ती ४ । ३१ । एतयोरेकमृणमिति लब्धिमृणं प्रकल्प्य ऋणभाज्ये ऋणक्षेपे गुणाप्ती ४ । ३१ । एते स्वतक्षणाभ्यामाभ्यां ७ । ३० शोधिते जाते ऋणभाज्ये धनक्षेपे (गुणाप्ती) ३ । १ । 
क्षयगे हरेऽप्येवमूह्यम\footnote{मुह्यं ।} ।
सूत्रम् -
हरहृतशुद्धे क्षेपे शून्ये जातेऽथवा गुणः खं स्यात्\footnote{स्मात् ।} ।
शून्ये तु भाज्यराशौ हारहृतः\footnote{हरहृत ।} क्षेपको लब्धिः ॥ ६३ ॥
उदाहरणम् -
को राशिः सप्तहतो नवभिर्युक्तोऽथवोनितः शुद्धम् ।
त्रिभिरूद्धतः\footnote{त्रिभिस्त्वद्धृतः ।} प्रयच्छति भागं तं गुणकमाचक्ष्व ॥ ३२ ॥
न्यासः - भा ७ क्षे ९ ह ३ । जाते गुणाप्ती ० । ३ एकेनेष्टेन ३ । १० द्बिकेन ६ । १७ ।
नवशुद्धौ गुणाप्ती ३ । ४ एकेनेष्टेन ६ । ११ द्बिकेन ९ । १८ ।
अपि च -
को राशिर्नवगुणितः शून्ययुतः\footnote{शुन्य ।} पञ्चभिर्हृतः\footnote{पंचभिहृ ।} शुद्धिम् ।
गच्छति\footnote{यच्छति ।} तं द्राग्राशिं गणकवर\footnote{वरं ।} ब्रूहि त्वं\footnote{तं ।} यदि वेत्सि ॥ ३३ ॥
न्यासः - भा ९ क्षे ० ह ५ । जाते गुणाप्ती ० । ० एकेनेष्टेन ५ । ९ द्विकेन १० । १८ ।
अपि च -
को राशिः शून्यहतो द्वादशयुक्तो विवर्जितो वाऽपि ।
चतुरुद्धृतो विशुध्यति तं गुणकं\footnote{गुण  ।} गणक मे कथय ॥ ३४ ॥
न्यासः - भा ० क्षे १२ ह ४ । जाते द्वादशक्षेपे गुणाप्ती ० । ३ । एकेनेष्टेन ४ । ३ । द्विकेन ८ । ३ । द्वाशशुद्धौ जाते ४ । ३ । एकेनेष्टेन ८ । ३ द्विकेन १२ । ३ । भाज्ये शून्ये\footnote{शून्य ।} लब्धिः एवं सर्वत्रापि (अविकृतैव) । 
सूत्रम् -
क्षेपं शुद्धिं\footnote{क्षेपशुद्धि ।} रूपं परिकल्प्य तयोः पृथग्गुणाप्ती ये ।
इष्टक्षेपविशुद्ध्या हते स्वहरतक्षिते भवतः ॥ ६४ ॥
उदाहरणम् -
प्रथमोदाहरणे दृढभाज्यहारौ भा २१ क्षे ७ ह १९ । रूपं क्षेपं प्रकल्प्य न्यासः -भा २१ क्षे १ ह १९ । रूपक्षेपे गुणाप्ती ९ । १० । इष्टसप्तकेन क्षेपेण गुणिते स्वहरतष्टे\footnote{नष्टे ।} जाते सप्तक्षेपे गुणाप्ती ६ । ७ । रूपशुद्धौ गुणाप्ती १० । ११ । इष्टमप्तक्षेपहते स्वहरतष्टे जाते सप्तशुद्धौ (गुणाप्ती) १३ । १४ ।
सूत्रम् -
यस्मिन् यस्मिन् कर्मणि यद् यत् परिभाषितं\footnote{शुध्यतरिभाषितं ।} समुदितं च । 
तस्मिन् तस्मिन् कर्मणि त (त्त)त्परिभाषितं भवति ॥ ६५ ॥
त्रैराशिके प्रमाणं हारः परिभाषिनोन्मितिभज्यिः ।
अवशिष्टमृणक्षेपो या लब्धिस्तत्प्रमाणं\footnote{अवशिष्टमृणक्षेपलब्धिस्ततप्रमाणं ।} स्यात् ॥ ६६ ॥
गुणकस्तु पूर्वशेषं तत्पूर्वं पूर्वमेवमपि ।
अनुपातेच्छायामथ ज्ञातायां\footnote{तेछायामथज्ञानायां च ।} तत्फलं वाच्यम् ॥ ६७ ॥
यो गुणकः सैवेच्छा\footnote{वेछा ।} या लब्धिस्तत्फलं\footnote{ततफलं ।} भवति ।
उदाहरणम् -
पंगुर्योजनषष्टिमेकसहितामब्दैस्त्रिपञ्चाशता\footnote{षष्ठिमेकसहिता-मब्दैमब्दैस्त्रिपंचशिता । }
रिङ्गन् क्रामति\footnote{क्रमति ।} योजनानि च कियत्सङ्ख्यानि येनासरत् ।
कालेनाशु वदार्य\footnote{नाश्रु विदा ।} तत्र घटिकाशेषे भवेद् र्विशति-
स्तत्सम्वत्सरमासवासरघटीमानानि\footnote{यानानि ।} चेच्छां पृथक् ॥ ३५ ॥
न्यासः - ६१ । ५३ । ० घटिकाशेषं २० । अत्र ``घटिकानां षष्ट्या\footnote{षष्ठ्या ।} दिनिमि" ति षष्टिर्भाज्यः\footnote{षष्ठिर्भाज्य ।} प्रमाणं हरः घटिकाशेषं शुद्धिरिति प्रकल्प्य कुट्टकार्थं न्यासः - भा ६० क्षे २० ह ६१ । (जाते गुणाप्ती ४१ । ४० । लब्धिर्घटिका ४० गुणो दिनशेषं ४१ । ``दिनत्रिंशता मास" इति त्रिंशद् भाज्यः प्रमाणं हरो दिनशेषं शुद्धिरिति प्रकल्प्य कुट्टकार्थं न्यासः - भा ३० क्षे ४१ ह ६१।) जाते गुणाप्ती ४० । १९ । लब्धिर्दिनानि १९, गुणो मासशेषं ४० । ``द्वादशभि-\footnote{दस ।} र्मासैर्वर्ष" मिति द्वादश भाज्यः, प्रमाणं हरो, मासशेषं शुद्धिरिति प्रकल्प्य कुट्टकार्थ\footnote{र्थ ।} न्यासः भा १२ क्षे\footnote{क्षे १९ ।} ४० ह ६१ । जाते गुणाप्ती ४४ । ८ । लब्धिर्मासाः ८ गुणो\footnote{गुणे ।} वर्षशेषं ४४ । त्रिपञ्च शद् भाज्यः, प्रमाणं हरो, वर्षशेषं शुद्धिरिति कुट्टकार्थं न्यासः - भा ५३ क्षे ४४ ह ६१ । जाते गुणाप्ती २५ । २१ । लब्धिर्वर्षाणि २१, गुण इच्छा २५ इति जातं त्रैराशिकम् ६१ । ५३ । २५ । लब्धं वर्षाणि २१ मासाः ८ दिनानि १९ घटिकाः ४०  घटिकाभागाः २० ६१ । एवं (सर्वत्र सुधिया) ऊह्यम् । 
भिन्नकुट्टके\footnote{कृट्ट ।} सूत्रम् -
ईप्सितलब्ध्या हारे\footnote{इप्सितलब्ध्याहारो ।} गुणिते क्षेपोनिते च भाज्याप्ते ।
गुणकारः स्याद्दृढे दृढेऽपि\footnote{स्याददृढदृढे ।} वा कुट्टके\footnote{कृट्ट ।} भवति ॥ ६८ ॥
उदाहरणम् -
कश्चित् त्रिंशद्गुणितो दशसंयुतो\footnote{दशसंयुक्तो ।} द्वादशोद्धृतो यत्र ।
यच्छति शुद्धं भागं कुट्टकगणितज्ञ\footnote{तज्ञं ।} तं कथय ॥ ३६ ॥
न्यासः - भा ३० क्षे १० ह १२ । एकेनेष्टेन जाते गुणाप्ती\footnote{गुणाप्ती ५ । २ ।} १-१५ । १ द्विकेन ७-१५। २ त्रिकेण १३-१५ । ३ । एवमिष्टवशादनेकधा\footnote{९ऽव ।} ।
(अपि च-)
कुट्टकप्रभवाभावेऽप्यभिन्नलब्ध्या\footnote{कुट्टकाप्रभावेभावेप्यभिन्नकल्पितलब्ध्या ।} सदा गुणो न स्यात् । 
कुट्टकप्राप्तिभावे\footnote{कृट्टक ।} ऽभीष्टवशात्कदाचिदभिन्न ॥ ६९ ॥
यत् किञ्चित् कुट्टककौशल्यं तत्\footnote{कुट्टककौशल्यात् ।} पुरतोऽनेकवर्णसमीकरणे वक्ष्ये ।
इति कुट्टकः ।
Note - It is to be noted that sutras 69 and 70 above not included in the treatment of this subject in the Ganita Kaumudi. On the other hand, between Sutras 64 and 65 above, there occurs in the Ganita-Kaumudi a set of four Sutras followed by six examples dealing with the residual and conjunct pulverizers sarga-kuttaka and samslista-kuttaka)
वर्गप्रकृतौ सूत्रमार्यासप्तकम् -
ह्रस्वमभीष्टं\footnote{भीष्ट ।} मूलं तद्वर्गः प्रकृतिसङ्गुणो युक्तः । 
हीनो वा येन कृतिः स्यात्तस्मा(त्त)त्पदं ज्योष्ठम् ॥ ७० ॥
ह्रस्वज्येष्ठक्षेपान्\footnote{क्षेपाः ।} क्रमशस्तेषामधो न्यसेत्तांस्तु\footnote{न्यसेत्ते तु ।} ।
अन्यान्येषां\footnote{अन्यांश्चैषां ।} न्यासस्तस्य भवेद् भावना नाम ॥ ७१ ॥
वज्राभ्यासौ\footnote{भ्यासो ।} ह्रस्वज्येष्ठ(क)योः संयुतिर्भवेद्ध्रस्वम् ।
लघुघातः प्रकृतिहतो ज्येष्ठवधेनान्वितो ज्येष्ठम् ॥ ७२ ॥
क्षिप्त्योर्घातः क्षेपः स्याद् वज्राभ्यासयो\footnote{स्याद्वित्राभ्या ।} विशेषो वा ।
ह्रस्वं लघ्वोर्घातः प्रतृतिघ्नो ज्येष्ठयोश्च वधः ॥ ७३ ॥
तद्विवरं ज्येष्ठपदं क्षेपः\footnote{क्षेप ।} क्षिप्त्योः प्रजायते घातः ।
ईप्सितवर्गविहृतः\footnote{इप्सितवर्गविहृतः ।} क्षेपः क्षेपः\footnote{क्षेपे ।} पदे तदिष्टाप्ते ॥ ७४ ॥
गुणिते वा तन्मूले गुणिते मूले तदा भवतः ।
इष्टकृतिगुणकशेषोद्धृतं तदिष्टं द्विसंगुणं भवति ॥ ७५ ॥
ह्रस्वं मूुलं च ततो रूपक्षेपेण साधयेज्ज्येष्ठम्\footnote{येज्ये ।} ।
तुल्यातुल्यपदानां भावनयाऽनन्तमूलानि ॥ ७६ ॥
उदाहरणम् -
अष्टाहता\footnote{हत ।} यस्य कृतिः सरूपा 
स्यान्मूलदा ब्रूहि सखे तमाशु ।
एकादशघ्ना यदि वा कृतिः का 
वर्गत्वमेत्येकयुता विचिन्त्य ॥ ३७ ॥
न्यासः - प्रकृतिः ८ क्षेपकः १ । अत्राभीष्टं ह्रस्वं मूलं कल्पितं १ । अस्य वर्गः १ प्रकृतिगुणः ८ रूपयुतः ९ मूलं ३ एतज्ज्येष्ठमूलं ३ । क्रमेण न्यासः - क १ ज्ये ३ क्षे १ । एषामधस्तान्त्यसेदिति भावनार्थं न्यासः -
प्र ८ क १ ज्ये ३ क्षे १ 
क १ ज्ये ३ क्षे १
``वज्राभ्यासौ\footnote{बज्राभ्यासो ।} ह्रस्वज्येष्ठयोः\footnote{ह्रस्दः ।} संयुतिर्भवेद्ध्रस्वमि"ति\footnote{द्वस्व ।} प्रथमकनिष्ट-द्वितीयज्येष्ठयोरभ्यासः ३, प्रथमज्येष्ठ-द्वितीयकनिष्ठयोरभ्यासः ३, अनयोर्युतिः ६, ह्रस्वं भवेत् । तघुघातः १ प्रकृतिहतः\footnote{हत ।} ८ ज्येष्ठघातेन ९ अन्वितः १७ ज्येष्ठपदं भवेत् । क्षिप्त्योर्घातः क्षेपः १ । क्रमेण न्यासः - (क ६ ज्ये १७ क्षे १ । ``तुल्यातुल्यपदानां भावनयाऽनन्तमूलानी" त्यसमभावनार्थं न्यासः) -
प्र ८ क १  ज्ये ३ क्षे १ 
क ६ ज्ये १७ क्षे १ 
समासभावनया जाते मूले क ३५ ज्ये ९९ क्षे १ । पुनर्भावनार्थं न्यासः -
प्र ८ क १ ज्ये ३ क्षे १
क ३५ ज्ये ९९ क्षे १ 
समासभावनया\footnote{अत्र तुल्यभा ।} जाते मूले क २०४ ज्ये ५७७ क्षे १ । एवमनन्तमूलानि ।
अथवा कनिष्ठमूलं रूपद्वयं कल्पितं २ । अस्य वर्गः ४ प्रकृति ८ हत ३२ श्चतुःक्षेप\footnote{चतु क्षेप ।} ४ युतो ३६ मूलं ज्येष्ठं ६ । क्रमेण न्यासः - क २ ज्ये ६ क्षे ४ । ``ईप्सितवर्गविहृतः क्षेपः" इति रूपक्षेपार्थं कल्पितमिष्टं रूपद्वयं\footnote{द्वय ।} २, अस्य वर्ग ४ हृतः क्षेपो ४ लब्धः क्षे १ इष्टद्वयेन हृते मूले जाते रूपक्षेपमूले क १ ज्ये ३ क्षे १ । एभ्यो भावनाभिस्तान्येव मूलानि भवन्ति ।
द्वितीयोदाहरणे न्यासः - प्रकृतिः ११ क्षेपकः १ । अत्र रूपमिष्टं कनिष्ठं १ । तद्वर्गः १ प्रकृतिगुणो द्विरूपोनो\footnote{द्विरुपनो ।} ९ मूलं ज्येष्ठं ३ । यथाक्रमं समासभावनार्थं न्यासः-
प्र ११ क १ ज्ये ३ क्षे २
क १ ज्ये ३ क्षे २ 
समासभावनया जाते मूले क ६ ज्ये २० क्षे ४ । ``ईप्सितवर्गविहृत\footnote{इप्सितवर्गविहृत ।}" इति रूपक्षेपे मूले क ३ ज्ये १० क्षे १ । अतः समासभावनया जाते मूले क ६० ज्ये १९९ क्षे १ ।
अथवा रूपपञ्चकक्षेपमूले क १ ज्ये ४ क्षे ५ । समासभावनया जाते पञ्चविंशतिक्षेपमूले क ८ ज्ये २७ क्षे २५ । अतो रूप (क्षेप) मूले क ८५ ज्ये २७-५ । अनयोः\footnote{अनयो ।} पूर्वमूलाभ्यामाभ्यां ३ । १० सह भावनार्थ न्यासः -
प्र ११ क ३ ज्ये १० क्षे १ 
क ८५ ज्ये २७-५ क्षे १ 
समासभावनया जाते मूले\footnote{मूंले १६१ । ५२४-५ । } १६१-५ । ५३४-५ । एवमनन्तमूलानि । अथवा ``वज्राभ्यासयोविंशेष" इति भावनार्थं न्यासः -
प्र ११ क ३ ज्ये १० क्षे १ 
क ८५ ज्ये २७-५ क्षे १ 
अन्तरभावनया जाते मूले १५ । ६५ । अनयोः\footnote{अनयो ।} पूर्वमूलाभ्यामाभ्यां ३ । १० विशेषभावनया जाते मूले ८५ । २७-५ । एवमनन्तमूलानि ।
``इष्टकृतिगुणकशेषोद्धृतमि"ति रूपक्षेपपदाभ्यां पुनः पुनः समासविशेषभावनाभिर्बहूनि\footnote{भिवहूनि ।} मूलानि भवन्ति । तद्यथा-प्रथमोदाहरणे रूपत्रयमिष्टं\footnote{रुप ।} प्रकल्प्य यथोक्तकरणेन जातं कनिष्ठं (६) । अस्य वर्गात्प्रकृतिगुणाद्रूपयुतान्मूलं ज्येष्ठं\footnote{ज्येष्ठं १९ ।} १७ । रूपपञ्चकेष्टेन जातं कनिष्ठं १०१७ । अतो ज्येष्ठं ३३ १७ । अनयोः\footnote{अनयो ।} पूर्वमूलाभ्यामाभ्यां ६ । १७ समासभावनया जाते मूले ३६८ १७ । १०४१ १७ । अथवा विशेषभावनया जाते मूले २८१७ । ८११७ । एवं द्वितीयोदाहरणे रूपत्रयेणेष्टेन\footnote{रुप} जाते मूले ३ । १० । पञ्चकेनेष्टेन जाते (मूले) ५७ । १८ ७ । अनयोः पूर्वमूलाभ्यामाभ्यां\footnote{अनयो पूर्वमूलाभ्यामाभ्या ।} ३ । १० समासभावनया जाते मूले १०४ ७ । ३४५ ७ । अन्तरभावनया जाते मूले ४७ । १५७ । एवमनंतमूलानि । 
इति वर्गप्रकृतिः ।
चक्रवालम् 
एकद्विचतुःक्षेपसाधनाय\footnote{द्वित्रिचतुःक्षेपसाधनया ।} चक्रवाले करणसूत्रमार्याचतुष्टयम् \footnote{मार्यद्वयम् ।} -
ह्रस्वज्येष्ठक्षेपान् भाज्यप्रक्षेपभाजकान् कृत्वा ।
कल्प्यो गुणो यथा तद्वर्गात् संशोधयेत् प्रकृतिम् ॥ ७७ ॥
प्रकृतेर्गुषत्रर्गे\footnote{प्रकृति गुणवर्गो ।} वा विशोधिते जायते तु यच्छेषम् ।
तत्\footnote{न} क्षेपहृतं क्षेपो गुणवर्गविशोधिते व्यस्तम् ॥ ७८ ॥
लब्धिः कनिष्ठमूलं तन्निजगुणकाहतं वियुक्तं च । 
पूर्वाल्पपदपरक्षिप्त्योर्घातेन जायते ज्येष्ठम् ॥ ७९ ॥
प्रक्षेपशोधनेष्वप्येकद्विचतुर्ष्वभिन्नमूले\footnote{द्वित्रिचतुष्काभिन्न} स्तः । 
द्विचतुक्षेपपदाभ्यां\footnote{द्विचतुः प्रक्षेपाभ्यां ।} रूपक्षेपाय भावना कार्या\footnote{विशुद्धा या ।} ॥ ८० ॥
उदाहरणम् -
कस्त्र्युत्तरेण गुणितोऽत्र शतेन वर्गः 
सैकः कृतित्वमुपयाति वदाऽऽशु तं मे ।
को वा\footnote{व ।} त्रिवर्जितशतेन हसस्तु वर्गो 
रूपान्वितः कृतिगतो\footnote{प्रकृति ।} भवति प्रचक्षव ॥ ३८ ॥
न्यासः - प्रकृतिः १०३ क्षेपकः १ । प्राग्वद्रूपत्रयशुद्धौ मूले क १ ज्ये १० क्षे ३ । अत्र ह्रस्वपदं भाज्यं, ज्येष्ठपदं क्षेपं, क्षेपं हरं\footnote{ह्रस्वपदभाज्यं ज्येष्ठपदं क्षेपं हरं क्षेपं ।} प्रकल्प्य कुट्टकार्थं न्यासः -भा १ क्षे १० ह ३ । कुट्टककरणेन लब्धो गुणः सक्षेपः क्षे ३ (गु) २ । क्षेपे ऋणेन\footnote{क्रमेण ।} त्रयेण गुणिते जातोऽपरो गुणः ११ । अस्य वर्गात् १२१ प्रकृति १०३ मपास्य\footnote{मयास्य ।} शेषं १८ क्षेप ३ हृतं जातः क्षेपः ६ । लब्धिः कनिष्ठमूलं ७ । एतन्निजगुणकेन ११ हतं ७७, पूर्वह्रस्वपदं १ परक्षेपः ६ घातेन ६ वियुक्तं ज्येष्ठं ७१ । ऋणधनमूलयोरुत्तरे कर्मणि क्रियमाणे न विशेषः । तस्मादृणमुलयोर्धनत्वं प्रकल्प्य षट्शोधने प्रकृतिः १०३ क ७ ज्ये ७१ क्षे ६ ।\footnote{क ७ ज्ये १ क्षे ६ ।} (पुनः) कुट्टकार्थ न्यासः -भा ७ क्षे ७१ ह ६ । जातो गुणः सक्षेपः क्षे ६ गु १ । क्षेपे (ऋणेनैकेन गुणिते जातोऽपरो गुणः ७ । अस्य वर्गं प्रकृतेरपास्य शेषं ५४ ``गुणवर्गविशोधिते व्यस्तमि" ति) जातमृणं\footnote{क्षेपे ऋणमिति जातंमृणं ।} ५४ क्षेपेण ६ हृतं क्षेपः ९ । लब्धिः\footnote{लब्धि ।} कनिष्ठं मूलं २० । एतन्निजगुणकाहतं १४०, पूर्वह्रस्वपदपरप्रक्षेपघातः ६३ अनेन वियुक्त जातं ज्येष्ठं २०३ । पुनः प्रकृतिः\footnote{प्रकृति ।} १०३ क २० ज्ये २०३ क्षे ६ । कुट्टकार्थं न्यासः - भा २० क्षे २० ३ ह ९ । जातो गुणः सक्षेपः क्षे ९ गु २ । एकेनेष्टेन जातोऽपरो गुणः ११ । अस्य वर्गात्प्रकृतिमपास्य शेषं १८, क्षेपेण ९ हृतः क्षेपः (२) । लब्धिः कनिष्ठपदं ४७ । एतन्निजगुणकहतं ५१७, पूर्वाल्पपदं २० परक्षेपः २ अनयोर्घातेनानेन ४० वियुक्तं जातं ज्येष्ठं ४७७ । पुनः प्रकृतिः १०३ क ४७ ज्येष्ठं ४७७ क्षे २ ।\footnote{क ४७७ क्षे २ ।} प्रक्षेपशोधनेष्वप्येकद्विचतुर्ष्वभिन्नमूलेषु भावनयेति समासभावनार्थं न्यासः -
प्र १०३ क ४७ ज्ये ४७७ क्षे २ 
क ४७ ज्ये ४७७ क्षे २ 
(समासभावनया) जाते चतुःक्षेपमूले क ४४८३८ ज्ये ४५५०५६ क्षे ४ । अतो रूपक्षेपमूले क २२४१९ ज्ये\footnote{क २२४१४ ज्ये २२०५२८ ।} २२७५२८ क्षे १ ।
द्वितीयोदाहरणे प्रकृतिः ९७ क १ ज्ये १० क्षे ३ । प्राग्वत्कुट्टकन्यासः -भा १ क्षे १० ह ३ । जातो गुणः सक्षेपः क्षे ३ गु २ । धनरूपत्रयेण (इष्टेन) जातो अपरो गुणः ११ । अस्य वर्गात्प्रकृतिमपास्य शेषं २४, क्षेपहृतं क्षेपः ८ । लब्धिः कनिष्ठमूलं ७ । अतो ज्येष्ठं ६९ । (एवं) प्र\footnote{प्रा ।} ९७ क ७ ज्ये ६९ क्षे ८ । पुनः कुट्टकः भा ७ क्षे ६९ ह ८ ।अतो गुणः सक्षेपः क्षे ८ गु ५ । धनरूपेणैकेन जातो अपरो गुणः १३ । वर्गादस्य १६९ प्रकृतिमपास्य शेषं ७२ । क्षेपहृतं क्षेपः ९ । लब्धिः कनिष्ठमूलं २० । अतो ज्येष्ठं १६९ । (पुनः) प्रकृतिः ९७ क २० ज्ये १९७ क्षे ९ । अस्य वर्गात्प्रकृतिमपास्य शेषं ९९ क्षेप ९ हृतं क्षेपः ११ । लब्धिः कनिष्ठमूलं ५३ । अतो ज्येष्ठं ५२२ । पुनः प्रकृतिः ९७ क ५३ ज्ये ५२२ क्षे ११ । कुट्टकेन जातो गुणः सक्षेपः क्षे ११ गु ८ । शून्येन\footnote{शुन्येन ।} जातो अपरो गुणः ८ । अस्य वर्गं प्रकृतेरपास्य शेषं ३३ । ``गुणवर्गविशोधिते व्यस्तमि"ति जातमृणं ३३ क्षेपहृतंं क्षेपः ३ लब्धिः कनिष्ठं ८६ । अतो ज्येष्ठं ८४७ । पुनः प्रकृतिः ९७ क ८६ ज्ये ८४७ क्षे ३ । कुट्टकेन जातो गुणः सक्षेपः क्षे ३ गु १ । ऋणरूपत्रयेण\footnote{रूपत्रयेण ।} जातो अपरो गुणः १० । अस्य वर्गात्प्रकृतिमपास्य शेषं ३ क्षेपहृतं क्षेपः १ । लब्धिः कनिष्ठं ५६९ । अतो ज्येष्ठं ५६०४ । प्राग्वदृणत्वे धनत्वमिति जाते धनात्मके रूपशुद्धिमूले क ५६९ ज्ये ५६०४ क्षे १ । समासभावनया जाते रूपक्षेपमूले क  ६३७७३५२ ज्ये ६२८०९६३३ । 
सूत्रमार्या -
रूपविशुद्धौ प्रकृतिः कृतियोगः स्यान्न चेत् खिलं\footnote{न चेतिवलं ।} तु तदा ।
अखिलप्रकृतौ प्राग्वत् साध्ये मूलेऽल्पकानल्पे ॥ ८१ ॥
उदाहरणम् -
कस्त्रयोदशनिघ्नश्च वर्गो व्येकः पदप्रदः ।
को वर्ग\footnote{वर्ण ।} एकषष्टिघ्नो\footnote{षष्ठि ।} निरेको मूलदो वद ॥ ३९ ॥
प्रथमोदाहरणे द्विकत्रिकयोर्वर्गयोगः प्रकृतिः १३ । प्राग्वच्चतुःशुद्धौ मूले\footnote{मुले ।} क १ ज्ये ३ क्षे ४ । अतो रूपशुद्धौ मूले १२ । ३२ । अथवा नवशुद्धौ मूले क १ ज्ये २ क्षे ९ । अतो रूपशुद्धौ १३ । २३ । चक्रवालकरणेनाभिन्ने ५ । १८ ।
द्वितीयोदाहरणे (षट्पञ्चकयोर्वर्गयोगः प्रकृतिः ६१ । प्राग्वत् ) षट्त्रिंशत्शुद्धौ मूले क १ ज्ये ५ क्षे ३६ । अतो रूपशुद्धौ मूले\footnote{मुले ।} १६ । ५६ । अथवा पञ्चविंशतिशुद्धौ मूले क १ ज्ये ६ क्षे २५ । अतो रूपशुद्धौ मूले १५ । ६५ । चक्रवालेनाभिन्ने वा ३८०५ । २९७१८ । एवमनन्त-मूलानि ।
अपि च -
वर्गः पञ्चगुणः कश्चित् चतुर्भिः संयुतः कृतिः ।
षट्त्रिंशताऽथवा युक्तः शतयुक्तोऽथवा भवेत् ॥ ४० ॥
प्रकृतिः ५ क १ ज्ये ३ क्षे ४ । क्षेपो येन गुणितः तन्मूलगुणे कनिष्ठज्येष्ठे तद्गुणक्षेपे मूले भवतः । अतस्त्रिगुणिते षट्त्रिशत्क्षेपमूले ३ । ९ । पञ्चगुणिते शतक्षेपमूले ५ । १५ । एवं बुद्धिमता विशोधने मूले ज्ञेये ।
सूत्रम् -
प्रकृतिरभीप्सितवर्गोद्धृता यथा शुद्धिमेति यल्लब्धम् ।
कल्प्यो गुणः कनिष्ठं छेदनमूलोद्धृतं\footnote{छेदेन ।} भवति ॥ ८२ ॥
उदाहरणम् -
द्वासप्ततिप्रगुणिता कृतिरेकयुक्ता 
मूलप्रदा\footnote{मुलप्रदा ।} भवति मे वद मित्र शीघ्रम् ।
पञ्चांशकेन गुणितोऽप्यथवा सरूपो 
वर्गः कृतित्वमुपयाति सखे विचिन्त्य ॥ ४१ ॥
प्रथमोदाहरणे प्रकृतिः ७२ ईप्सितवर्गेण\footnote{इप्सितवर्गेण ।} ९ हृता विशुद्धलब्धं ८ । इयं प्रकृतिः ८ क १ ज्ये ३ क्षेे १ । अत्र कनिष्ठं छेदमूलेनानेन ३ हृतं जातं कनिष्ठं\footnote{कनिष्ठं ३ ।} १३ । एवं जाते ह्वस्वज्येष्ठे\footnote{ह्रस्वंज्ये ।} १३ । ३ ।
द्वितीयोदाहरणे प्रकृतिः १५। इयं पञ्चमांशवर्गेण १ २५ हृता विशुद्धं लब्धं ५ । प्राग्वद् रूपक्षेपमूले क ४ ज्ये ९ क्षे १ । कनिष्ठं छेदमूलेनानेन १५ हृतं जातं कनिष्ठं २० । एवं जाते ह्नस्वज्येष्ठ\footnote{ह्रस्वः ज्ये ।} २० । ९ । ``तुल्यातुल्यपदानां भावनयाऽनन्तमूलानि\footnote{नयान्नत ।}" ।
वर्गगतायां प्रकृतौ सूत्रम् -
क्षिप्तिरभीष्टविभक्ता द्विधा तदिष्टोनसंयुता\footnote{दिष्ठा न ।} दलिता ।
आद्या\footnote{आद्य ।} प्रकृतिपदाप्ता क्रमशोऽल्पानल्पमूले ते ॥ ८३ ॥
उदाहरणम् -
वर्गो नवहतः\footnote{वहृतः ।} कश्चिद् दशाढ्यो\footnote{कश्चिदुशाढ्यो ।} वा दशोनितः ।
मूलदो जायते तं मे गणितज्ञ वद द्रुतम्\footnote{वदाशु त्वं ।} ॥ ४२ ॥
प्रकृति ९ क्षे १० । अत्र क्षिप्तिः १० द्विधैकेनेष्टेन हृता १० । १० इष्टेनोनयुता दलिता ९२ । ११-२ । अनयोराद्या\footnote{अत्रयो ।} प्रकृतिपदेनानेन ३ हृता जाते मूले ३२ । ११-२ । द्विकेनेष्टेन मूले १२ । ७२ ।
द्वितीयोदाहरणे प्रकृतिः ९ क्षे १० । प्राग्वदेकेनेष्टेन जाते मूले ११-६ । ६२ । द्विकेन जाते मूले ७६ । ३२ । एते धनमूले वा\footnote{एव ।} भवतः । ऋणत्वे धनत्वे न विशेषः । एवमनन्तमूलानि\footnote{मनंतिम् ।} । रूपक्षेपपदाभ्यां समासान्तरभावनाभिर्बहूनि मूलान्युत्पद्यन्ते\footnote{द्यते ।} । 
प्रकृतिसमक्षेपविशुद्धावुद्धाहरणम् -
का कृतिर्दशभिः क्षुण्णा दशाढ्या वा दशोनिता ।
मूलदा जायते विद्वन् तां द्रुत\footnote{तादृशं ।} वद वेत्सि चेत् ॥ ४३ ॥
प्रकृतिः १० क्षेपः १० । अत्र दशशुद्धौ मुले\footnote{मुले ।} १ । ० । ``ह्रस्वमभीष्टं मूलमि" त्यादिना रूपशुद्धौ मूले १ । ३ । क्षे १ । पूर्ववत् दशशुद्धौ मूले १ । ० क्षे १० । भावनया जाते दशक्षेपमूले ३ । १० ।
द्वितीयोदाहरणे प्रकृतिः १० क्षे १० । प्राग्वद्दशशुद्धौ मूले १ । ० ।  ``इष्टकृतिगुणकशेषोद्धृतमि"ति\footnote{ईष्ट ।} (त्रिकेनेष्टेन) रूपक्षेपमूले ६ । १९ । पूर्वमूलाभ्यामाभ्यां १ । ० समासभावनया\footnote{सहभावनया ।} जाते दशशुद्धौ मूले\footnote{मुले ।} १९ । ६० । अन्तरभावनया त एव मूले १९ । ६० । 
अपि च -
क्षयगैकादशघ्नः को वर्गः षष्टिसमन्वितः । 
मूलदो\footnote{मूलदा ।} जायते तं मे\footnote{तमे ।} वद कोविद\footnote{कोविद्वन् ।} सत्वरम् ॥ ४४ ॥
* In place of this example, the Ganita-Kaumudi gives another similar example. See Ganita-Kaumudi, Part II, Ex, 8, pp. 242 243.
न्यासः - प्र ११ क्षे ६० । अत्र ``ह्रस्वमभीष्टं मूलमि" त्यादिना षट्टिक्षेपमूले १ । ७ ।
अधुना द्वादशक्षेपमूले १ । १ समासभावनया जाते चतुश्चत्वारिंशदधिकशतक्षेपमूले २ । १० । अतो\footnote{ जाते चतुश्चत्वारिंशत्क्षेरपमूले २ । १० अतो ।} रूपक्षेपमूले १६ । ५६ । ज्येष्ठमूले धनर्णे\footnote{धनर्ण ।} । धनस्य मूलं धनमृणं वा भवति । परिकर्मसु उक्तवदत्र धनर्णयोरविशेषः । तथा हि -``इष्टकृतिगुणकशेषोद्धृतमि" ति इष्टं\footnote{शेषोद्धृतः छत मितीष्टं ।} रूपं १ अस्य कृतिः\footnote{कृति ।} १ प्रकृतिर्विशोधिता शेषं १२ । अनेन द्विगुणमिष्टं २ हृतं जातं कनिष्ठं १६, अतो ज्येष्ठं ५६ । एवमिष्टवशादनन्तमूलानि ।
तथा हि\footnote{तथाहि रूपमिष्टं १ अस्मात्प्रकृतिं विशोध्य शेषं १२ अनेन द्विगुणमिस्टं २ हृतं जातं कनिष्ठं ३ वा ज्येष्ठं ३ एवमिष्टवशादानत्यः ।}-
प्रक्षेपेषु बहुषु (वा) शुद्विसु वा\footnote{बहुत्वा शुद्धषु ।} निजधिया पदे ज्ञेये ।
रूपक्षेपाय तयोर्भावनयाऽनन्तमूलानि ॥ ८४ ॥
यस्य न बुद्धिः स्वांते न गणितलेशोऽपि तस्य\footnote{बुद्धिः स्वतेन गणितं कलशोपि तस्यं ।} स्यात् ।
तस्मान्निजया बुद्ध्या समूह्य\footnote{तस्मान्निजधिया बुद्ध्वा समुह्य । } मखिलं तु गणितमिदम् ॥ ८५ ॥
उदाहरणम् -
कस्त्रयोदश(सं) निघ्नो वर्गः सप्तदशाधिकः ।
वर्जितो वा पृथङ्मूलप्रदः\footnote{पृशूलप्रदः ।} स्याद् वद मे द्रुतम्\footnote{त्रतं } ॥ ४५ ॥
न्यासः - प्रकृतिः १३ क्षे १७ । अत्र रूपत्रयक्षेपमूले १ । ४ क्षे ३ । अत्र बुद्धिः । क्षेपगुणं क्षेपं प्रकल्प्य\footnote{प्रक्षिप्य ।} (न्यासः) - प्र १३ क्षे ५१ । अत्रैकपञ्चाशत्क्षेपमूले १ । ८ क्षे ५१ । अनयोः पूर्वमूलाभ्यां सह भावनार्थं न्यासः - प्र १३ क १ ज्ये ८ क्षे ५१ 
क १ ज्ये ४ क्षे ३ 
समासभावनया त्रिपञ्चाशताधिकशतक्षेपमूले\footnote{त्रिपञ्चाशत्क्षेपमूले ।} १२ । ४५ क्षे १५३ । ``ईप्सितवर्गविहृतः क्षेप" इति सप्तदशसंख्यः क्षेपो यथा भवति तथा कल्पितः इष्टः त्रयरूपः (३) (अस्य) वर्गः ९ अनेन हृतः क्षेपः १७ । ``पदे तद्ष्टाप्ते" इति त्रिहृतमूले सप्तदशक्षेपे क ४ ज्ये १५ क्षे १७ । अन्तरभावनया प्राग्वज्जाते सप्तदशक्षेपमूले क ४३ ज्ये १९-३ क्षे १७ ।
द्वितीयोदाहरणे न्यासः - प्र १३ क्षे १७ । प्राग्वज्जाते सप्तदशक्षेपमूले (४ । १५ क्षे १७ । रूपशुद्धिमूलाभ्यामाभ्यां ५ । १८ क्षे १ समासभावनया जाते सप्तदशशुद्धौ मूले)  १४७ ।२३० । अन्तरभावनया जाते ३ । १० ।
अत्र (अ) मूलराशेरासन्नमूलानयने सूत्रम् -
मूलं ग्राह्यं यस्य च रूपक्षेपजे पदे\footnote{पदं ।} तत्र ।
ज्येष्ठं ह्रस्वपदेन च समुद्धरेनमूलमासन्नम् \footnote{पदेनोद्धरे ।} ॥ ८६ ॥
उदाहरणम् -
दशानामपि रूपाणां पञ्चमांशस्य वा वद ।
आसन्नमूलं जानासि यदीमां प्रकृतिक्रियाम् ॥ ४६ ॥
न्यासः - प्रकृतिः १० । अत्र रूपक्षेपमूले ६ । १९ वा २२८ । ७२१ वा ८६५८ । २७३७९ ।\footnote{अत्र क्षेपमूले ६ । ६ वा ३२८ । ७२१ वा ८६५६ ।} अल्पेनानल्पमुद्धरेदिति दशानामासन्नमूलं १९ -६ वा ७२१ २२८ वा २७३७९ -८६५८ ।\footnote{वा ३७३७९ ।}
द्वितीयस्य न्यासः - प्रकृतिः १५ । अत्रेष्टेनानेन १२ प्राग्वद्रूपक्षेपमूले २० । ९ वा ३६० । १६१ वा ६४६० । २८८९ ।\footnote{मूले ९ । २० । वा ३६० । ३६१ वा ६४५ रूप ६० । २८८६ ।} अत आसन्नं पञ्चांशस्य मूलं ९ २० वा १६१३६० वा २८८९-६४६०।\footnote{मूलं ९-२० वा १६०-१६१  वा २८८९-६४४० ।}
इति सकलकलानिधिरसिंहनन्दन -गणितविद्याचतुरानन-नारायणपण्डितविरचिते बीजगणितावतंसे वर्गप्रकृतिः समाप्ता ।
[समाप्ता बीजक्रिया ।]
बीजम्\footnote{अतः अभृति बीजम् ।}
यस्मादेतत्सकलं\footnote{यस्मात्ततसकलं ।} विश्वमनंतं प्रजायते व्यक्तम् ।
अव्यक्तादपि बीजाच्छिवं\footnote{वीजा छिवं ।} च गणितं च तं\footnote{सदा} नौमि ॥ १ ॥
अव्यक्तसमीकरणं वर्णसमत्वं च मध्यमाहरणम् ।
भावितसमत्वमस्मिन् बीजानि वदन्ति चत्वारि ॥ २ ॥
तत्रादौ तावदव्यक्तसाम्ये करणसूत्रमार्यापञ्चकम् -
यावत्तावच्चिह्रितमेकं वा बहुमितं तु परिकल्प्य\footnote{मितं रूपरीकल्प्य ।}।
रूपाढ्यं वा रूपोनितमथवाऽव्यक्तमानमिति ॥ ३ ॥
माने तस्मिन्नेवोद्देशालापवत्\footnote{तस्मिन्नुवदंशालाएकव } समाचरेत्\footnote{तौ ।} कर्म ।
फलसिद्ध्यै द्वौ\footnote{समाहरेत् ।} पक्षौ तुल्यौ कार्यौ प्रयत्नेन ॥ ४ ॥
एकस्मादव्यक्त\footnote{एकैकस्मा ।} विशोधयेदन्यतस्तु रूपाणि ।
शेषेणाव्यक्तेन च समुद्धरेद्रूपशेषमिह ॥ ५ ॥
अव्यक्तस्य\footnote{अव्यक्त ।} च राशेर्मानं व्यक्तं प्रजायते नूनम् ।
अज्ञातेषु बहुषु वा यावत्तावद् द्विकादिसंगुणितम् ॥ ६ ॥
भक्तं रूपैर्युक्तं विवर्जितं वा प्रकल्पयेदेवम् ।
निजबुद्ध्या विज्ञेयं क्वचिदव्यक्तस्य च मानम् ॥ ७ ॥
यस्मिन्ने को वर्णो वर्तते तदव्यक्तसाम्यं बीजं, यत्र बहवो वर्णा यावत्तावत्कालकादयो वर्तन्ते तदनेकवर्णसाम्यं\footnote{तदचेक ।}, यत्र च वर्णानां वर्गादिसमीकरणं क्रियते तन्मध्यमाहरणं, यत्राव्यक्तवर्णानां भावितसमीकरणं तद्भावितसमत्वमिति बीजचतुष्टयमाद्याः प्राहुः । तेषु प्रथमे तावदव्यक्तसाम्ये यत्रोदाहरणे योऽज्ञातो राशिस्तन्मानं यावदेकं द्व्यादि वा सरूपमरूपं वा रूपैरूनं युतं वा प्रकल्प्य तस्य राशेर्वासनानुसारेणोद्दे शकालापवद्योगवियोगगुणनभजनत्रैराशिकपञ्चराशिकश्रेढीक्षेत्रखातादिक्रियया\footnote{रेणोद्देशः काल्यायव ।} द्वौ पक्षौ समौ कार्यौ । यस्मिन्नालापे पक्षयोः समत्वं न विद्यते तदेकपक्षः केनचित्संगुणितो भक्तो युतो वर्जितो वा निजबुद्ध्या पक्षौ समौ कार्यौ । समपक्षयोरेकपक्षाव्यक्तमन्यपक्षाद्विशोध्य अन्यपक्षरूपाणि\footnote{व्यक्तवर्गादिकमपि इतरेपक्षरूपाणि ।} इतरपक्षरूपेभ्यो विशोध्य\footnote{रूपेभ्यो पिशोध्य ।} तत्र यद्रूपशेषमव्यक्तशेषेण हरेत् । यल्लक्षितं तदेकस्याव्यक्तस्य\footnote{क्तस्व ।} मानं व्यक्तं तेन कल्पितराशिरुत्थाप्य\footnote{राशिरुपाप्य ।} व्यक्तो राशिः क्रियते । यत्र द्वयादयोऽव्यक्तराशयो भवन्ति तत्रैकमव्यक्त शेषं द्व्यादिभिरभीष्टैर्गुणितं भक्तं वा इष्टै रूपैरूनं युतं वा यावतावत्प्रकल्पयेत् । अथवा, एकमव्यक्तं शेषाणि व्यक्तानि मानानि कल्प्यानीति ।
उदाहरणानि -
समानमौल्या वणिजोऽष्टघोटा 
एकस्य रूपाणि शतानि षट् च ।
ऋणे\footnote{ऋणं ।} शतेऽन्यस्य च वाजिनोऽर्क-
मिताः समौ तौ\footnote{मिताः स्वस्वौ ।} च किमश्वमूल्यम् ॥ १ ॥

(अपूर्णम् )


