\documentclass[twoside,12pt,openany]{book}
तिथिनिर्ण्णयः 
॥ विष्णुं विश्वेश्वरं नत्व तदुपोषणशुद्ध्ये ।
मूलग्रन्थानुसारेण क्रियते तिथिनिर्ण्णयः ॥ १ ॥
\footnote{}भूश्रीभिन्नाकिचिन्त्यो(१६,१०,४२४) नात् 
कल्यहात् काल (३१) वर्द्धितात् ।
गरुडध्येय (११,३२३) वाक्याप्तं 
त्यक्त्वा सौरं वृथारफलम् ॥ २ ॥

ग्रन्थेऽयमेकस्मिन् मूर्लैकोशावलम्बेन लिखिते श्रीपेजावरमठीये प्राचीनकोशे उपलब्धः । श्रीपलिमास्मठीये तिथिविमर्शग्रन्थे च प्राचीने- ``आचार्यैस्तथैव तिथिनिर्ण्णयेऽभिहितमित्यार्येणैव विनिर्ण्णयः" इत्युक्तम् । सब्प्रति तु बहुषु स्थलेष्वन्यान्यमठेषु चास्य हस्तलिखितानि पुस्त कान्युपलब्धानि । तस्मादस्याऽचार्यकृतितानिर्ण्णयः । तदयमप्याचार्याणामपूर्वो ग्रन्थ इदानीमिदमुपज्ञमुपस्थाप्यते ।
\footnote{} एतेनाऽचार्याणां कालविवयोऽपि विवादः परिहृतो भवति । ``चतुःसहस्रे निशतोत्तरे गते" इत्याचार्यै स्वयमेवोक्तम् । तस्मात् स एव काल इत्येके । ``त्रिशताब्दोत्तरचतु सहस्राब्देभ्य उत्तरे । एकोनचच्वारिंर्शाब्द्धेः" इत्यन्यग्र वचनात् एकोनचत्वारिंशदुत्तरत्वं च तात्पर्यनिर्ण्णये पूरणीयमित्यपरे । शेषं विहाय शतकमात्रपरिगष्नेन निर्ण्णंयोक्तिरिति तेषां मतम् । अतयोर्द्वयोः पक्षयोः कः साधीपान् ? द्वितीय इति सम्प्रति सिद्धम् । तथाहि - पद्ब्रन्थ करेणकाले कलिगताहानि - १६,१०,४२४ (भूश्रीभिन्नाकिचिन्त्य) इति स्वयमेवाभिहितम् । तथाच वदार्ना कलिगताब्दाः -
१६,१०,४२४ X ३१ = (४४८९ वर्षाणि)
११, ३२३
यदि ४३०० गताब्दे भगवत्पादाविर्भावः ततः १०९ वर्गनन्तरमेतद्बन्थकरणं भवति ।  नूच वन भवति । ``एकोनाशीतिवर्षाणि भूत्वा मानुषदृष्टिगः। पिङ्गलाब्दे माघशुद्धैनबम्यां बदरीं ययौ" इत्यादिवचनविरोधात् । ततश्च पूर्वोक्तरीत्या ३९ अब्दपूरणे तु सर्वं सुसङ्गते भवति । अवतारकालः ४३३९ कलिगताब्दाः । ततः सप्तनिवर्षानन्तरं (४३३९+७० =४४०९) एतद्ब्रन्थरचनमिति । तथाच बदरीनिर्याणान्नववर्षपूर्वतनोऽयं ग्रन्थः ।
राश्याद्यं मद्ध्यमं कुर्याद् 
गो(३)घ्नाद् धीसूनुनाग (३०,०७९)जाः ।
कलास्त्यक्त्वा ध्रुवं कुर्याद् 
देशा-धार-हरा-र्पकम् (११-२८-२९-५८) ॥ ३ ॥
अत्र यत् क्वचित् कोशेषु ``भूगीभिन्नाकि-" इति ``कल्यब्दात्" इति, ``कालवर्जितात्" इति च पठ्यतं तद् गणितविरोधाल्लिपिकृतां प्रमादः । तथाहि क्रमः-
साम्प्रतं कल्यहम् - १८,५३,६१४ (प्रमादिवत्सरांये माघशुद्धे मघ्वनवमीदिने-शक-१८९६) एतद्ब्रन्थकरणकालिनकल्यहम् - १६,१०,४२४ (भूश्रीभिन्नाकिचिन्त्य) 
तदूनं कल्यहम् - २,४३,१९०़
तस्य च कालेन (३१) वृद्धिः । गरुडध्येयेन (११, ३२३) विभागश्च । तथाहि क्रमः-
२,४३,१९० X ३१
११,३२३)१५,३८,८९० (६६५
६७,९३८
७४,५०९
६७९३८
६५७१०
५६६१५
९०९५
अत्र लब्धं ६६५ सौरभगणापरनामकं सौरं, वृथाफलमित्युच्यते । तथाचैतद्ग्रन्थकरणात् परमद्ययावत् ६६५ वर्षाण्यतीतानीति स्थितम् (४४०९+६६५=५०७४ साम्प्रतः कलिगताब्दः) । तदेतद् फलं त्यक्त्वा अवशिष्टेन (९०९५) राश्याद्यं मद्ध्यमं कुर्यादिति ।
१ अत्रापि श्लोके यत् क्वचित् ``धीसूनुनागणाः" इति, ``कलाश्चत्वा" इति, ``देशाधारनराप्रकम्" इति च पठ्यते तल्लिऽपकृतामेव प्रमादः । तथाहि मद्ध्यमरव्यादिनिष्पत्तिक्रमः-
तत्र राशिलब्ध्यर्त्यमवशिष्टं द्वादशभिर्हन्यात्, ततश्च गरुडध्येयेन विभज्यात् ।
१०१५X१२
११३२३)१०९१४०(१ राशिः
१०१९०७
७२३३
ततो राशिभागार्त्थं अवशिष्टं त्रिंशता हन्यात्, गरु़डध्येयेन विभज्यात् -
७२३३X३०
११३२३)२१६९९०(१९ भागाः
११३२३
१०३७६०
१०१९०७
१८५३
भागनां प्रत्येकं षष्टिः कला इति कलार्त्यं पष्ठ्या हन्यात्, गरुडप्येयेन विभजेत् -
१८५३X६०
११३२३)१११९८०( ९ कलाः 
१०१९०७
९२७३
कलानां षष्टिर्विकला इति विकलार्त्यं पुनः पष्ट्या हन्तात्, गरुडध्येयेन विभजेच्च-
१२७३X६०
११३२३)५५६३८०(४९ विकलाः
४५२९२
१०३८६०
१०१९०७
१५५३
ततथ लब्धा राश्या द्यः- (९ ए. १९ भा ०९ क ४९ वि )। एतांश्च गो (३) गुणितात् भूश्रीभिन्नाकिचिन्त्योनात् (२४३१९०) धीसूनुनागे (३००७९) न जाताः विभज्याऽप्ताः याः कलास्ताभिर्हापयेत् । ततश्च मद्ध्यमरविलाभः । तथाहि -
२४३१९०X३
३००७९)७२९५७०(२४ कलाः ।
६०१५८
१२७९९०
१२०३१६
७६७४

७६७४X६०
३००७९)४६०४४०(१५ विकलाः
३००७९
१५९६५०
१५०३९५
९२५५

अनंन्त (६००) वृद्धाद् बौधाङ्ग-
तुल्ये(१६,३९३) नेन्दुः शुका (१५) हतात् ।
प्राज्ञाञ्जलिभृ(४३,८०२) दाप्तोन-
स्तार-शोभा-तिना-किनी (०१-७६-६५-२६) ॥ ४ ॥
रा. भा. क. वि 
०९ १९ ०९ ४९ -
		 २४ १५
०९ १८ ४५ ३४
भीश्रीभिन्नाकिचिन्त्याहे ध्रुवं राश्याद्यं ग्रन्थकृतैवोक्तम् - देशाधारहरार्पकमिति ।
											रा. भा. क. वि.
तथाच १६१०४२४ दिने मद्ध्यरविः -	११ - २८ - २९ - ५८ +
		२४३१९०		"		"		-	०९ - १८ - ४५ - ३४
१८५३६१४ मध्वनवमीदिने  " 		-	०९ - १७ - १५ - ३२
१ अत्र श्वोके ``अनन्तवृद्धात् त्वौधाङ्ग" इति, ``तुलेनेन्दुशुकाहतात्" इति च यत् क्वचित् कोशान्तरे पठ्यते स लिपिकृत्प्रमादः । चतुर्त्थचरणे ``तारेशोभातिनाकनी" इति क्वचित् पाठः । उभयथाऽपि सङ्ख्यायामविरोधः । मद्ध्यचन्द्रकरणविधिरत्र निरूपितः । तथाहि क्रमः- भूश्रीभिन्नाकिचिन्त्योनस्य (२४३१९०) अनन्ते (६००) न वृद्धिः बोवाङ्गतुल्ये (१६३९३) न विभागश्च । ततः पूर्ववद् राश्यादिकरणम् । ततश्च भूश्रीभिन्नाकिचिन्त्योनस्य शुके (१५) न गुणितस्य प्राज्ञाञ्जलिभृता (४३८०२) विभागे यल्लब्धं तावत्यः कलास्ततो हापयेत् । तेन भूश्रीभिन्नाकिचिन्त्योना हस्य (२४३१९०) मद्ध्यचन्द्रलाभ; । भूश्रीभिन्नाकिचिन्त्याहस्य (१६१०४२८) मथ्यचन्द्रस्तु आचार्यैरेवाभिहितः- ``तार-शोभा-तिना-किनी"(०१ रा. ०६. भा. ४५ क. २६ वि.) इति ।
तथाच गणितक्रमः - 
२४३१९०X६००
१६३९३)१४५९१४०००(८९००
१३११४४
१४७७००
१४७५३७
१६३००
दिनेभ्यो द्रागरागा(३,२३२)प्तः 
चन्द्रोच्चः स्यान्निभा(४०) हतात् ।
१६३००X१२
१६३९३)१९५३००(११ राशिः 
१८०३२३
१५२७७X३०
१६३९३)४५८३१०(२७ भागाः
३२७८६
१३०४५०
११४७५१
१५६९९X६०
९४१९४०

१६३९३)९४१९४०( ५७ कलाः
८१९६५
१२२२९०
११४७५१
७५३९X६०
१६३९३)४५२३४०(२७ विकलाः 
३२७८६
१२४४८०
११४७५१
९७२९

२४३१९०X१५
४३८०२)३६४७८५०(८३ कलाः =
३५०४१६७ (१भा. २३क.)
१४३६९०
१३१४०६
१२२८४X६०
४३८०२)७३७०४०(१६ विकलाः
७००८३२
३६२०८

रा. भा. क. वि.
११ - २७ - ५७ - २७ -
		१ - २३ - १७
११ - २६ - ३४ - १०
तथाव १६१०४२४ दिने मद्ध्यनन्द्रः		१ - ०६ - ४५ - २६ +
		२४३१९०	"		"			 ११ - २६ - ३४ - १०
		१८५३६१४ दिने 	"				१ - ०३ - १९ - ३६
१ ``दिनेन" इति पाठान्तरम् । यत् क्वचित् ``दिनेभ्यो दांगरागाप्त". इति, ``दिनेन पाद्रान्नरागार्तः छन्दोच्चः" इति च पठ्यते स लिपिकृत्प्रमादः ।
जगत्सेनाङ्ग (३०,७३८) लब्धोनः
श्रेष्ठ-चिन्त्यो-म्बुना-र्चने (०६-०३-१६-२२) ॥ ५ ॥
अनेन श्लेकेन चन्द्रोच्चकरणविघिमाहाऽचार्यः । भूश्रीभिन्नाकिचिन्त्येनेभ्यो दिनेभ्यः द्रागरागे(३२३२) न विभागे आप्तः राश्यादिः, निभ (४०) गुणितात् तस्मादेवाह्नः जगत्सेनाङ्गैः (३०७३८) विभागे लब्धाभिः कलाभिरूनश्च चन्द्रोच्चः स्यात् इति । तथाहि -
३२३२)२४३१९०(७५
२२६२४
१९६९५०
१६१६०
७९०

७९०X१२
३२३२)९४८०( २ राशिः
६४६४
३०१६X३०
३२३२)९०४८०(२७ भागाः
६४६४
२५८४०
२२६२४
३२१६

३२१६X६०
३२३२)१९२९६०(५९ कलाः		
१६१६०
३१३६०
२९०८८
२२७२

२२७२X६०
३२३२)१३६३२०( ४२ विकलाः
१२९२८
७०४०
६४६४
५७६
२ - २७ - ५९ - ४२ -
	  ५ - १६ - २८
२ - २२ - ४३ -१४
भूश्रीभिन्नाकिचिन्त्याहस्य चन्द्रोच्चं 
आचार्यैरेवोक्तम् - `श्रेष्ठ-चिन्त्यो-म्बुना-
र्चने' (०६-०३-१६-२२) इति ।

२४३१९०X४०
३०७३८)९७२७६००(३१६ कलाः = 
१२२१४	५-१६
५०६२०
३०७३८
१९८८२०
१८४४२८
१४३९२X६०
३०७३८)८६३५२०(२७ विकलाः
६१४७६
२४८७६०
२४५९०४
२८५६

लङ्कास्वाम्या(?वन्त्या) दिरेखायाः पूर्वपश्चिमदेशमोः ।
ग्रहाणां मद्ध्यसंस्कारलिप्ता ऋणधनं क्रमात् ॥ ६ ॥
पाप(११)घ्नादध्वसङ्ख्याना-
दर्क्क(१०) लब्धविलिप्तिकाः ।
अर्क्कस्येन्दोरनर्क्क(१००)घ्नात् 
सानुभू (४०७) लब्धलिप्तिकाः ॥ ७ ॥
												रा. भा. क. वि .
तथाट १६,१०,४२४ दिने चन्द्रोचम् -		०६ - ०३ - १६ - २२
		२,४३,१९० 	" 		"			    ०२ - २२ - ४३ - १४
		१८,५३,६१४	"		"				०८ - २५ - ५९ - ३६
१ अत्र सर्वत्र कोशेषु ``लङ्कास्वाम्यादिरेखायाः" इति पठ्यते । अर्त्थस्तु न स्फुटः। अर्त्थानुगुण्याय ``लङ्कावन्त्यादिरेखायाः" इति पठनीयमिति प्रतिभाति । तृतीयचरणे च ``गणानां मद्ध्यसंस्कार -" इति कचित् पाठः प्रामादिके एव । चतुर्त्थतरणे ``ऋणघनक्रमात्" इति कचित् । पूर्वदेशस्यर्ण पश्चिमदेशस्य च धनं क्रमाद् भवतीत्यर्त्थः ।
तथाचायं श्लोकार्त्थः - एतच्चोदितं मद्ध्यमं राश्याद्यं भूमद्ध्यरेस्त्रादेशे । देशान्तरे तु तदनुगुणतया पुनः संस्कारः करणीय इत्याह - लङ्केत्यादि । तत्र च का भूमद्ध्यरेस्वा विवक्षिता? आह-लङ्कावन्त्यादिरेस्वायाः । लङ्कावन्त्यादिदेशस्पृक् परितो भूमिं दक्षिणोत्तरगामिनी रेखा लङ्कावन्त्यादिरेखा । सैव चात्र विवक्षिता मद्ध्यरेखा । ततः पूर्वदेशे अध्वसंख्यानुसारेण लिप्तानां ऋणम् , कलानां वियोजनम् , पश्चिमे चाध्वसङ्क्यानुसारेण धनम् , कलानां संयोजनं क्रमात् करणीयमित्यर्थः ।
२ तथाच गणनाक्रममाह - पापघ्नात् । अध्वा केन्द्ररेखातः यावद्योजनदूरः तावतीनां संख्यानां पापेन (११) गुणनम् ; अर्क्केण(१०) विभागश्च । ततो लब्धाः विलिप्तिकाः, विकला इति यावत् , पूर्वपश्चिमदेशयोः क्रमेण वियोजयेत् , संयोजयेच्चेत्यर्थः ।
अध्वाX११ सोऽयं मद्ध्यमरवेः देशसंस्कारः ।
१०
एवं मद्ध्यमचन्द्रस्य देशसंस्कारार्थं अध्वसंख्यानस्य अनर्क्लेण (१००) गुणनन; सानुभू (४०७) विभागश्च । ततो लब्धाः,  लिप्तिकाः कला इति यावत्, पूर्वपश्चिमदेशयोः क्रमेण वियोजयेत् , संयोजयेच्चेत्यर्थः ।
अध्वाX१०० सोऽयं मद्ध्यमचन्द्रस्य देशसंस्कारः ।
४०७
तथाहि क्रमः । रजतपीठपुरे राश्याद्यं मद्ध्यमं करणीयं चेत्, मद्ध्यरेखातः पश्चिमसमुद्रपर्यन्तं सूर्यपरिक्रमस्य यावत्यः विघटिकास्तावत्य एव विकलाः मद्ध्यमे राश्याद्ये संयोजनीया इति स्थितिः । अध्वसंख्यानस्व पापेन  
अर्द्गदोर्ज्वाफलाच्छुद्धात् यथाऽर्क्ते तत्फलात् तथा ।
गो(३)ध्नाद् दिव्य(१८) प्रजा(८२) भ्यां वि-
लिप्ता लिप्ता इनाब्जयोः१ ॥ ८ ॥
गुणने, अर्केण विभागे परिक्रमविघटिकासमसंख्या विकला लभ्यन्ते । तथाहि - भूपरिधिः ३३०० योजनानीते स्थितिः । (३२९९ योजनानीति च पक्षान्तरं करणत्नकृतो देवाचार्यस्य ।) एतावान् षष्टिघटिकासु सूर्यपरिक्रममार्गः । (१ यो. = 71/2 Miles) तेन घटिकायां (३३००-६० = ५५) पञ्चपञ्चाशद् योजनानि । विघटिकायां च ५५-६० योजनानि । अत्रतु पश्चिमदेशे २४ विघटिकाः बिलम्बेन दर्शनमिति ततो गुणनेऽध्वसंख्यानलाभः -
५५X२४
२० =२२ योजनानि 
२४ विघटिकास २२ योजनानि सूर्यपरिक्रममार्ग्ग इति लङ्कावन्त्यादिरेखातः पश्चिमसमुद्रतीरस्य २२ योजनानि अध्वसंख्यानम् । तस्य च पापेन (११) गुणनेऽर्केण (१०) विभागे च विघटिकासमसंख्यविलिप्तिकालाभः -
२२X११ =	२४२	=		२४(२/१०) विलिप्तिकाः ।	
१०			१०				
तथाच २४ विकला मद्ध्यमराश्यादौ योजनीयाः -
मद्ध्यरेखायां मध्यमरविः-		९ - १७ - १५ - ३२
संस्कारार्त्थं संयोजनीया विकलाः -				२४
रजतपीठे मद्ध्यमरविः -			९ - १७ - १५ - ५६
एवं मद्ध्यमचन्द्रेऽपि पूर्वोक्तदिशा संस्कारो यथा -
२२X१००
४०७)२२००(५ कलाः
२०३५
१६५X६०
४०७)१९००(२४ विकलाः
८१४
१७६०
१६२८
१३२

तथाच मद्ध्यरेखायां मद्धयमचन्द्रः 	१ - ३ - १९ - ३६
संयोजनीयाः कलाः -								५ - २८
रजतपीठे मद्ध्यमचन्द्रः -				 १ - ३ - २५ - ००
(अत्र विभाजकं ४१७ चेत् इतोऽपि सूक्ष्मः संस्कारः स्यादिति - ज्योतिर्विदो वदन्ति । तदात्वे तु `सानुभू -' इति स्थाने `सान्यभू -(४१७)' इति पठनीयमिति भाति ।
१ अत्र ``दिष्येप्रजाभ्यां लिप्तिकालिप्त इनाब्जयोः" इति क्वचित् पाठः प्रामादिकः । ``विलाप्ता लिप्त इना ब्जयोः" इति क्वचित् । तत्रापि दीर्घलोपो लिपिकृत्प्रमादादेवेति सम्भाव्यते । 
\footnote{}देशदोःशुद्धये दानक्ष्विष्णुभ्यां स्युर्विलिप्तिकाः ।
\footnote{}उञ्चं सूर्यस्य नियतं दुष्टा स्त्री(२-१८) भागराशयः ॥ ९ ॥

\footnote{}शरीरनुत् (२२५) धीभवनः(४४९) कथञ्चनो(३७१)
नळीजनो(८९०) मानपटुः(११०५) शुकालपः (१३१५) ।
निरामयो (१५२०) धीः पथिको (१७१९) [नृपाधिको](१९१०)
बुधोनरः(२०९३) सुप्तप(?ख)रः(२२६७) कलाविराट् (२४३१) ॥ १० ॥

महाशरो (२५८५) दूरसरो (२७२८) धमीहरिः\footnote{}(२८५९)
हसन्धुरो(२९७८) वेदनगः(३०८४) सुसङ्कुलः(३१७७) ।
\footnote{}तमःखगः(३५२६) पारबलं(३३२१) [रसो बली\footnote{}](३३७२)
धनावलिः(३४०९) कालभृगु(३४३१) र्जगद्भगः (३४३८) ॥ ११ ॥
\footnote{}इमाश्चतुर्विंशतिज्याः स्फुटत्वायार्क्कसोमयोः ।
तथाचायं श्लोकार्त्थः- उक्तो देशसंस्कारः । दोर्ज्यासंस्कारः सम्प्रति कत्थ्यते -अर्क्कदोर्ज्याफलानयनविधिरु त्तरत्र १५ श्लोके स्फुटीकरिष्यते । तदेतत् जोर्ज्याफलं रविमन्द्रफलमिति चोच्यते । तत् फलं गो(३) सङ्ख्यया हत्वा दिघ्य (१८) संख्यया विभजेत् । लब्धा विलिप्तिकाः, विकला इति यावत् सूर्यन्य संस्कारार्त्थम् । तदेव ज्याफलं गो (३) संख्ययैव हत्वा प्रजा (८१) संख्यया विभागे लब्धा लिप्ताः, कला इति यावत्, चन्द्रस्य संस्कारार्त्थम् । संस्कारार्थं कि करणीयाः ?  हातव्या उत योक्तव्याः ? आह - यथाऽर्क्के तत्फलात् तथा । मेषादि तुलादि च गोळद्वयम् । तत्र अर्क्कस्य मेषादित्वे ऋणम्, तुलादित्वे धनमित्युपरिष्टाद् वक्ष्यति-स्वस्वगोळयोः सऋणं धनमिति (श्लो. १५) । दोर्ज्याफलेऽप्ययं न्यायः समानः - मेषादिगोळस्यर्णम् , तुलादिगोळस्य धनमिति । संस्कारक्रम उपरिष्टाद् भवति । 
१ ``देशयोः शुद्धये" इति क्वचित् । ``देशदोःशुचये" इति च क्वचित् । ``दानक्षिष्णुभ्याम्" इत्यपि क्वचिन् । सर्वथाऽप्यस्फुटार्त्थं श्लोकार्द्धम् ।
२ चन्द्रोच्चकरणविधिः पूर्वमुक्तः । सूर्योच्चं तु नियतमित्याह - उच्चम् । अत्र ``दृष्टा स्त्री" इति क्वचित् । 
उभयथाऽपि सङ्ख्यायां न विरोधः । तथाच सूर्योच्चम् - रा.	भा. 	क. 	वि.
क्वचिच्च ``भागनाशयोः" इति पाठः प्रामादिक एव ।  ०२ - १८ - ०० - ००
३ इदानीं दोर्ज्यासंस्कारं निरूपयितुमादौ चतुर्विंशतिज्या आह द्वाभ्यां श्लोकाभ्याम् - शरीरनुदित्यादिना । अत्र ``शरीररद्धीभवनः" इति कवचित्प्राठः प्रामादिकः । उत्तरार्द्धे च ``निरालयो-" इति पाठः प्रामादिकः । ``नृपाधिको-" इति कोशेषु लुप्तम् । सङ्ख्यानुगुण्याय च पूरितम् । ``सुप्तपरः" इति कोशेषु पठ्यते । संख्यानुगुण्याय ``सुप्तखरः" इति पठितम् ।
४ ``वमीहरिः" इति क्वचित् पाठः प्रामादिकः ।
५ ``नमःखगः" इति च क्वचित् पाठः प्रामादिकः ।
६ ``रसो बली" इति चतुरक्षराणि कोशेषु लुप्तानीति सङ्ख्यानुगुण्याय पूरितानि ।
७ ज्यार्द्धमेवात्र ज्याशब्देनोच्यते । तथा ह्युक्तम् - ``अर्द्धज्यैव ज्याभिधानाऽत्र वेद्या" इति । अत एवोत्तरव ``शुभाङ्गपरिमाणेन यदि ज्यार्द्धं न पूर्यते" इत्याह ।

चतुर्विंशतिवाक्यानि त्रिराशीनामिमान् विदुः\footnote{} ॥ १२ ॥
\footnote{}शुभाङ्ग(३-४५) परिमाणेन यदि ज्यार्द्धं न पूर्यते ।
वर्त्तमानज्यया ह(?हि)त्वा मुरारि (२२५) फलसङ्ग्रहः ॥ १३ ॥

१एकस्मिन राशावष्टौ ज्या इति त्रिराशीनामाहत्य चतुर्विशतिर्ज्या भवन्ति । पुनश्चतुर्त्थादारभ्य षष्ठपर्यन्तमिमा एव चतुर्विंशतिर्ज्याः । सप्तमान्नवमपर्यन्तम् , नवमाच्चाऽद्वादशमिति । ओजयोः क्रमेणानोजयोर्विपर्ययेणेति विशेषः । तथाच मेषादित्रिकं राशिचक्रस्य प्रथमः पादः । कर्क्कादिकन्यान्तं त्रिकं द्वितीयः पादः । तुलादित्रिकं तृतीयः पादः । मकरादिमीनान्तं त्रिकं चतुर्त्थः पाद इति । तदेतत् समनन्तरश्लोके स्फुटम् - ``राशिचक्रं चतुष्पादम्" इति ।
तथाचैवं क्रमेण ज्याः -
१) शरीरनुत् (२२५)	
२) धीभवनः (४४९)
३) कथञ्चनः (६७१)
४) नळीजनः (८९०)
५)मानपटुः (११०५)
६) शुकालयः (१३१५)
७) निरामयः (१५२०)
८) धीःपथिकः (१७१९)
९) नृपीधिकः (१९१०)
१०) बुधोनरः (२०९३)
११) सुप्तखरः (२२६७)
१२) कलाविराट् (२४३१)
१३) महाशरः (२५८५)
१४) दूरसरः (२७२८)
१५) धमीहरिः (२८५९)
१६) हसन्धुरः (२१७८)
१७) वेदनगः (३०८४)
१८) सुसङ्कुलः (३१७७)
१९) तमःखगः (३२५६)
२०) पारबलम् (३३२१)
२१) रसो बली (३३७२)
२२) धनावलिः (३४०९)
२३) कालभृगुः (३४३१)
२४) जगद्भगः (३४३८)
२ अत्र श्लोके क्वचित् कोशे ``यदि ज्यार्त्थम्" इति पाठः प्रामादिकः । तृतीयपादे च यद्यपि सर्वत्र ``वर्त्तमानज्यया हत्वा" इति पठ्यते । `हित्वा' इति पठनीयमिति भाति ।
तथाचायमर्त्थः-
एकैकस्य राशेस्त्रिंशतिर्भागा इति त्रिराशीनां नवतिर्भागा भवन्ति । तत्र च चतुर्विंशतिर्ज्याः । तथाचैकैकस्य ज्यायाः परिमाणं शुभाङ्ग (३-४५) सम्मितं भवति । तदेतच्छुभाङ्गपरिमाणमित्युच्यते -
२४)९०( ३ भागाः 
७२
१८ X ६०
२४)१०८०( ४५ कलाः
१६
१२०
१२०
यदि च निश्शेषं ज्यार्द्धं न पूर्यते तदाऽतीतज्यां वर्त्तमानज्यया हित्वा शेषं मुरारि (२२५)णा विभजेत् । तल्लब्धस्य फलस्यातीतज्यया सङ्ग्रहे निष्कृष्टज्यालाभ इति । गणितक्रम उपरिष्टाद् भविष्यति ।
(३ X ६० = १८० + ४५ = २२५ कलाः)

\footnote{}राशिचक्रं चतुष्पादमोजानोजद्विपादयोः ।
\footnote{} अतीतानागतौ भागौ भुज इत्युच्यते बुधैः ॥ १४ ॥

\footnote{} स्वोच्चोनार्क्काब्जयोद्देर्ज्याि गो (३) स (७) द्भ्यां वर्द्धिताः क्रमात् ।
अज (८०) लब्धकलाः स्वस्वगोळयोः सऋणं धनम् ॥ १५ ॥

१ - दोर्ज्येत्यत्र दोरिति किमुच्यते ? आह - रासिचक्रम् । राशि वक्रं पूर्वोक्तिदशा चतुष्पादम् । तत्र मोपादित्रिकं तुलादित्रिके चेति द्वौ पादौ ओजौ । कर्क्कादित्रिकं मकरादित्रिकं चेति द्वौ पादाबनोजौ, युग्माविति यावत् । तत्र ओजद्विपादयोः अतीतः पादो भुज इत्युच्यते । अनोजद्विपादयोरनागतः पादो भुज इत्युच्यते वुधैरिति । एतदप्पु त्तरत्र गणिते स्फुटीभविष्यति ।
२ क्वचित् ``अतीतानागतौ पादौ" इति पाठः ।
३ ``सोच्चोनार्क्का-" इत्यादिः कोशपाठः प्रामादिकः । स्वोच्चोनार्क्कस्य गो(३) सङ्ख्यया गुणनम् , अजेन (८०) विभागश्च । स्वोच्चोनचन्द्रस्य सत् (७) सङ्खयया गुणनम् । अजेन (८०) विभागश्च । लब्धाः कलाः मेषादित्वे ऋणमिति मद्ध्यमराश्यादेर्हापयेत् । तुलादित्वे धनमिति मद्ध्यमराश्यादेर्योजयेत् । तथाचायं गाणितक्रमः - 
सूर्यस्फुटं यथा -
पूर्वोक्तदिने मद्ध्यमरविः - 	९ - १७ - १५ - ५६ -
				ख्युच्चम् -		२ - १८ - ०० - ००
		स्वोच्चोनरविः -		६ - २९ - १५ - ५६ रविमद्ध्यकेन्द्रम् ।
षड् राशयोऽतीताः । तथाच तृतीयं त्रिकमोजपादः । उक्तं हि - ``राशिचक्रं चतुष्पादमोजानोजद्विपादयोः" इति । ओजस्यातीतो भागो भुज इत्यप्युक्तम् - ``अतीतानागतौ भागौ भुज इत्युच्यते बुधैः" इति । तथाचात्र भुजांशः विकला विहाय - २९ भागाः १५ कलाः । भागानामपि कलात्वेन परिवर्त्तने आहत्य - २९ X ६० = १७४० + १५ = १७५५ कलाः । तस्य च भुजांशस्य शुभाङ्गपरिमाणेन (२२५) विभागः- 
२२५)१७५५( ७
१५७५
१८०
सप्तमं ज्यार्द्धं पूर्ण्णम । ततः १८० कला अवशिष्टाः । ततः वर्त्तमानज्ययाऽष्टम्याऽतीतां सप्तमीं ज्यां हित्वा मुरारि (२२५) फलसङ्ग्रहे निष्कृष्टज्यालाभः । तदेतदुक्तम् - ``शुभाङ्गपरिमाणेन यदि ज्यार्द्ध न पूर्यते । वर्त्तमानज्यया हित्वा मुरारिफलसङ्ग्रहः" इति ।
वर्त्तमानज्या (धीःपथिकः) 	१७१९
अतीतज्या (निरामयः) 		१५२०
								  १९९
१९९ X १८० = ७९६ = १५९ (१/५)
	२२५			 ५			
तथाच अतीतज्यायः १५९ कलाभिः सङ्ग्रहे निष्कृष्टज्यालाभः ।
१५२०
१५९
१६७९ 
तथाच निष्कृष्टज्या १६७९ । तस्या गो (३) सङ्ख्यया गुणनम् , अज (८०) सङ्ख्यया विभागः । तदेतदुक्तम् - ``दोर्ज्या गोसद्भ्या वर्द्धिताः क्रमात् । अजलब्धकलाः स्वम्वगोळयोः सऋणं धनम्" इति ।
१६७९ X ३
८०)५०३७( ६३ कलाः = (१ भा. श्क.)
	४८० तदेतत् दोर्ज्याफलमित्युच्यते ।
	२३७
	२४०
मेषादित्वे ऋणम् , तुलादित्वे धनमिति चोक्तम् ``स्वस्वगौळयोः सऋणं धनम्" इति । अस्य च तुलादित्वाद्धनमेव । ततश्चैताः कलाः मद्ध्यमस्य राश्यादेर्योजनीयाः-
९ - १७ - १५  - ५६ 
	०१  - ०३ 
९ - १८ -  १८ - ५६
ततः दोर्ज्यासंस्कारार्थं दोर्ज्याफलात् गो(३) घ्नात् दिव्येन (१८) विभागे लब्धा विलिप्ताः स्वस्वगोळयोः सऋणं धनमित्युक्तम् - ``अर्क्कदोर्ज्याफलाच्छुद्धाद् यथाऽर्क्के तत्फलात् तथा । गोव्नाद् दिव्यप्रजाभ्यां विलिप्ता लिप्ता इनाब्जयोः" इति तथाच दोर्ज्याफलम् -
एता विकला अपि तुलादित्वाद्योजनीयाः ।
तद्यथा - ९ - १८ - १८ - ५६ 
							  १०
		   ९ - १८ - १९ - ०६ रविस्फुटम् ।
६३X३
१८)१८९(१० विकलाः 
१८०
अथः चन्द्रस्फुचं यथा -
मद्ध्यमचन्द्रः - ०१ - ०३ - २५ - ००
तस्य अर्क्कदोर्ज्यांफलेन संस्कारः । तथाहि अर्क्कदोर्ज्याफलम् - ६३कलाः । तस्य गो (३) सङ्क्यया गुणनम् , प्रजा (८२) सङ्खयया विभागः । लब्धाः कला योजनीयाः । उक्तं हि ``यथाऽर्क्के तत्फलात् तथा । गोध्नाद् दिव्यप्रजाभ्यां विलिप्ता लिप्ता इनाब्जयो." इति । तथाच अर्क्कदोर्ज्याफलम् X गो = ६३ X ३ = १८९
																		प्रजा 						८२		८२
८२)१८९( २ कलाः 
	१६४
	२५ X ६०
८२) १५००( १८ विकलाः
		८२
		६८०
		६५६
०१ - ०३ - २५ - ००
			 ०२ - १८
०१ - ०३ - २७ - १८ दोर्ज्याफलसंस्कृतो मद्ध्यमचन्द्रः ।
[तदेकदिनगा लिप्ता ग्रहाणां स्वस्वभुक्तयः\footnote{}]
\footnote{}कल्यब्दौघो धेनुभवो(४४०९) युक्तः सौरैर्वृथाफलैः ।
एतस्मान्मापते (६१५) ल्लब्धं राश्याद्यायनमुच्यते ॥ १६ ॥

\footnote{}प्रभारत्नं(२४२) धीसवनं (४७९)
गानस्थानं (७७३) जनेधनम् (९०८) ।
देहिनित्यं (१०८८) सुगप्रायं (१२३७)
सावलोक्यं (१३४७) तटिद्वपुः (१४१६) ॥ १७ ॥

१. इदमर्द्धमुपरिष्टाद् (श्लो. २०) भविष्यति । पुनरत्रापि पठनं पुनरुक्तमिति भाति । लिपिकृतत्प्रमादः स्यात् । 
२ अयनांशनिर्ण्णयः सम्प्रति क्रत्थ्यते -
सौरैः वृथाफलैः द्वितीयश्लोकोक्तैः (६६५) युक्तः धेनुभवः (४४०९) सम्प्रति गतः कल्यब्दौघो भवति । ४४०९ + ६६५ = ५०७४ तस्य मापते (६१५)र्विभागे राश्पाद्यायनलाभः ।
६१५)५०७४( ८ राशयः 
		४९२०
		१५४ X ३० 
		४६२० 
		
६१५)४६२०( ७ भागाः 
	४३०५
	३१५ X ६०
६१५)१८९००( ३० कलाः 
१८४५०
४५०

४५० X ६०
६१५)२७०००(४४ विकलाः 
२४६० 
२४००
२४६०
तथाचायनांशः - ०८ - ०७ - ३० - ४४ 
३ अस्य दोर्ज्यासंस्कारार्त्थं नववाक्यान्याहानेन । अत्र `धीवसनं' इति क्वचित् कोशपाठः प्रामादिकः । तथा तृतीयवरणे - ``सुप्रमयम् " इति च । एतानि वाक्यानि भगवत्पादः मूलग्रन्थेभ्य उद्दधारेत्यवगम्यते । अत एव भागवत्पादात् प्राचीने (क्रि. श. ६८९) देवचार्यनाम्ना विदुषा विरचिते करणरत्नाख्येऽपूर्वतमे ग्रन्थेऽपि एतान्येव वाक्यानि पठ्यन्ते ।
तथाचैवं क्रमेण नव ज्याः -
१) प्रभारत्रम् (२४२)
२) धीसवनम् (४७९)
३) गानस्थानम् (७०३)
४) जनेधनम् (९०८)
५) देहिनित्यम् (१०८८)
६) सुगप्रायम् (१२३७) 
७) सावलोक्यम् (१३४७)
८) तटिद्वपुः  (१४१६)
९) नवभार्या (१४४०)
 
 पूर्वोक्तस्यायनांशस्य (०८-०७-३०-४४) ओजपादत्वादतीतः पादांशो भुजः । तथाच विकलां विहाय भुजांशः - ०२ -०७ -३० तस्य कलात्वेन परिवर्त्तने २X३० = ६० + = ६७ X ६० = ४०२० + ३० = ४०५० = कलाः । ततश्च ज्ञोनन्तो (६००)ऽत्र तु हारकः ।
 ६००)४०५०(६
 ३६००
 ४५०
 
ततश्चन्द्रेच्चं हापयेत् - 
मद्ध्यमचन्द्र - ०१ - ०३  - २७ - १८ 
चन्द्रोच्चम् -  ०८ - २५ - ५९ - ३६ 
स्वोच्चोनचन्द्र ०८ - ०७ - २७ - ४२ (चन्द्रकेन्द्रम् )
अस्यानोजपादत्वादनागतः पादो भुजः-
०६ - ०० - ०० - ००-
०४ - ०७ - २७ - ४२
०१ - २२ - ३२ १८ भुजांशः 
भुजांशस्य कलात्वेन परिवर्त्तने - ३० + २२ = 
५२ X ६० = ३१२० + ३२ = ३१५२ कलाः ।
ततः पूर्ववदेव ``शुभाङ्गपरिमाणेन-" इत्याद्युक्तसंस्कारः -
२२५)३१५२(१४
		२२५
		९०२
		९००
			२
तथाच पञ्चदशी वर्त्तमानज्या (धमीहरिः )	- २८५१
चतुर्द्दशी अतीतज्या (दूरसरः) 					- २७२८ 
														१३१
१३१ X २ 
२२५) २६२ (१
		२२५
२७२८
	 १ 
२८२९
ततः ``गोसद्भयां वर्द्धिताः क्रमात् । अजलब्धकलाः स्वस्वगोळयोः सऋणं धनम् " इत्युक्तः संस्कारः -
२७२९ X ७
८०) १९१०३ (२३८ कलाः						(३भा. ५८क. ४७विक.)
	  १६० =
	  ३१०
	  २४०
	  ७०३
	  ६४०
	  ६३ X ६० 
८०)३७८०( ४७ । विकलाः 
	३२०
	५८०
	५६०
	२०

मेषादित्वात् दोर्ज्याफलसंस्कृतात् मद्ध्यमाद् राश्यादेरेतद्धापनीयम् -
०१ - ०३ - २७ - १८ -
	   ०३ - ५८ - ४७
०० - २९ - २८ - ३१  चन्द्रस्फुटम् ।

नवभार्ये (१४४०) ति वाक्यानि 
ज्ञोनन्तो (६००)ऽत्र तु हारकः ।
तद्दोर्ज्यालिप्तिका भानौ 
युक्त्वा त्यक्त्त्वाऽथ गोळयोः ॥ १८ ॥

\footnote{}अर्क्कपूज्यादिवाक्योक्त -
दोर्विलिप्ताश्चरोदिताः ।
अर्क्कपूज्यः (१११०) सुधाकरः (२१९७) 
रतिक्रीडो (३२६२) नुतः प्रभुः (४२६०) ॥ १९ ॥

षड्ज्यार्द्धमतीतम् । ततः सप्तम्या वर्त्तमानज्यया अतीतज्यां हापयेत् - 
१३४७ -
१२३७ 
  ११०

३४५० X ११० = १६५ = ८२ ॥
		६००         	२
		
१२३७ +	
	८३
१३२० कलाः

६०)१३२०( २२ भागाः
	१२०
	१२०
	१२० 

तदेतज्ज्याफलं सूर्यस्फुटे योजनीयम् -़
०९ - १८ - १९ -  ०६ +
	   २२
१० - १० - १९ - ०६
तदेतदुक्तम् - ``तद्दोर्ज्यालिप्तिका भानौ युक्त्वा त्यक्त्वाऽथ गोळयोः" इति । युक्त्वा अथवा त्यक्त्वेत्यर्त्यः । ब्यवस्थितविकल्पोऽयम् । पूर्ववन्मेवादित्वे त्यक्त्वा, तुलादित्वे युक्त्वा इत्यर्त्थः ।
१ अस्य पुनरर्क्कपूज्यादिवाक्येन चरदलसंस्कारः करणीय इत्याह - अर्क्केत्यादि । अत्रोत्तरार्द्धे ``अर्क्कः पूज्यः" इत्यपि कोशेषु पठ्यते । उभयथाऽपि न दोषः । उत्तरश्लोके ``स्मरादितः" इत्यपि पाठः । स्मराजितः इति क्वचित् पाठः प्रामादिकः । तथाचैवं क्रमेण नवज्या.-  
१) अर्क्कपूज्यः (१११०)
२) सुधाकारः (२१९७)
३) रतिक्रीडः (३२६२)
४) नुतः प्रभुः (४२६०)
५) अलंकृष्णः (५१३०)
६) हितोद्देशः (५८६८)
७) गतिभूतः (६४६३)ट
८) स्मरार्द्दितः (६८२५)
९) शशिधाता (३९५५)

तथाचायं क्रमः -
अयनांशसंस्कृतं सूर्यस्फूटम् - १० - १०  - १९  - ०६
अस्यानोजपादत्वादनागतो भागो भुजः -
१२ - ०० - ०० - ००
१० - १० - १९ - ०६		कलात्वेन परिवर्त्तने ३० +
०१ - १९ - ४० - ५४		१९ = ४९X६० = २९४०
ज्ञोनन्तोऽत्र तु हारकः -	+४० = २९८० कलाः ।
६००)२९८०( ४
	   २८००
	   ५८०
चतस्रो ज्या अतीताः । तथाच पञ्चमी ज्या (अलंकृष्णः)- ५१३० विलिप्ताः ।
	
अलङ्कृष्णो (५१३०) हितोद्देशो (५८६८)
गतिभूतः (६४६३) स्मरार्दितः (६८२५) ।
शशिधाते (६९५५)ति वाक्यानि 
ज्ञोनन्तो (६००)ऽत्र तु हारकः ॥ २० ॥
\footnote{}तदेकदिनगा लिप्ता ग्रहाणां स्वस्वभुक्तयः ।
\footnote{}चरार्द्धात् स्वस्वभुक्तिघ्नादनन्ताङ्ग (३६००) हृताः कलाः ।
अत्र विलिप्तापदेन गुर्वक्षराणि ग्राह्याणीति भाति । षट्टिर्ग्गुर्वक्षराणि एका विनाडिकेति अस्य षष्ट्या विभागे चरविनाडिकालाभः -
६०)५१३०( ८५ ॥ विनाडिकाः
	४८०
	३३०
	३००
	३०
यद्यपि सर्वत्र कोशेषु - ``विलिप्ताः" इत्येव । पठ्यते । तथाऽपि गणितसौष्ठवाय गुर्वक्षरग्रहणेन गणितक्रमोऽस्मभिः प्रदर्शितः । तथाऽन्यान्यदेशेषु तत्तदक्षांशभेदेन भिद्यमानश्चरदलसंस्कारोऽप्यस्य संक्षेपशास्त्रत्वान्न ग्रन्थकृता प्रदर्शित इति भाति ।
१ अथ चरदलसंस्कारार्त्थं मद्ध्येऽपेक्षितं रविचन्द्रगातिस्फुटमाह - तदेकदिनगा । अत्र ``स्वस्वभूतयः" इति क्वचित् कोशपाठः प्रमादिकः । एकस्मिन् दिने यावती सूर्यस्य स्फुटगतिः सैव सूर्यभुक्तिः । यावती चन्द्रस्य सा चन्द्रभुक्तिरिति ।
तथाचैकस्मिन् दिने रविगतिः - ५९ कलाः ८ विकलाः
एकस्मिन् दिने चन्द्रगतिः 	- ७९० 	" 	३५	 " 
मद्ध्यमा गतिरियम् । स्फुटगतिविधिः सङ्क्षेपशास्त्रत्वान्नात्र विवृतः । अथापि पूर्वोत्तरदिनयोः स्फुटयोरन्तरमेव स्फुटगतिर्भवति ।
तथाहि दशमीदिने रविस्फुटम् -  ०९ -  १९ - २० - १२ - (स्फुटप्रक्रिया च नवमीरविस्फटवदेव 
		नवमीदिने रविस्फुटम् -	  ०९ - १८ - १९ - ०६ करणीयेति स्फुटमिति न पृथग्विलिख्यते )
	स्फुटगतिः - (रविभुक्तिः) -   ०० - ०१ - ०१ - ०६
कलात्वेन परिवर्त्तने ६१ कलाः ६ विकलाः । तथाच तदेकदिनगाः ६१/१० लिप्ताः तद्दिनभुक्तिः । 
एवं दशमीदिने चन्द्रस्फुटम् -	०१ - १३ - २६ - ४२ -
	नवमीदिने चन्द्रस्फुटम् -	०० - २९ - २८ - ३१ 
	स्फुटगतिः (चन्द्रभुक्तिः)-   ०० - १३ - ५८ - ११
कलात्वेन परिवर्त्तेने १३X६० = ७८० +५८ = ८३८ कलाः ११ विकलाः तदेकदिनगा चन्द्रभुक्तिः ।
२ इदानीं चरदलसंस्कारमाह - चरार्द्धात् । अत्र ``स्वस्वभूतिघ्नात्" इति क्रचित्पाठः प्रामादिकः । चरार्द्धस्य स्वस्वभुक्त्या गुणनम् । एकस्मिन् दिने अनन्ताङ्ग (३६००) मिता विनाडिका इति तावतीभिः संख्याभिर्ल्लब्धं विभ
गतास्तु नााडका ज्ञयाः शिष्टा ..................................
\footnote{}एकाऽतिद्वादशीवृद्धौ नो चेद् वृद्धौ तु षोडश । 
द्वयेकलिप्ती समे ह्रासे चतुष्कादुत्तरं त्विदम् ॥ २५ ॥
१ ततः पूर्वोक्तसंस्कारसंस्कृतात् चन्द्रस्फुटात् सूर्यस्फुटं त्यजेत् -
तथाच संस्कृतं चन्द्रस्फुटम् - ०० - २९ - ३८ - ३१ 		कलात्वेन परिवर्त्तने -
		" 			रविस्फुटम् - ०९ - १८ - १९ - ५०		३X३० = ९० + ११ = १०१ X
									०३ - ११ - १८ - ८१		६० = ६०६० + १८ = ६०७८ कलाः ।
२ इतः परं तिथिघटिकानिष्पादनप्रकारप्रदर्शकः सार्द्धः श्लोको न स्फुटाभिप्रायः । एतावत्पर्यन्तं स्फुटविधानं निरूप्यात्रान्ते मद्ध्यममानेन तिथिनिर्ण्णयवर्ण्णनं किमाशयमिति न ज्ञायते । कथमिदं सङ्गम्येतेति महान्तः प्रमाणम् । निष्कृष्टघटिकालाभार्त्थं स्फुटविधानमेव ह्यत्राप्यपेक्षितम् । तथाचेयमत्रापेक्षिता गणितप्रक्रिया -
एकस्मिंन् तिथौ सूर्यस्य चन्द्रस्य च गत्यन्तरं ७२० कला इति तावत्या सङ्ख्यया विभागे समतीतास्तिथयो लभ्यन्ते - 
७२०)६०७८( ८ गतास्तिथयः 
	   ५७६०
	   ३१८
तथाच नवमी तिथिरित्यायातम् । नवमीतिथौ चन्द्रेणाक्रान्ताः कलाः ३१८ इति च स्थितम् । तथाचास्मिन् तिथौ चन्द्रेणेतः परमाक्रन्तव्याः कलाः - ७२० - ३१८ = ४०२ इत्यायातम् । एतद्दिने ८३८ कलाः चन्द्रभुक्तिः, ६१ कलाः रविभुक्तिरिति चोक्तम् । तयोरन्तरमेव हि तद्दिने षष्टिघटिकासु स्फुटं गत्यन्तरम् - ८३८ - ६१ = ७७७ कलाः । तथाच षष्टिघटिकासु ७७७ कलाश्चेत्  ४०२ कलातिक्रमणार्त्थँ कियत्यो घटिका अपेक्षिता इति निर्ण्णये नवमीतिथेर्न्निष्कृष्टघटिकालाभः -
४०२ X ६०	=	४०२ X ६०  						३३ X ६०
	७७७		७७७)२४१२०( ३१ घटिकाः	७७७)१९८०( २ विघटिकाः
					   २३३१							   १५५४
						८१०								४२६
						७७७
						  ३३
तथाच तद्दिने नवमीतिथिः ३१ घटिकापर्यन्तमिति सर्वं सुस्थम् ॥
३ इदानीं तिथिशुद्धौ सत्यां उपोषणशुद्धये वेधनिर्ण्णयमाह - एका । तिथीनामतिवृद्धिर्वृद्धिः समत्वं ह्रास इति चतुष्टयी गतिर्भवति । यदि दशम्येकादशीद्वादश्यः उत्तरोत्तरं वर्द्धन्ते, सा च वृद्धिः चतुःपञ्चषड्घटिकात्मिका, तदाचेत् । लब्धाः कलाः प्रातश्चेत् स्वस्वस्फुटात् हापयेत् । सायं चेद् योजयेत् । इदमुत्तरगोळे । दक्षिणे चेत् प्रातर्द्धनं सायमृणमिति विपर्ययेणेति ।
तथाचायं क्रमः - ८५ ॥ चरविनाडिका इत्युक्तम् । तथाच चरार्द्धम् ४२ ॥ भवति । तस्य सूर्यभुक्त्या गुणनम् । अनन्ताङ्गेन (३६) विभागश्च -
चरार्द्धम् X लधिभुक्तिः = ४२.७५ X ६१.१	४२.७५ X ६१.१
	अनन्ताङ्गम् 					३६००				४.२७५
														४२.७५
														२५६५.
														२६१२.०२५
विकलात्वेन वरिवर्तने -
२६१२.०२५X६०
१५६७२१.९
 
३६००)१५६.७२१.५( ४४ विकलाः
		१४४००
		१२७२१.५
		१४४००
तदेता विकलाः सूर्वस्फुटे योचनीयाः - ०९ - १८  - १९ - ०६ सूर्यस्फुठम् 
																	४४
											 ०९ - १८ - १९ - ५० चरदलसंस्कृतं सूर्यस्फुटम् 
चन्द्रेऽप्येवं संस्कारः कर्तव्यः -
चरार्द्धम् X चन्द्रभुक्तिः =	४२.७५ X ८३८
	अनन्ताङ्गम् 					३६००

तदेताः कलाः चन्द्रस्फुटे योजनीयाः -
०० - २९ - २८ - ३१ चन्द्रस्फुटम् ।
			 १०
०० - २९ - ३८ - ३१ चरदलसंस्कृतं चन्द्रस्फुटम् 
१ ``देशान्तरे दोर्बिवरसंस्कार-" इति कचित्पाठः । श्लोकार्त्यः स्फुटः ।

४२.७५ X ८३८
३४२
१२८२.५
३४२००
३६००)३५८२४.५( १० कलाः 
		३६०००
\footnote{}उदयप्यापिनी दर्शा पौर्ण्णमासी तु यामिका ।
मद्ध्याह्रव्यापिनी श्रोणा उपोष्या विष्णुतत्परैः ॥ २६ ॥

उपवासफलप्रेप्सुर्ज्ज्रह्याद् \footnote{}ङुक्तचतुष्टयम् ।
पूर्वापरे तु सायाह्रे सायम्प्रातस्तु मद्धयमे ॥ २७ ॥

\footnote{}कलार्द्धं द्वादशीं दृष्ट्वा निशीथादूर्ध्वमेव तु ।
आमद्ध्याह्नाः क्रियाः सर्वाः कर्त्तत्र्याः शम्भुशासनात् ॥ २८ ॥

इति श्रीमदानन्दतीर्थभगवत्पादविरचितः तिथिनिर्ण्णयः समाप्तः 

सा श्रुतिवृद्धिरित्युच्यते । तद् यथा - ४५ वटिकाः दशनी, ५०घटिकाः एकादशी, ५५ घटिकाः द्वादशी - इत्यादि । यदि सा वृद्धिरेकद्वित्रिघटिकात्मिका तदा केकलं वृद्धिः साम्यवृद्धिरिति चोच्यते । तद् यथा - २३घटिकाः दशमी, २४घटिकाः एकादशी, २५घटिका द्धादशी इत्यादि । यदा न वृद्धिर्न्न ह्रासस्तदा साम्यम् । तद्यथा - ४६ ॥ घटिकाः दशमी, ४७ घटिकाः एकादशी, ४६ ॥ घटिकाः द्वादशी इत्यादि । तिथीनामुत्तरोत्तरं ह्रासो ह्रासः । तद् यथा - ५५ घटिकाः दशमी, ५० घटिकाः एकादशी, ४० घटिकाः द्वादशी इत्यादि । तथाचायं श्लोकार्त्थः - अतिवृद्धौ नो चेत् अतिवृद्धावसत्यां एकद्विविकघटिकात्मिकायां समवृद्धाविति यावत् । चतुष्कादुत्तरं , `चतस्रो घटिकाः प्रातः' इत्युक्त - घटिकाचतुष्टयात् प्राक् एका घटिका दशमीवेधरहिता चेदुपोष्या भवत्येकादशी । ततश्च वृद्धिपक्षे ५५ घटिकातः परं दशमीसद्भावे सैकादशी विद्धा भवति । `वृद्धौ तु षोडश' इत्यपरं वाक्यम् । वृद्धौ अतिवृद्धौ चतुरादिघटिकावृद्धाविति यावत् , घटिकाचतुष्टयात् प्राक् षोडशलिप्तयः दशमीवेधवर्जिताश्चेदेकादश्युपोष्या भवति । द्वादश लिप्तय एका घटिका । पञ्चविघटिका एका लिप्तिः । तथाच षोडश लिप्तय इति एका घटिका विंशतिर्विघटिकाः । एवञ्च चतुरादिघटिकावृद्धइपक्षे ५४घ.  ४०विघटिकातः परं दशमीस्पर्शे सैकादशी विद्धा भवति । समतिथिपक्षे द्वे लिप्ती, दशविघटिकाः । इदमपि घटिकाचतुष्टयात् प्राक् । तथाच साम्यपक्षे ५५घ. ५० विघटिकातः परं दशमीस्पर्शे सैकादशी विद्धा भवति । तिथिह्रासे तु चतुष्कादुत्तरं एका लिप्तिः । पञ्च विघटिका इति यावत् । तथाच ह्रासपक्षे ५५ घ. ५५ विघटिकातः परं दशमी..शं सैकादशी विद्धा भवतीति ।
१ इदानीं विष्णुपञ्चकोजपवीसिनां व्यवस्थामाह - उदयव्यापिती । दर्शा अमावास्य उदयव्यापिनी चेदुपोष्या भवति । ``एकत्रस्थितचन्द्रार्क्कदर्शनाद् दर्श उच्यते" इति प्राचीनम् । तिथिशब्दस्योभयलिङ्गत्वाद् दर्शस्तिथिः दर्शा तिथिरित्युभयथाऽपि भवति । पौर्ण्णमासी तु यामिकां यामव्यापिनी ग्राह्य । याम इति ७ ॥ घटिकाः । श्रोणा श्रवणेत्यकोऽर्त्थः । भागवतादिषु प्रयोगात् ।
२ ``भक्तचतुष्टयम्" इति पाठान्तरम् । भुक्तमिति भोजनम् ।
३ इदानीं साधनद्वांदश्यां विधानमाह - कैलार्द्धम् । २॥ विघटिकाः इति यावत् । तदानीं द्वादशीव्रतलोपनिवारणाय निशीथादूर्ध्वमेव आरभ्य सर्वाः क्रियाः कर्त्तव्याः । सूर्योदयात् परं आमद्ध्याह्नात् करणीया अपि तस्मिन् दिने सूर्योदयात् पूर्वमेवार्द्धरात्रात्परमारभ्य कृताश्चेन्न प्रत्यवायः, प्रत्युत द्वादशीव्रतनिष्ठानिमित्तं पुण्यमेव प्राप्नोतीत्यर्त्थः । 
